\documentclass[a4paper]{article}
\usepackage[margin=1in]{geometry}

\usepackage{showkeys}
\usepackage{clan-macros}
\usepackage{preamble}

\title{Clans}
\author{Frank Tsai}

\addbibresource{bib.bib}

\setlist[enumerate, 1]{% global settings for the first level in enumerate environments
  leftmargin = \parindent, % item text indentation
  align = left,
  labelwidth=\parindent,
  labelsep = *
}

\begin{document}
\maketitle

Personal notes on~\cite{Frey25}.

\section{Clans, models of clans, and a weak factorization system on models}

\subsection{The definition of clans}
\begin{definition}
  A \emph{clan} is a small category $\cT$ equipped with a class $\cT_{\dag}$ of distinguished morphisms called \emph{display maps}, subject to the following conditions:
  \begin{enumerate}
  \item the pullback of a display map $p : \Gamma' \display \Gamma$ along an arbitrary map $s : \Delta \to \Gamma$ exists, and is a display map
    \begin{center}
      \DiagramSquare{
        nw = {\Delta'},
        nw/style = {pullback},
        ne = {\Gamma'},
        se = {\Gamma},
        sw = {\Delta},
        north = {s'},
        east = {p},
        east/style = {display},
        south = {s},
        west = {p'},
        west/style = {display},
      }
    \end{center}
  \item all isomorphisms are display maps;
  \item display maps are stable under composition;
  \item $\cT$ has a terminal object, and the unique morphism $\Gamma \to \termObj{\cT}$ is a display map for all $\Gamma \in \cT$.
  \end{enumerate}
\end{definition}

\begin{definition}
  Let $\cT$ and $\cQ$ be clans.
  A \emph{morphism of clans} $F : \cT \to \cQ$ is a functor $F : \cT \to \cQ$ from the underlying category of $\cT$ to the underlying category of $\cQ$ preserving display maps, pullbacks of display maps, and the terminal object.
  Clans, clan morphisms, and natural transformations between them assemble into a 2-category $\Clan$.
\end{definition}

\begin{example}
  Let $\cT$ be a small finitely complete category.
  There is a canonical clan structure, called the finite limit clan structure, on $\cT$: every morphism is a display map.
\end{example}

\begin{example}
  Let $\cT$ be a small category with finite products.
  The finite product clan structure on $\cT$ consists of all product projections as display maps.
\end{example}

\begin{remark}
  Although a clan is \emph{defined} to be a small category, we consider some \emph{large} clans to simplify some definitions.
\end{remark}

\subsection{Models and representable models}
\begin{definition}
  Let $\cT$ be a clan.
  A \emph{model} of $\cT$ is a morphism of clans $\cT \to \Set$, where $\Set$ is equipped with the finite limit clan structure.
\end{definition}

\begin{remark}
  For any two models $F, G : \cT \to \Set$, we consider any natural transformation as a homomorphism of models; hence the category of models $\Mod(\cT)$ is the full subcategory of the functor category $\funCat{\cT}{\Set}$ spanned by models.
\end{remark}

\begin{remark}
  Let $\cT$ be a clan.
  The category $\cT$ has a canonical limit sketch structure.
  Hence, as a category of models of a limit sketch, $\Mod(\cT)$ is locally finitely presentable, and the inclusion functor $\Mod(\cT) \into \funCat{\cT}{\Set}$ creates filtered colimits.
\end{remark}

\begin{remark}
  Let $\cT$ be a clan.
  For any object $\Gamma \in \cT$, the representable functor $\Hom{\cT}(\Gamma,-) : \cT \to \Set$ preserves all limits in $\cT$; hence it is a model of $\cT$.
  This means that the Yoneda embedding $\yon : \cT\op \to \funCat{\cT}{\Set}$ restricts along the inclusion as follows:
  \begin{center}
    \begin{tikzpicture}[diagram]
      \node (1) {$\cT\op$};
      \node (2) [right = of 1] {$\funCat{\cT}{\Set}$};
      \node (3) [above = of 2] {$\Mod(\cT)$};
      \draw [embedding] (1) to node [swap] {$\yon$} (2);
      \draw [embedding] (3) to node {} (2);
      \draw [exists,->] (1) to node {$H$} (3);
    \end{tikzpicture}
  \end{center}
\end{remark}

\begin{lemma}
  Let $\cT$ be a clan.
  For any object $\Gamma \in \cT$, the representable model $H(\Gamma)$ is finitely presentable.
\end{lemma}
\begin{proof}
  By the Yoneda lemma, the functor $\Hom{\Mod(\cT)}(H(\Gamma),-) : \Mod(\cT) \to \Set$ is isomorphic to the evaluation functor at $\Gamma$.
  Since filtered colimits in $\Mod(\cT)$ are computed pointwise, this functor preserves filtered colimits.
\end{proof}

\begin{definition} 
  A \emph{comodel} of a clan $\cT$ in a cocomplete category $\fX$ is a functor $F : \cT\op \to \fX$ sending $\initObj{\cT\op}$ to $\initObj{\fX}$ and display-pushouts in $\cT\op$ to pushouts in $\fX$.
  We write $\CoMod_{\fX}(\cT)$ for the 2-category of comodels in $\fX$.
\end{definition}

\begin{proposition}
  Let $\cT$ be a clan.
  For any cocomplete category $\fX$, precomposition with $H$ gives rise to an equivalence
  \[
    \CoCont(\Mod(\cT),\fX) \eqv \CoMod_{\fX}(\cT).
  \]
\end{proposition}
\begin{proof}
  The precomposition functor $- \circ H : \CoCont(\Mod(\cT),\fX) \to \CoMod_{\fX}(\cT)$ is fully faithful because $H$ is.
  For any comodel $F : \cT\op \to \fX$, we show that there is a continuous functor $G : \Mod(\cT) \to \fX$ such that $G \circ H \iso F$.
  We take $G$ to be the Yoneda extension of $F$ along $H$; then clearly $G \circ H \iso F$.
  It remains to show that $G$ is cocontinuous.
  To this end, we show that $G$ is a left adjoint.
  Since $F$ is a comodel, the functor $\fX(F(-),X) : \cT \to \Set$ is a model for all $X \in \fX$.
  Hence the nerve functor lands in $\Mod(\cT)$ and the usual realization-nerve adjunction restricts to the desired adjunction.
\end{proof}

\subsection{A weak factorization system}
\begin{definition}
  Let $\cT$ be a clan.
  \begin{enumerate}
  \item A map $f : A \to B$ in $\Mod(\cT)$ is \emph{full} if it has the right lifting property against all maps $H(p)$ for $p$ a display map.
  \item A map $f : A \to B$ in $\Mod(\cT)$ is an \emph{extension} if it has the left lifting property against all full maps.
  \item A model $A \in \Mod(\cT)$ is a \emph{$0$-extension} if the unique morphism $\initObj{\Mod(\cT)} \to A$ is an extension.
  \end{enumerate}
\end{definition}

\begin{remark}
  Essentially, full maps are regular epimorphisms and $0$-extensions are projective objects.
\end{remark}

\begin{remark}
  Let $\cT$ be a clan, and let $G = \{H(p) \mid p \in \cT_{\dag}\}$.
  The class of full maps is precisely $\rLift{G}$, and the class of extensions is precisely $\lLift{(\rLift{G})}$.
  By Quillen's small object argument, $(\lLift{(\rLift{G})},\rLift{G})$ is a weak factorization system on $\Mod(\cT)$.
\end{remark}

\begin{notation}
  We use the notation $\extension$ for extensions.
\end{notation}

\begin{remark}
  Let $\cT$ be a clan.
  For every display map $p : \Delta \display \Gamma$, $H(p) : H(\Gamma) \extension H(\Delta)$ is an extension.
  In particular, $H(\Gamma)$ is a $0$-extension for all $\Gamma \in \cT$ since $\Gamma \display \termObj{\cT}$ is always a display map.
\end{remark}

\section{$(E,F)$-categories and the biadjunction}
For any clan $\cT$, we can equip $\Mod(\cT)$ with a weak factorization system.
To make this association functorial, we define $(E,F)$-categories and their morphisms.
We would like to think of a morphism of clans $F : \cT \to \cQ$ as an interpretation of $\cT$ in $\cQ$.
Since the functor $F$ is a morphism of clans, the precomposition functor $- \circ F : \funCat{\cQ}{\Set} \to \funCat{\cT}{\Set}$ restricts along the inclusion to a functor $- \circ F : \Mod(\cQ) \to \Mod(\cT)$.

\begin{remark}
  A map $f : A \to B$ in $\Mod(\cT)$ is full if and only if the naturality square
  \begin{center}
    \DiagramSquare{
      nw = {A(\Delta)},
      ne = {A(\Gamma)},
      se = {B(\Gamma)},
      sw = {B(\Delta)},
      north = {A(p)},
      east = {f_{\Delta}},
      south = {B(p)},
      west = {f_{\Gamma}},
    }
  \end{center}
  is a weak pullback in $\Set$ for all display maps $p : \Delta \display \Gamma$.
\end{remark}

\begin{remark}
  In a finite product clan, full maps are regular epimorphisms and $0$-extensions are regular projective objects.
\end{remark}

\begin{lemma}
  The (restricted) precomposition functor $- \circ F : \Mod(\cQ) \to \Mod(\cT)$ preserves small limits, filtered colimits, and all full maps.
\end{lemma}
\begin{proof}
  Since the precomposition functor $- \circ F : \funCat{\cQ}{\Set} \to \funCat{\cT}{\Set}$ preserves both limits and colimits, the restricted precomposition preserves those limits and colimits that are computed pointwise, \ie all small limits and filtered colimits.
  Now, let $f : A \to B$ be full in $\Mod(\cQ)$.
  The component of $f \circ F$ at an arbitrary display map $p$ is precisely $f_{F(p)}$.
  Since $F$ preserves display maps, the naturality square of $f \circ F$ at any $p$ is a weak pullback, \ie $f \circ F$ is full.
\end{proof}

\begin{definition}
  An $(E,F)$-category is a locally finitely presentable category $\cL$ with a weak factorization system $(E,F)$ whose maps are called \emph{extensions} and \emph{full maps}.
  A morphism of $(E,F)$-categories is a functor $F : \cL \to \cQ$ preserving small limits, filtered colimits, and full maps.
  We write $\EFCat$ for the 2-category of $(E,F)$-categories, morphisms, and natural transformations.
\end{definition}

\begin{construction}
  The assignment $\cT \mapsto \Mod(\cT)$ extends to a 2-functor $\Mod : \Clan\op \to \EFCat$ as follows:
  \begin{itemize}
  \item $\cT \mapsto \Mod(\cT)$ equipped with the canonical weak factorization system cofibrantly generated by $\{H(p) \mid p \in \cT_{\dag}\}$;
  \item $F : \cT \to \cQ \mapsto - \circ F : \Mod(\cQ) \to \Mod(\cT)$.
  \end{itemize}
\end{construction}

\begin{lemma}
  If $F : \cL \to \cQ$ is a morphism of $(E,F)$-categories, then it has a left adjoint $L : \cQ \to \cL$ which preserves finitely presentable objects and extensions.
  Conversely, if $L : \cQ \to \cL$ is a cocontinuous functor preserving extensions and finitely presentable objects, then $L$ admits a left adjoint $F : \cQ \to \cL$ which is a morphism of $(E,F)$-categories.
\end{lemma}
\begin{proof}
  Since $F$ preserves limits and filtered colimits, it has a left adjoint by the Adjoint Functor Theorem for locally presentable categories.
  If $q \in \cQ$ is a finitely presentable object, then since $\Hom{\cL}(Lq,-) \iso \Hom{\cQ}(q,F-)$ and $F$ preserves filtered colimits, $Lq$ is finitely presentable.
  Finally, suppose that $f : A \extension B$ in $\cQ$ and let the following square be an arbitrary commuting square:
  \begin{center}
    \DiagramSquare{
      nw = {LA},
      ne = {C},
      se = {D},
      sw = {LB},
      west = {Lf},
      east/style = {->>},
    }
  \end{center}
  Since $F$ preserves full maps, transporting this square along the adjunction yields a commuting square of an extension against a full map.
  Then the left lifting property gives a map $B \to FC$ making the two triangles commute; hence the desired lifting $LB \to C$ is given by transposition.

  Conversely, if $L : \cQ \to \cL$ is cocontinuous and preserves extensions and finitely presentable objects, then it admits a right adjoint by the usual Adjoint Functor Theorem.
  By the Yoneda lemma, to show that $F$ preserves filtered colimits, it suffices to show that $\Hom{\cQ}(q, F(\colim\cD)) \iso \Hom{\cQ}(q, \colim F\cD)$ naturally in $q$.
  Since locally finitely presentable categories admit a dense set of generators comprised of finitely presentable objects, it suffices to assume that $q$ is finitely presentable.
  Then the rest of the argument follows by standard abstract nonsense.
  Finally, since $L$ preserves extensions, $F$ preserves full maps by abstract nonsense.
\end{proof}

In the opposite direction, we would like to associate each $(E,F)$-category $\cT$ with a clan.
An evident guess is to send each $(E,F)$-category $\cT$ to the category of left adjoints from $\Set$ to $\cT$ preserving finitely presentable objects and extensions.
The problem is that we do not know how to equip this category with a class of display maps, but observe that the category described above is equivalent to the full subcategory of $\cT$ spanned by finitely presentable $0$-extensions.

\begin{definition}
  Given an $(E,F)$-category $\cL$, we write $\fC(\cL) \subseteq \cL$ for the full subcategory spanned by finitely presentable $0$-extensions.
\end{definition}

\begin{lemma}
  Let $\cL$ be an $(E,F)$-category.
  $\fC(\cL)\op$ equipped with extensions as display maps is a clan.
\end{lemma}
\begin{proof}
  We show that $\fC(\cL)\op$ has pullbacks against any display map and that display maps are stable under pullback; all other conditions are trivial.
  To this end, suppose that $\Gamma, \Gamma'$, and $\Delta$ are finitely-presentable 0-extensions, and that $p : \Gamma \extension \Gamma'$ is an extension and $s : \Gamma \to \Delta$ is an arbitrary morphism in $\cL$.
  We consider the following pushout:
  \begin{center}
    \DiagramSquare{
      nw = {\Gamma},
      ne = {\Gamma'},
      se = {\Delta'},
      se/style = {pushout},
      sw = {\Delta},
      north = {p},
      north/style = {extension},
      west = {s},
      south/style = {extension},
    }
  \end{center}
  Note that extensions are stable under pushout; hence it remains to show that $\Delta'$ is a finitely-presentable 0-extension.
  It is finitely presentable because $\Gamma'$ and $\Delta$ are and filtered colimits commute with finite limits.
  It is a 0-extension because the unique map $\initObj{\cL} \to \Delta'$ can be factorized as $\initObj{\cL} \extension \Delta \extension \Delta'$, and extensions are stable under composition.
\end{proof}

\begin{construction}
  Let $f : \cL \to \cQ$ be a morphism of $(E,F)$-categories.
  It admits a left adjoint $L : \cQ \to \cL$ preserving finitely presentable objects and extensions; hence $L$ restricts to a morphism of clans $\fC(\cQ)\op \to \fC(\cL)\op$, extending the object assignment $\fC(-) : \EFCat \to \Clan\op$ to a pseudofunctor.
\end{construction}

\begin{lemma}
  Let $\cL$ be an $(E,F)$-category.
  The inclusion functor $\fC(\cL) \into \cL$ is a comodel.
\end{lemma}

\begin{construction}
  For any $(E,F)$-category $\cL$, since the inclusion $j : \fC(\cL) \into \cL$ is a comodel, the nerve of $j$ is a functor $\eta_{\cL} : \cL \to \Mod(\fC(\cL)\op)$.
\end{construction}

\begin{construction}
  For any clan $\cT$, the restricted Yoneda embedding $H : \cT\op \to \Mod(\cT)$ further restricts to $H' : \cT\op \to \fC(\Mod(\cT))$ since $H(\Gamma)$ is a finitely presentable 0-extension for all $\Gamma \in \cT$.
  Taking the opposite categories yields $\varepsilon_{\cT} : \cT \to \fC(\Mod(\cT))\op$.
\end{construction}

\begin{theorem}
  $\fC(-)\op$ is left biadjoint to $\Mod(-)$.
  Hence there is an equivalence between hom-categories.
  \[
    \Clan(\cT,\fC(\cL)\op) \eqv \EFCat(\cL,\Mod(\cT))
  \]
\end{theorem}

\section{Cauchy complete clans}

\begin{definition}
  A clan $\cT$ is \emph{Cauchy complete} if the underlying category is Cauchy complete and retracts of display maps are display maps.
\end{definition}

\begin{remark}
  If $\cC$ is Cauchy complete, then retracts of representable functors in $\funCat{\cC\op}{\Set}$ are representable.
\end{remark}

\begin{remark}
  For every $(E,F)$-category $\cL$, the clan $\fC(\cL)\op$ is Cauchy complete.
\end{remark}

\begin{definition}
  A model $A : \cT \to \Set$ is \emph{flat} if $\El(A)$ is filtered.
\end{definition}

\begin{lemma}
  A model $A : \cT \to \Set$ is flat if and only if it is a filtered colimit of representable models.
\end{lemma}
\begin{proof}
  By density of representable functors, we have $A \iso \colim(\El(A) \to \cT\op \xto{H} \Set)$.
  If $A$ is flat, then $A$ is clearly a filtered colimit of representables.
  The other direction follows from the fact that representable functors are flat and flat functors are closed under filtered colimits.
\end{proof}

\begin{corollary}
  For any clan $\cT$, the 0-extensions in $\Mod(\cT)$ are flat.
\end{corollary}
\begin{proof}
  \todo Look up the \emph{fat small object argument}.
\end{proof}

\begin{theorem}
  If $\cT$ is a Cauchy complete clan, then $\varepsilon_{\cT} : \cT \to \fC(\Mod(\cT))\op$ is an equivalence of clans.
\end{theorem}
\begin{proof}
  \todo Redo this proof.
  For all intents and purposes, $\varepsilon_{\cT}$ is the (restricted) Yoneda embedding.
  Hence it is fully faithful.
  We show that it is essentially surjective on objects.
  Let $C \in \Mod(\cT)$ be a finitely-presentable 0-extension.
  Then $C$ is a filtered colimit of representables; hence finite presentability implies that $C$ is a retract of some representable functor.
  Since $\cT$ is Cauchy complete, $C$ is representable; hence the equivalence.

  To show that the inverse is a morphism of clans, we show that $\varepsilon_{\cT}$ reflects extensions to display maps.
\end{proof}

\section{Clan-algebraic categories}

\begin{definition}
  An $(E,F)$-category $\cL$ is called \emph{clan-algebraic} if
  \begin{enumerate}
  \item[(D)] the inclusion $j : \fC(\cL) \into \cL$ is dense;
  \item[(CG)] the weak factorization is cofibrantly generated by maps between finitely-presentable $0$-extensions;
  \item[(FQ)] equivalence relations $(p,q) : R \mono A \times A$ in $\cL$ with \emph{full} components $p, q$ are effective, and have \emph{full} coequalizers.
  \end{enumerate}
\end{definition}

\begin{lemma}
  Full maps in clan-algebraic categories are regular epimorphisms.
\end{lemma}
\begin{proof}
  Let $f : A \to B$ be a full map in a clan-algebraic category $\cL$.
  The lifting property against 0-extensions implies that for any $0$-extension $E$ and morphism $E \to B$, there is a lifting $E \to A$ making the following triangle commute:
  \begin{center}
    \begin{tikzpicture}[diagram]
      \node (1) {$A$};
      \node (2) [below = of 1] {$B$};
      \node (3) [left = of 2] {$E$};
      \draw [->>] (1) to node {$f$} (2);
      \draw [->] (3) to node {} (2);
      \draw [->,exists] (3) to node {} (1);
    \end{tikzpicture}
  \end{center}
  Hence $\eta_{\cL}(f) = j_{N}(f) : \Hom{\cL}(j-,A) \to \Hom{\cL}(j-,B)$ is pointwise surjective.
  This means that $\eta_{\cL}(f)$ is a regular epimorphism.
  Since left adjoints preserve regular epimorphism, $j_{\otimes}(j_N(f))$ is a regular epimorphism, but $j_{\otimes}j_N \iso \id{\cL}$ by density.
\end{proof}

\begin{theorem}
  If $\cL$ is clan-algebraic, then $\eta_{\cL} : \cL \to \Mod(\fC(\cL)\op)$ is an equivalence.
\end{theorem}
\begin{proof}
  By density, $\eta_{\cL}$ is fully faithful; it remains to show that it is essentially surjective.
  To this end, it suffices to show that the unit of the adjunction $j_{\otimes} \adj j_{N}$ at component $A \in \Mod(\fC(\cL)\op)$ is an isomorphism.
  Every $A$ is the colimit of a nice diagram, and $j_{\otimes}$ and $j_N$ preserve colimits of nice diagrams; hence the result.
  \todo Define \emph{nice diagrams}.
\end{proof}

\printbibliography
\end{document}

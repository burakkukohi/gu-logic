\documentclass[a4paper]{article}
\usepackage[margin=1in]{geometry}

\usepackage{showkeys}
\usepackage{clan-macros}
\usepackage{preamble}

\title{Clans}
\author{Frank Tsai}

\addbibresource{bib.bib}

\setlist[enumerate, 1]{% global settings for the first level in enumerate environments
  leftmargin = \parindent, % item text indentation
  align = left,
  labelwidth=\parindent,
  labelsep = *
}

\begin{document}
\maketitle

Personal notes on~\cite{Frey25}.

\section{Clans, models of clans, and WFS on models}

\begin{definition}
  A \emph{clan} is a small category $\cT$ equipped with a class $\cT_{\dag}$ of distinguished morphisms called \emph{display maps}, subject to the following conditions:
  \begin{enumerate}
  \item the pullback of a display map $p : \Gamma' \display \Gamma$ along an arbitrary map $s : \Delta \to \Gamma$ exists, and is a display map
    \begin{center}
      \DiagramSquare{
        nw = {\Delta'},
        nw/style = {pullback},
        ne = {\Gamma'},
        se = {\Gamma},
        sw = {\Delta},
        north = {s'},
        east = {p},
        east/style = {display},
        south = {s},
        west = {p'},
        west/style = {display},
      }
    \end{center}
  \item all isomorphisms are display maps;
  \item display maps are stable under composition;
  \item $\cT$ has a terminal object, and the unique morphism $\Gamma \to \termObj{\cT}$ is a display map for all $\Gamma \in \cT$.
  \end{enumerate}
\end{definition}

\begin{definition}
  Let $\cT$ and $\cQ$ be clans.
  A \emph{morphism of clans} $F : \cT \to \cQ$ is a functor $F : \cT \to \cQ$ from the underlying category of $\cT$ to the underlying category of $\cQ$ preserving display maps, pullbacks of display maps, and the terminal object.
  Clans, clan morphisms, and natural transformations between them assemble into a 2-category $\Clan$.
\end{definition}

\begin{remark}
  Although a clan is \emph{defined} to be a small category, we sometimes use large clans to simplify some definitions.
\end{remark}

\begin{definition}
  Let $\cT$ be a clan.
  A \emph{model} of $\cT$ is a morphism of clans $\cT \to \Set$, where $\Set$ is equipped with the finite limit clan structure, \ie every morphism is a display map.
\end{definition}

For any two models $F, G : \cT \to \Set$, we consider any natural transformation as a homomorphism of models; hence the category of models $\Mod(\cT)$ is the full subcategory of the functor category $\funCat{\cT}{\Set}$ spanned by models.

\subsection{Categorical properties of $\Mod(\cT)$}

Every clan $\cT$ is evidently a finite-limit sketch; hence $\Mod(\cT)$ is a reflective subcategory of $\funCat{\cT}{\Set}$, the inclusion functor $\Mod(\cT) \into \funCat{\cT}{\Set}$ creates filtered colimits, and in particular, $\Mod(\cT)$ is locally finitely presentable.
Since representable functors $\Hom{\cT}(\Gamma,-) : \cT \to \Set$ preserve limits, the image of the Yoneda embedding $\yon : \cT\op \to \funCat{\cT}{\Set}$ lands in $\Mod(\cT)$.
In other words, the Yoneda embedding $\yon$ restricts to a functor $H : \cT\op \to \Mod(\cT)$ in the following configuration:
\begin{center}
  \begin{tikzpicture}[diagram]
    \node (1) {$\cT\op$};
    \node (2) [right = of 1] {$\funCat{\cT}{\Set}$};
    \node (3) [above = of 2] {$\Mod(\cT)$};
    \draw [embedding] (1) to node [swap] {$\yon$} (2);
    \draw [embedding] (3) to node {} (2);
    \draw [exists,->] (1) to node {$H$} (3);
  \end{tikzpicture}
\end{center}
Moreover, $H(\Gamma)$ is finitely presentable since filtered colimits in $\Mod(\cT)$ are computed pointwise in $\funCat{\cT}{\Set}$.

\subsection{The WFS on models}

Since nonequivalent clans may have the same model, we cannot hope for a duality theorem \ala Gabriel-Ulmer without further modification.

\begin{definition}
  Let $\cT$ be a clan.
  \begin{enumerate}
  \item A map $f : A \to B$ in $\Mod(\cT)$ is \emph{full} if it has the right lifting property against all maps $H(p)$ for $p$ a display map.
  \item A map $f : A \to B$ in $\Mod(\cT)$ is an \emph{extension} if it has the left lifting property against all full maps.
  \item An model $A \in \Mod(\cT)$ is a \emph{$0$-extension} if the unique morphism $\initObj{\Mod(\cT)} \to A$ is an extension.
  \end{enumerate}
\end{definition}

\begin{remark}
  Let $E$ and $F$ be the class of extensions and full maps respectively.
  By the small object argument, $(E,F)$ is a weak factorization system.
\end{remark}

\begin{notation}
  We use the notation $\extension$ for extensions.
\end{notation}

\section{$(E,F)$-categories and the biadjunction}

\begin{definition}
  An $(E,F)$-category is a locally finitely presentable category $\cL$ with a weak factorization system $(E,F)$ whose maps are called \emph{extensions} and \emph{full maps}.
  A morphism of $(E,F)$-categories is a functor $F : \cL \to \cQ$ preserving small limits, filtered colimits, and full maps.
  We write $\EFCat$ for the 2-category of $(E,F)$-categories, morphisms, and natural transformations.
\end{definition}

\begin{lemma}
  If $F : \cL \to \cQ$ is a morphism of $(E,F)$-categories, then it has a left adjoint $L : \cQ \to \cL$ which preserves finitely presentable objects and extensions.
  Conversely, if $L : \cQ \to \cL$ is a cocontinuous functor preserving extensions and finitely presentable objects, then $L$ admits a left adjoint $F : \cQ \to \cL$ which is a morphism of $(E,F)$-categories.
\end{lemma}

\begin{lemma}
  For any morphism $F : \cT \to \cQ$ of clans, the precomposition $- \circ F : \Mod(\cQ) \to \Mod(\cT)$ is a morphisms of $(E,F)$-categories.
  Hence the assignment $\cT \mapsto \Mod(\cT)$ extends to a 2-functor $\Mod : \Clan\op \to \EFCat$.
\end{lemma}

\begin{definition}
  Given an $(E,F)$-category $\cL$, we write $\fC(\cL) \subseteq \cL$ for the full subcategory spanned by finitely presentable $0$-extensions.
\end{definition}

\section{Cauchy complete clans and the fat small object argument}

\begin{definition}
  A model $A : \cT \to \Set$ is \emph{flat} if $\El(A)$ is filtered.
\end{definition}

\begin{lemma}
  A model $A : \cT \to \Set$ is flat if and only if it is a filtered colimit of representable models.
\end{lemma}

\begin{theorem}
  If $\cT$ is a Cauchy complete clan, then $\eta_{\cT} : \cT \to \fC(\Mod(\cT))\op$ is an equivalence of clans.
\end{theorem}

There is an idempotent pseudomonad on $\EFCat$.

\section{Clan-algebraic categories}

\begin{definition}
  An $(E,F)$-category $\cL$ is called \emph{clan-algebraic} if
  \begin{enumerate}
  \item[(D)] the inclusion $j : \fC(\cL) \into \cL$ is dense;
  \item[(CG)] the weak factorization is cofibrantly generated by $E \cap \sC(\cL)$;
  \item[(FQ)] equivalence relations $(p,q) : R \mono A \times A$ in $\cL$ with \emph{full} components $p, q$ are effective, and have \emph{full} coequalizers.
  \end{enumerate}
\end{definition}

\section{Quillen's small object argument}

Let $(L,R)$ be a weak factorization system.
One can show that $L$ is closed under isomorphisms, composition, and pushouts; and $R$ is closed under isomorphisms, composition, and pullbacks.

\begin{theorem}
  Let $E$ be a set of morphisms in a locally finitely presentable category $\cC$; then $(...,...)$ is a weak factorization system on $\cC$.
\end{theorem}

\printbibliography
\end{document}

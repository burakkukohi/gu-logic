\documentclass[a4paper]{article}
\usepackage[margin=1in]{geometry}

\usepackage[final]{showkeys}
\usepackage{clan-macros}
\usepackage{preamble}

\title{Clans}
\author{Frank Tsai}

\addbibresource{bib.bib}

\setlist[enumerate, 1]{% global settings for the first level in enumerate environments
  leftmargin = \parindent, % item text indentation
  align = left,
  labelwidth=\parindent,
  labelsep = *
}

\begin{document}
\maketitle

Personal notes on~\cite{Frey25}.

\section{Introduction}
The Gabriel-Ulmer duality is a contravariant biequivalence between the $2$-category of finite limit categories and lex functors; and the $2$-category of finitely presentable categories and finitary right adjoints.
The category of graphs is finitely presentable, yet there exist two nonequivalent GAT presentations.
The mismatch stems from sort dependencies; this paper introduces a mathematical gadget to reflect this data in the semantics.

\section{Clans and their models}

\subsection{The definition of clans}
\begin{definition}
  A \emph{clan} is a small category $\cT$ equipped with a class $\cT_{\dag}$ of distinguished morphisms called \emph{display maps}, subject to the following conditions:
  \begin{enumerate}
  \item the pullback of a display map $p : \Gamma' \display \Gamma$ along an arbitrary map $s : \Delta \to \Gamma$ exists, and is a display map
    \begin{center}
      \DiagramSquare{
        nw = {\Delta'},
        nw/style = {pullback},
        ne = {\Gamma'},
        se = {\Gamma},
        sw = {\Delta},
        north = {s'},
        east = {p},
        east/style = {display},
        south = {s},
        west = {p'},
        west/style = {display},
      }
    \end{center}
  \item all isomorphisms are display maps;
  \item display maps are stable under composition;
  \item $\cT$ has a terminal object, and the unique morphism $\Gamma \to \termObj{\cT}$ is a display map for all $\Gamma \in \cT$.
  \end{enumerate}
  A \emph{morphism of clans} $F : \cT \to \cQ$ is a functor $F : \cT \to \cQ$ from the underlying category of $\cT$ to the underlying category of $\cQ$ preserving display maps, pullbacks of display maps, and the terminal object.
  Clans, clan morphisms, and natural transformations between them assemble into a 2-category $\Clan$.
\end{definition}

\begin{example}
  Let $\cT$ be a small finitely complete category.
  The finite limit clan structure has every morphism in $\cT$ as a display map.
\end{example}

\begin{example}
  Let $\cT$ be a small category with finite products.
  The finite product clan structure on $\cT$ consists of all product projections as display maps.
\end{example}

\begin{remark}
  Although a clan is \emph{defined} to be a small category, we consider certain \emph{large} clans to simplify some definitions.
\end{remark}

\subsection{Models and representable models}
\begin{definition}
  Let $\cT$ be a clan.
  A \emph{model} of $\cT$ is a morphism of clans $\cT \to \Set$, where $\Set$ is equipped with the finite limit clan structure.
  For any two models $F, G : \cT \to \Set$, we consider any natural transformation as a homomorphism of models; hence the category of models $\Mod(\cT)$ is the full subcategory of the functor category $\funCat{\cT}{\Set}$ spanned by models.
\end{definition}

\begin{remark}
  The category of models $\Mod(\cT)$ of a clan $\cT$ has a canonical limit sketch structure.
  Hence $\Mod(\cT)$ is locally finitely presentable, and the inclusion functor $\Mod(\cT) \into \funCat{\cT}{\Set}$ creates filtered colimits.
\end{remark}

\begin{remark}
  Let $\cT$ be a clan.
  For any object $\Gamma \in \cT$, the representable functor $\Hom{\cT}(\Gamma,-) : \cT \to \Set$ preserves all limits in $\cT$; hence it is a model of $\cT$.
  This means that the Yoneda embedding $\yon : \cT\op \to \funCat{\cT}{\Set}$ restricts along the inclusion as follows:
  \begin{center}
    \begin{tikzpicture}[diagram]
      \node (1) {$\cT\op$};
      \node (2) [right = of 1] {$\funCat{\cT}{\Set}$};
      \node (3) [above = of 2] {$\Mod(\cT)$};
      \draw [embedding] (1) to node [swap] {$\yon$} (2);
      \draw [embedding] (3) to node {} (2);
      \draw [exists,->] (1) to node {$H$} (3);
    \end{tikzpicture}
  \end{center}
\end{remark}

\begin{lemma}
  Let $\cT$ be a clan.
  For any object $\Gamma \in \cT$, the representable model $H(\Gamma)$ is finitely presentable.
\end{lemma}
\begin{proof}
  Write $i$ for the inclusion functor, and let $\colim D$ be a filtered colimit in $\Mod(\cT)$.
  \begin{align*}
    \Hom{\Mod(\cT)}(H(\Gamma), \colim D) &\iso \Hom{\funCat{\cT}{\Set}}(\yon(\Gamma), i\colim D) && \text{$i$ is fully faithful}\\
                                    &\iso \Hom{\funCat{\cT}{\Set}}(\yon(\Gamma), \colim iD) && \text{$i$ preserves filtered colimits}\\
                                    &\iso \colim\Hom{\funCat{\cT}{\Set}}(\yon(\Gamma), iD) && \text{tininess of representable}\\
                                    &\iso \colim\Hom{\Mod(\cT)}(H(\Gamma), D) && \text{$i$ is fully faithful}
  \end{align*}
\end{proof}

The category $\funCat{\cT}{\Set}$ is the free cocompletion of $\cT\op$; hence for any cocomplete category $\fX$, every functor $f : \cT\op \to \fX$ corresponds bijectively to a cocontinuous functor $f_{\otimes} : \funCat{\cT}{\Set} \to \fX$.
When $\fX\op$ is equipped with the maximal clan structure, we show a similar correspondence between morphism of clans $f\op : \cT \to \fX\op$ and cocontinuous functor out of $\Mod(\cT)$.

\begin{definition}
  A \emph{comodel} of a clan $\cT$ in a cocomplete category $\fX$ is a functor $f : \cT\op \to \fX$ such that $f\op : \cT \to \fX\op$ is a morphism of clans.
  We write $\CoMod_{\fX}(\cT)$ for the 2-category of comodels in $\fX$.
\end{definition}

\begin{proposition}\label{0000}
  Let $\cT$ be a clan.
  For any cocomplete category $\fX$, precomposition with $H$ gives rise to an equivalence
  \[
    \CoCont(\Mod(\cT),\fX) \eqv \CoMod_{\fX}(\cT).
  \]
\end{proposition}
\begin{proof}
  The precomposition functor $- \circ H : \CoCont(\Mod(\cT),\fX) \to \CoMod_{\fX}(\cT)$ is fully faithful because $H$ is.
  For any comodel $f : \cT\op \to \fX$, we take $g : \Mod(\cT) \to \fX$ to be the left Kan extension of $f$ along $H$; then clearly $g \circ H \iso f$ since $H$ is fully faithful.

  Observe that $g$ is the Yoneda extension of $f$ restricted to $\Mod(\cT)$.
  Since $f$ is a comodel, the functor $\fX(f(-),X) : \cT \to \Set$ is a model for all $X \in \fX$.
  Hence the nerve functor lands in $\Mod(\cT)$, and the usual realization-nerve adjunction restricts.
  Then $g$, as a left adjoint, is cocontinuous.
\end{proof}

\section{$(E,F)$-categories and the biadjunction}

On the syntactic side, display maps allow us to have fine-grained control over which limits are preserved.
On the semantic side, the category of models of any clan is a locally finitely presentable category, but this data alone cannot recover the underlying clan since there are nonequivalent clans with equivalent models.
To correct this, we place a weak factorization system on finitely presentable categories.

\subsection{A weak factorization system}

Given a display map $\Delta \display \Gamma$ in a clan $\cT$, we can intuitively think of the morphism $H(\Gamma) \to H(\Delta)$ as an embedding of models.
\begin{definition}
  Let $\cT$ be a clan.
  \begin{enumerate}
  \item A map $f : A \to B$ in $\Mod(\cT)$ is \emph{full} if it has the right lifting property against all maps $H(p)$ for $p$ a display map.
  \item A map $f : A \to B$ in $\Mod(\cT)$ is an \emph{extension} if it has the left lifting property against all full maps.
  \item A model $A \in \Mod(\cT)$ is a \emph{$0$-extension} if the unique morphism $\initObj{\Mod(\cT)} \to A$ is an extension.
  \end{enumerate}
\end{definition}

\begin{remark}
  Let $G = \{H(p) \mid p \in \cT_{\dag}\}$.
  The class of full maps is $F = \rLift{G}$, and the class of extensions is $E = \lLift{(\rLift{G})}$.
  By Quillen's small object argument, $(E,F)$ forms a weak factorization system.
\end{remark}

\begin{remark}
  Let $\cT$ be a clan.
  For every display map $p : \Delta \display \Gamma$, $H(p) : H(\Gamma) \to H(\Delta)$ is an extension.
  In particular, $H(\Gamma)$ is a $0$-extension for all $\Gamma \in \cT$ since $\Gamma \display \termObj{\cT}$ is always a display map.

  In a finite product clan, full maps are precisely the regular epimorphisms, and $0$-extensions are precisely the projective objects.
  Hence finitely presentable $0$-extensions are precisely the sifted tiny objects, \ie finitely generated free algebras.
\end{remark}

\subsection{$(E,F)$-categories}

Every category of models $\Mod(\cT)$ of a clan $\cT$ is locally finitely presentable, and it comes equipped with a natural weak factorization system.
Hence a sensible domain for semantics is the class of finitely presentable categories equipped with a weak factorization system.
Taking finite product clans as a reference, a suitable notion of a morphism between $(E,F)$-categories is a finitary right adjoint preserving full maps.

\begin{definition}
  An $(E,F)$-category is a locally finitely presentable category $\cL$ equipped with a weak factorization system $(E,F)$, whose maps are called \emph{extensions} and \emph{full maps} respectively.
  A morphism of $(E,F)$-categories $\cL \to \cT$ is a finitary right adjoint $\cL \to \cT$ preserving full maps.
  We write $\EFCat$ for the 2-category of $(E,F)$-categories, morphisms, and natural transformations.
\end{definition}

\begin{lemma}
  If $\ell : \cT \to \cL$ is a cocontinuous functor between $(E,F)$-categories that additionally preserves extensions and finitely presentable objects, then its right adjoint $r : \cL \to \cT$ is a morphism of $(E,F)$-categories.
\end{lemma}
\begin{proof}
  Existence of right adjoint is given by the Adjoint Functor Theorem.
  The rest follows by standard abstract nonsense.
\end{proof}

Every morphism of clans $f : \cT \to \cQ$ induces a functor $- \circ f : \Mod(\cQ) \to \Mod(\cT)$ by precomposition.
We think of $f$ as interpreting $\cT$ in $\cQ$, and the functor $- \circ f$ as ``forgetting'' irrelevant structures.

\begin{lemma}
  The precomposition functor $- \circ f : \Mod(\cQ) \to \Mod(\cT)$ is a morphism of $(E,F)$-categories.
\end{lemma}
\begin{proof}
  The (unrestricted) precomposition functor $- \circ f : \funCat{\cQ}{\Set} \to \funCat{\cT}{\Set}$ admits left and adjoints given by Kan extensions; moreover $- \circ f$ and its left adjoint restrict to an adjunction along the inclusion $\Mod(\cT) \into \funCat{\cT}{\Set}$.
  Since the inclusions create filtered colimits, $- \circ f : \Mod(\cQ) \to \Mod(\cT)$ preserves filtered colimits.
  To see that full maps are preserved, note that full maps are regular epimorphisms; hence they are preserved by left adjoints.
\end{proof}

\begin{remark}
  In an arbitrary $(E,F)$-category, full maps are not necessarily regular epimorphisms; however, they are always regular epimorphic in the category of models of a clan (equipped with the canonical weak factorization system.)
\end{remark}



\begin{construction}
  The assignment $\cT \mapsto \Mod(\cT)$ extends to a 2-functor $\Mod : \Clan\op \to \EFCat$ as follows:
  \begin{itemize}
  \item $\cT \mapsto \Mod(\cT)$ equipped with the weak factorization system cofibrantly generated by $\{H(p) \mid p \in \cT_{\dag}\}$;
  \item $f : \cT \to \cQ \mapsto - \circ f : \Mod(\cQ) \to \Mod(\cT)$.
  \end{itemize}
\end{construction}

As a completeness theorem, we would like to associate each $(E,F)$-category with a clan.
In a finite product theory $\cT$, we take the dual of the full subcategory of $\Mod(\cT)$ spanned by sifted tiny objects.
In the clanic language, these are finitely presentable $0$-extensions.

\begin{definition}
  Given an $(E,F)$-category $\cL$, we write $\fC(\cL) \subseteq \cL$ for the full subcategory spanned by finitely presentable $0$-extensions.
\end{definition}

\begin{lemma}
  Let $\cL$ be an $(E,F)$-category.
  $\fC(\cL)\op$ equipped with $\fC(\cL)_{\dag} = \{\Delta \to \Gamma \mid \Gamma \to \Delta~\text{is an extension}\}$ is a clan.
\end{lemma}
\begin{proof}
  We show that $\fC(\cL)\op$ has pullbacks against any display map and that display maps are stable under pullback; all other conditions are trivial.
  To this end, suppose that $\Gamma, \Gamma'$, and $\Delta$ are finitely-presentable 0-extensions, and that $p : \Gamma \to \Gamma'$ is an extension and $s : \Gamma \to \Delta$ is an arbitrary morphism in $\cL$.
  We consider the following pushout:
  \begin{center}
    \DiagramSquare{
      nw = {\Gamma},
      ne = {\Gamma'},
      se = {\Delta'},
      se/style = {pushout},
      sw = {\Delta},
      north = {p},
      north/style = {extension},
      west = {s},
      south/style = {extension},
    }
  \end{center}
  Note that extensions are stable under pushout; hence it remains to show that $\Delta'$ is a finitely presentable 0-extension.
  It is finitely presentable because $\Gamma'$ and $\Delta$ are and filtered colimits commute with finite limits.
  It is a 0-extension because the unique map $\initObj{\cL} \to \Delta'$ can be factorized as a composite of extensions $\initObj{\cL} \to \Delta \to \Delta'$, and extensions are stable under composition.
\end{proof}

\begin{construction}
  Every morphism of $(E,F)$-categories $f : \cL \to \cQ$ admits a left adjoint $\ell : \cQ \to \cL$ preserving finitely presentable $0$-extensions.
  Hence $\ell$ restricts to a morphism of clans $\fC(\cQ)\op \to \fC(\cL)\op$, extending the object assignment $\fC(-)\op : \cL \to \fC(\cL)\op$ to a pseudofunctor.
\end{construction}

\begin{remark}
  Let $\cL$ be an $(E,F)$-category.
  The inclusion functor $\fC(\cL) \into \cL$ is a comodel.
\end{remark}

\begin{construction}
  For any $(E,F)$-category $\cL$, since the inclusion $j : \fC(\cL) \into \cL$ is a comodel, the nerve of $j$ is a cocontinuous functor $\eta_{\cL} : \cL \to \Mod(\fC(\cL)\op)$ by \cref{0000}.
\end{construction}

\begin{construction}
  For any clan $\cT$, the restricted Yoneda embedding $H : \cT\op \to \Mod(\cT)$ further restricts to $\varepsilon_{\cT}\op : \cT\op \to \fC(\Mod(\cT))$, since $H(\Gamma)$ is a finitely presentable 0-extension for all $\Gamma \in \cT$.
  Taking the opposite categories yields $\varepsilon_{\cT} : \cT \to \fC(\Mod(\cT))\op$.
\end{construction}

\begin{theorem}
  $\fC(-)\op$ is left biadjoint to $\Mod(-)$.
  Hence there is an equivalence between hom-categories.
  \[
    \Clan(\cT,\fC(\cL)\op) \eqv \EFCat(\cL,\Mod(\cT))
  \]
\end{theorem}
\begin{proof}
  $\eta$ and $\varepsilon$ in the two constructions above are the unit and counit respectively.
\end{proof}

\section{Cauchy complete clans}

The syntax is broken in the sense that the counit $\varepsilon$ is not an equivalence at every component.
The problem stems from the fact that a retract of a finitely generated free model is not necessarily free.
To correct this, we restrict to \emph{Cauchy complete} clans.

\begin{definition}
  A clan $\cT$ is \emph{Cauchy complete} if the underlying category is Cauchy complete and retracts of display maps are display maps.
\end{definition}

\begin{remark}
  If $\cC$ is Cauchy complete, then retracts of representable functors in $\funCat{\cC\op}{\Set}$ are representable.
\end{remark}

\begin{remark}
  For every $(E,F)$-category $\cL$, the clan $\fC(\cL)\op$ is Cauchy complete.
\end{remark}

\begin{definition}
  A model $A : \cT \to \Set$ is \emph{flat} if $\El(A)$ is filtered.
\end{definition}

\begin{lemma}
  A model $A : \cT \to \Set$ is flat if and only if it is a filtered colimit of representable models.
\end{lemma}
\begin{proof}
  By density of representable functors, we have $A \iso \colim(\El(A) \to \cT\op \xto{H} \Set)$.
  If $A$ is flat, then $A$ is clearly a filtered colimit of representables.
  The other direction follows from the fact that representable functors are flat and flat functors are closed under filtered colimits.
\end{proof}

\begin{corollary}\label{0001}
  For any clan $\cT$, the 0-extensions in $\Mod(\cT)$ are flat.
\end{corollary}
\begin{proof}
  By the \emph{fat small object argument}.
\end{proof}

\begin{theorem}
  If $\cT$ is a Cauchy complete clan, then $\varepsilon_{\cT} : \cT \to \fC(\Mod(\cT))\op$ is an equivalence of clans.
\end{theorem}
\begin{proof}
  The component $\varepsilon_{\cT}$ is fully faithful because it is the Yoneda embedding.
  We show that it is essentially surjective on objects.
  Let $C \in \Mod(\cT)$ be a finitely-presentable 0-extension.
  Then $C$ is a filtered colimit of representables by \cref{0001}; hence finite presentability implies that $C$ is a retract of some representable functor.
  Since $\cT$ is Cauchy complete, $C$ is representable; hence the equivalence.
\end{proof}

\section{Clan-algebraic categories}

\begin{definition}
  An $(E,F)$-category $\cL$ is called \emph{clan-algebraic} if
  \begin{enumerate}
  \item[(D)] the inclusion $j : \fC(\cL) \into \cL$ is dense;
  \item[(CG)] the weak factorization is cofibrantly generated by maps between finitely-presentable $0$-extensions;
  \item[(FQ)] equivalence relations $(p,q) : R \mono A \times A$ in $\cL$ with \emph{full} components $p, q$ are effective, and have \emph{full} coequalizers.
  \end{enumerate}
\end{definition}

\begin{lemma}
  Full maps in clan-algebraic categories are regular epimorphisms.
\end{lemma}
\begin{proof}
  Let $f : A \to B$ be a full map in a clan-algebraic category $\cL$.
  The lifting property against 0-extensions implies that for any $0$-extension $E$ and morphism $E \to B$, there is a lifting $E \to A$ making the following triangle commute:
  \begin{center}
    \begin{tikzpicture}[diagram]
      \node (1) {$A$};
      \node (2) [below = of 1] {$B$};
      \node (3) [left = of 2] {$E$};
      \draw [->>] (1) to node {$f$} (2);
      \draw [->] (3) to node {} (2);
      \draw [->,exists] (3) to node {} (1);
    \end{tikzpicture}
  \end{center}
  Hence $\eta_{\cL}(f) = j_{N}(f) : \Hom{\cL}(j-,A) \to \Hom{\cL}(j-,B)$ is pointwise surjective.
  This means that $\eta_{\cL}(f)$ is a regular epimorphism.
  Since left adjoints preserve regular epimorphism, $j_{\otimes}(j_N(f))$ is a regular epimorphism, but $j_{\otimes}j_N \iso \id{\cL}$ by density.
\end{proof}

\begin{lemma}
  For any flat model $A \in \Mod(\fC(\cL)\op)$, the unit $\eta_A : A \to j_N(j_{\otimes}(A))$ is an isomorphism; hence $j_{\otimes}$ is fully faithful on flat models.
\end{lemma}

\begin{fact}\label{0002}
  Every object $A$ in an $(E,F)$-category has a \emph{nice diagram} whose colimit is the coequalizer of a pair of full maps, and the coequalizer is $A$.
  Moreover, $j_{\otimes}$ preserves nice diagrams and $j_N$ preserves quotients of nice diagrams.
\end{fact}

\begin{remark}
  By construction every object in a nice diagram is a $0$-extension; hence by \cref{0001}, every $0$-extension in $\Mod(\fC(\cL)\op)$ is flat.
\end{remark}

\begin{theorem}
  If $\cL$ is clan-algebraic, then $\eta_{\cL} : \cL \to \Mod(\fC(\cL)\op)$ is an equivalence.
\end{theorem}
\begin{proof}
  By density, $\eta_{\cL}$ is fully faithful; it remains to show that it is essentially surjective.
  To this end, it suffices to show that the unit of the adjunction $j_{\otimes} \adj j_{N}$ at every component $A \in \Mod(\fC(\cL)\op)$ is an isomorphism.
  By \cref{0002}, every $A$ is a coequalizer determined by a nice diagram $A_{\bullet}$; hence we can prove the result by direct computation:
  \begin{align*}
    j_N(j_{\otimes}(A)) &\iso \colim(j_Nj_{\otimes}(A_{\bullet})) && \text{\cref{0002}}\\
                  &\iso \colim(A_{\bullet}) && \text{fully faithful functors reflect colimits}\\
                  &\iso A
  \end{align*}
\end{proof}

\subsection{Clan-algebraic weak factorization systems on $\Cat$}

\begin{align*}
  F &= \rLift{\{0 \to 1, 2 \to \mathbb{2}\}}\\
  F_O &= \rLift{\{0 \to 1, 2 \to \mathbb{2}, 2 \to 1\}}\\
  F_A &= \rLift{\{0 \to 1, 2 \to \mathbb{2}, \PP \to \mathbb{2}\}}\\
  F_{OA} &= \rLift{\{0 \to 1, 2 \to \mathbb{2}, 2 \to 1, \PP \to \mathbb{2}\}}
\end{align*}

\begin{observation}
  \begin{enumerate}
  \item $F$ consists of full and surjective-on-object functors.
  \item $F_O$ consists of full and bijective-on-object functors.
  \item $F_A$ consists of fully faithful and surjective-on-object functors.
  \item $F_{OA}$ consists of fully faithful and bijective-on-object functors.
  \end{enumerate}
\end{observation}

\begin{construction}
  In $F_O$, the generator $2 \to 1$ corresponds to a display map $(x : O) \to (x,y : O)$.
  We postulate a new type $E_O(x,y)$ so that $(x : O)$ is isomorphic to an extension of $(x,y : O)$.
  To satisfy this requirement, we require that $x = y$ whenever $E_O(x,y)$ is inhabited, and $E_O(x,y)$ must be contractible.
  This is precisely the extensional identity type on objects; we obtain the required display map as the following composite:
  \[
    (x : O) \iso (x,y : O, p : E_O(x,y)) \xto{p} (x, y : O)
  \]
\end{construction}

\begin{construction}
  In $F_A$, the generator $\mathbb{P} \to \mathbb{2}$ corresponds to a display map $(f : A(x,y)) \to (f, g : A(x,y))$.
  We postulate the extensional identity type on morphisms, and obtain the required display map as the following composite:
  \[
    (f : A(x,y)) \iso (f, g : A(x,y), p : E_A(f,g)) \xto{p} (f,g : A(x,y))
  \]
\end{construction}

\printbibliography
\end{document}

\documentclass[a4paper]{article}

\usepackage{preamble}
\addbibresource{references/refs.bib}

\newcommand{\Th}{{\mathsf{Th}}}
\newcommand{\N}{\mathbb{N}}

\title{LOG111 Hand-in 3}
\author{Frank Tsai}

\begin{document}

\maketitle

\section{}
\begin{proof}
  By construction, the set $\Gamma_{1} \cup \Gamma_{2}$ is unsatisfiable, so by compactness, there is a finite unsatisfiable subset $\Delta \subseteq \Gamma_{1} \cup \Gamma_{2}$.
  
  Consider
  \begin{align*}
    \Delta_{1} := \Delta \setminus \Gamma_{2} && \Delta_{2} := \Delta \setminus \Gamma_{1}.
  \end{align*}
  We claim that $\Delta_{1}$ and $\Delta_{2}$ respectively axiomatize $\Th(\Gamma_{1})$ and $\Th(\Gamma_{2})$.
  We prove that this is the case for $\Delta_{1}$; the argument for $\Delta_{2}$ is completely analogous.

  We need to prove that for any formula $\varphi$, $\Gamma_{1} \vdash \varphi$ iff $\Delta_{1} \vdash \varphi$.
  To this end, it suffices to prove their semantic counterpart by soundness and completeness.
  
  The if direction is an immediate consequence of monotonicity.
  In the other direction, suppose that $\Gamma_{1} \vDash \varphi$ and let $M \vDash \Delta_{1}$.
  If $M \vDash \Gamma_{1}$ then we are done.
  On the other hand, if $M \nvDash \Gamma_{1}$ then it follows that $M \vDash \Delta_{2}$, but this means that $M \vDash \Delta_{1} \cup \Delta_{2}$ contradicting the fact that $\Delta = \Delta_{1} \cup \Delta_{2}$ is unsatisfiable.
\end{proof}

\section{}
\begin{enumerate}
\item
  \[\begin{prooftree}
      \hypo{c_{x} < c_{y}}
      \hypo{[c_{y} < c_{x}]^1}
      \infer2[T]{c_{x} < c_{x}}
      \infer1[R]{\bot}
      \infer1[RAA$_{2}^1$]{c_{y} \nless c_{x}}
    \end{prooftree}\]
\item
  \begin{mathpar}
    (X,R) \nvDash \bot\and
    (X,R) \vDash c_{x} < c_{y}~\text{iff $(x,y) \in R$}\and
    (X,R) \vDash c_{x} \nless c_{y}~\text{iff $(x,y) \notin R$}
  \end{mathpar}
  $\Gamma \vDash \varphi$ iff for every $(X,R)$, if $(X,R)$ satisfies every formula in $\Gamma$ then $(X,R)$ satisfies $\varphi$.
\item To prove soundness, we can do an induction on the height of the derivation tree.
  
  The base case is immediate.
  In the induction case, we do a case analysis on the last applied rule.
  When the last applied rule is $\text{RAA}_{1}$, the induction hypothesis yields $\Gamma,c_{x} \nless c_{y} \vDash \bot$.
  Thus, for any $(X,R)$ satisfying $\Gamma$, $(X,R)$ must satisfy $c_{x} < c_{y}$, i.e., $\Gamma \vDash c_{x} < c_{y}$.
\item Given a formula $\varphi$ that is not $\bot$, we shall use the informal notation $\lnot\varphi$ to denote the opposite of $\varphi$, e.g., if $\varphi \equiv c_{x} \nless c_{y}$ then $\lnot\varphi \equiv c_{x} < c_{y}$.
  Note that the set of formulas is enumerable because $X \times X$ is countable.
  We write $c_{n_1} < c_{n_2}$ or $c_{n_1} \nless c_{n_2}$ for the $n$-th enumeration.
  \begin{construction}\label{const:context-completion}
    Let $\Gamma$ be consistent, we extend $\Gamma$ by taking the fixed point of the following process and we name the resulting set $\Gamma^*$.
    \begin{mathpar}
      \Gamma^{*} = \bigcup\{\Gamma_{n} \mid n \in \N\}\and
      \Gamma_{0} = \Gamma\and
      \Gamma_{n+1} =
      \begin{cases}
        \Gamma_{n},c_{n_1} < c_{n_2} & \text{if the resulting set is consistent,}\\
        \Gamma_{n},c_{n_1} \nless c_{n_2} & \text{otherwise.}
      \end{cases}
    \end{mathpar}
  \end{construction}
  
  \begin{lemma}\label{thm:completion-consistent}
    If $\Gamma$ is consistent, then the set $\Gamma^*$ as in \cref{const:context-completion} is consistent and complete in the sense that for any $a,b \in X$, either $c_{a} < c_{b}$ or $c_{a} \nless c_{b}$.
  \end{lemma}
  \begin{proof}
    Completeness is evident: if the pair $(n_1, n_2)$ enumerates $(a,b)$, then either $c_a < c_b$ or $c_a \nless c_b$ is added at step $n$.

    Now, we show that each $\Gamma_{n}$ is consistent.
    We proceed by induction on $n$.
    The base case is trivial.
    In the induction step, if $\Gamma_{n+1} = \Gamma_{n},c_{n_1} < c_{n_2}$ then there is nothing left to prove.

    If $\Gamma_{n+1} = \Gamma_{n},c_{n_1} \nless c_{n_2}$, then $\Gamma_{n},c_{n_1} < c_{n_2} \vdash \bot$.
    Suppose that $\Gamma_{n},c_{n_1} \nless c_{n_2} \vdash \bot$.
    Then we have a derivation of $\Gamma_{n} \vdash c_{n_1} < c_{n_2}$ as follows.
    \[
      \begin{prooftree}
        \hypo{[c_{n_1} \nless c_{n_2}]}
        \ellipsis{}{\bot}
        \infer1[ARR]{c_{n_1} < c_{n_2}}
      \end{prooftree}
    \]
    But this gives a derivation of $\Gamma_{n} \vdash \bot$.
    \[
      \begin{prooftree}
        \hypo{[c_{n_1} \nless c_{n_2}]}
        \ellipsis{}{\bot}
        \infer1[ARR]{c_{n_1} < c_{n_2}}
        \ellipsis{}{}
        \infer1{\bot}
      \end{prooftree}
    \]
    This contradicts the induction hypothesis.
    If $\Gamma^* \vdash \bot$ then there is an inconsistent subset $\Gamma' \subseteq \Gamma^*$.
    This subset lies in $\Gamma_{n}$ for some $n$, rendering this set inconsistent.
    But this is a contradiction.
  \end{proof}
  
  \begin{lemma}\label{thm:model-existence}
    If $\Gamma$ is consistent then $\Gamma$ has a model.
    Moreover, if $\Gamma \nvdash \varphi$, then this model also satisfies $\lnot\varphi$, where $\varphi$ is not $\bot$.
  \end{lemma}
  \begin{proof}
    By \cref{thm:completion-consistent}, $\Gamma^*$ is consistent and complete.
    Consider the following relation.
    \[
      R = \{(x,y) \mid c_{x} < c_{y} \in \Gamma^{*}\}
    \]
    We postpone the additional proof obligations for irreflexivity and transitivity.

    To show that $(X,R) \vDash \Gamma$, it suffices to show that $(X,R) \vDash \Gamma^*$.
    Let $\varphi \in \Gamma^*$.
    Note that it must be either $c_x < c_y$ or $c_x \nless c_y$ because $\Gamma^*$ is consistent.
    In the former case, there is nothing left to prove.
    In the latter case, suppose that $(x,y) \in R$.
    Then by definition $c_x < c_y \in \Gamma^*$.
    This yields $\Gamma^* \vdash \bot$ as follows.
    \[
      \begin{prooftree}
        \hypo{c_x < c_y}
        \hypo{c_x \nless c_y}
        \infer2[$\bot$]{\bot}
      \end{prooftree}
    \]
    This is a contradiction.

    It remains to show the second half of the statement.
    Note that if $\Gamma \nvdash \varphi$, then $\Gamma,\lnot\varphi$ is consistent because if $\Gamma,\lnot\varphi \vdash \bot$ then we have a derivation of $\Gamma \vdash \varphi$.
    \[
      \begin{prooftree}
        \hypo{[\lnot\varphi]}
        \ellipsis{}{\bot}
        \infer1[RAA]{\varphi}
      \end{prooftree}
    \]
    Thus, $\lnot\varphi$ has to be added to $\Gamma^*$ at some point.
    Then it follows by construction that $(X,R) \vDash \lnot\varphi$.
  \end{proof}
  \begin{proof}[Proof of irreflexivity]
    If $(x,x) \in R$, then by definition $c_{x} < c_{x} \in \Gamma^{*}$.
    But this means that there is a derivation of $\Gamma^* \vdash \bot$.
    \[
      \begin{prooftree}
        \hypo{c_{x} < c_{x}}
        \infer1[R]{\bot}
      \end{prooftree}
    \]
    This contradicts the fact that $\Gamma^*$ is consistent.
  \end{proof}
  \begin{proof}[Proof of transitivity]
    Let $(x,y),(y,z) \in R$.
    We need to show that $(x,z) \in R$.
    By completeness, either $c_{x} < c_{z} \in \Gamma^*$ or $c_{x} \nless c_{z} \in \Gamma^*$.
    In the former case, there is nothing left to prove.
    On the other hand, if $c_{x} \nless c_{z} \in \Gamma^*$ then we have a derivation $\Gamma^* \vdash \bot$.
    \[\begin{prooftree}
        \hypo{c_{x} < c_{y}}
        \hypo{c_{y} < c_{z}}
        \infer2[T]{c_{x} < c_{z}}
        \hypo{c_{x} \nless c_{z}}
        \infer2[$\bot$]{\bot}
      \end{prooftree}\]
    This contradicts the fact that $\Gamma^*$ is consistent.
  \end{proof}
  
  \begin{theorem}[Completeness]
    If $\Gamma \vDash \varphi$ then $\Gamma \vdash \varphi$.
  \end{theorem}
  \begin{proof}
    Suppose that $\Gamma \nvdash \varphi$ and that $\Gamma \vDash \varphi$.
    The first assumption implies that $\Gamma$ is consistent, so by \cref{thm:model-existence}, $\Gamma$ has a model $(X,R)$ which also satisfies $\lnot\varphi$.
    But this is a contradiction because $(X,R)$ must satisfy its opposite $\varphi$ by the second assumption.
  \end{proof}
\item
  \begin{proof}
    It suffices to repair the induction step of \cref{thm:completion-consistent}.
    The proof of the first half of \cref{thm:model-existence} goes through without any problem.

    There is nothing to repair in the first case of the induction step.
    In the second case, we know that $\Gamma_{n},c_{n_1} < c_{n_2} \vdash_i \bot$.
    Suppose that $\Gamma_{n},c_{n_1} \nless c_{n_2} \vdash_i \bot$.
    Let us consider the last rule of this derivation.
    Only R and $\bot$ are immediately applicable.

    The premise of R is a positive formula, so it must come from $\Gamma_{n}$.
    But this is impossible because $\Gamma_{n}$ is consistent by the induction hypothesis.
    Thus, we can rule out R.

    The only applicable rule is $\bot$.
    This rule demands $c_{n_1} < c_{n_2}$, so $\Gamma_{n} \vdash_{i} c_{n_1} < c_{n_2}$.
    But this means that $\Gamma_{n}$ is not consistent after all.
  \end{proof}
\item Note that the intuitionistic fragment cannot derive negative formulas, so consider $c_{x} < c_{y} \vDash c_{x} \nless c_{y}$.
  This clearly holds.
  In fact, Problem A and soundness imply this.
\end{enumerate}

%\printbibliography

\end{document}

\documentclass[a4paper]{article}

\usepackage{preamble}
\addbibresource{references/refs.bib}

\newcommand{\Th}{{\mathsf{Th}}}
\newcommand{\N}{\mathbb{N}}

\title{LOG111 Hand-in 3}
\author{Frank Tsai}

\begin{document}

\maketitle

\section{}
\begin{proof}
  By construction, the set $\Gamma_{1} \cup \Gamma_{2}$ is unsatisfiable, so by compactness, there is a finite unsatisfiable subset $\Delta \subseteq \Gamma_{1} \cup \Gamma_{2}$.
  
  Consider
  \begin{align*}
    \Delta_{1} := \Delta \setminus \Gamma_{2} && \Delta_{2} := \Delta \setminus \Gamma_{1}.
  \end{align*}
  We claim that $\Delta_{1}$ and $\Delta_{2}$ respectively axiomatize $\Th(\Gamma_{1})$ and $\Th(\Gamma_{2})$.
  We prove that this is the case for $\Delta_{1}$; the argument for $\Delta_{2}$ is completely analogous.

  We need to prove that for any formula $\varphi$, $\Gamma_{1} \vdash \varphi$ iff $\Delta_{1} \vdash \varphi$.
  To this end, it suffices to prove their semantic counterpart by soundness and completeness.
  
  The if direction is an immediate consequence of monotonicity.
  In the other direction, suppose that $\Gamma_{1} \vDash \varphi$ and let $M \vDash \Delta_{1}$.
  If $M \vDash \Gamma_{1}$ then we are done.
  On the other hand, if $M \nvDash \Gamma_{1}$ then it follows that $M \vDash \Delta_{2}$, but this means that $M \vDash \Delta_{1} \cup \Delta_{2}$ contradicting the fact that $\Delta = \Delta_{1} \cup \Delta_{2}$ is unsatisfiable.
\end{proof}

\section{}
\begin{enumerate}
\item
  \[\begin{prooftree}
      \hypo{c_{x} < c_{y}}
      \hypo{[c_{y} < c_{x}]^1}
      \infer2[T]{c_{x} < c_{x}}
      \infer1[R]{\bot}
      \infer1[RAA$_{2}^1$]{c_{y} \nless c_{x}}
    \end{prooftree}\]
\item
  \begin{mathpar}
    (X,R) \nvDash \bot\and
    (X,R) \vDash c_{x} < c_{y}~\text{iff $(x,y) \in R$}\and
    (X,R) \vDash c_{x} \nless c_{y}~\text{iff $(x,y) \notin R$}
  \end{mathpar}
  $\Gamma \vDash \varphi$ iff for every $(X,R)$, if $(X,R)$ satisfies every formula in $\Gamma$ then $(X,R)$ satisfies $\varphi$.
\item To prove soundness, we can do an induction on the height of the derivation tree.
  
  The base case is immediate.
  In the induction case, we do a case analysis on the last applied rule.
  When the last applied rule is $\text{RAA}_{1}$, the induction hypothesis yields $\Gamma,c_{x} \nless c_{y} \vDash \bot$.
  Thus, for any $(X,R)$ satisfying $\Gamma$, $(X,R)$ must satisfy $c_{x} < c_{y}$, i.e., $\Gamma \vDash c_{x} < c_{y}$.
\item 
  \begin{proof}
    We show that if $\Gamma$ is consistent then $\Gamma$ is satisfiable.
    First, $X \times X$ is countable because $X$ is, so the set of formulas is enumerable.
    It is sufficient to enumerate formulas of the form $c_{x} < c_{y}$ as we have no use for the other formulas.
    We write $c_{n_1} < c_{n_2}$ for the $n$-th enumeration.

    Let $\Gamma$ be a consistent set of formulas.
    We extend $\Gamma$ as follows:
    \begin{mathpar}
      \Gamma^{*} = \bigcup\{\Gamma_{n} \mid n \in \N\}\and
      \Gamma_{0} = \Gamma\and
      \Gamma_{n+1} =
      \begin{cases}
        \Gamma_{n},c_{n_1} < c_{n_2} & \text{if the resulting set is consistent,}\\
        \Gamma_{n} & \text{otherwise.}
      \end{cases}
    \end{mathpar}
    By a simple induction on $n$, one can show that each $\Gamma_{n}$ is consistent.
    If $\Gamma^{*} \vdash \bot$, then there is a subset $\Gamma' \subseteq \Gamma^{*}$ such that $\Gamma' \vdash \bot$.
    Note that $\Gamma' \subseteq \Gamma_{n}$ for some $n$ (each approximation of $\Gamma^{*}$ is cumulative), but this contradicts the fact that $\Gamma_{n}$ is consistent.
    
    We are now ready to define a relation:
    \[
      R = \{(x,y) \mid c_{x} < c_{y} \in \Gamma^{*}\}
    \]
    If $(x,x) \in R$, then by definition $c_{x} < c_{x} \in \Gamma^{*}$.
    But this means that $\Gamma^{*}$ is inconsistent as $\Gamma^* \vdash \bot$ via R.
    Thus, $R$ is irreflexive.

    Let $(a,b),(b,c) \in R$.
    We need to show that $(a,c) \in R$.
    Suppose that $c_{a} < c_{c} \notin \Gamma^*$.
    Then it follows by construction that $\Gamma^*,c_{a} < c_{c}$ is inconsistent.
    But this gives a derivation $\Gamma^* \vdash \bot$ as follows.
    \[\begin{prooftree}
        \hypo{c_{a} < c_{b}}
        \hypo{c_{b} < c_{c}}
        \infer2[T]{c_{a} < c_{c}}
        \hypo{[c_{a} < c_{c}]^*}
        \ellipsis{}{\bot}
        \infer1[RAA$_2^*$]{c_{a} \nless c_{c}}
        \infer2[$\bot$]{\bot}
      \end{prooftree}\]
    This contradicts the fact that $\Gamma^*$ is consistent.
    Thus, $R$ is transitive.

    $R$ satisfies all the positive formulas in $\Gamma^*$ by construction.
    It also satisfies all the negative formulas because if a negative formula $c_a \nless c_b$ is not satisfied then $(a,b) \in R$.
    By construction, this means that $c_a < c_b \in \Gamma^*$, rendering $\Gamma^*$ inconsistent.

    Now, suppose that $\Gamma \nvdash \varphi$ and that $\Gamma \vDash \varphi$.
    Then $\Gamma$ is consistent, so we can use the construction above to extract a model.
    Since $\Gamma,\lnot\varphi$ is also consistent, $\lnot\varphi \in \Gamma^*$ (this relies on one of the RAA rules).
    But this means that $(X,R)$ has to satisfy both $\varphi$ and $\lnot\varphi$, which is impossible.
  \end{proof}
\end{enumerate}

%\printbibliography

\end{document}

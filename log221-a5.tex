\documentclass[a4paper]{article}
\usepackage[margin=1in]{geometry}

\usepackage{preamble}
\usepackage{log221-a5-macros}
\usepackage{showkeys}

\title{LOG 221 - A5}
\author{Frank Tsai}

\begin{document}

\maketitle

\begin{lemma}\label{0000}
  Every basic axiom of arithmetic is derivable in $\PAo$ with height at most 2.
\end{lemma}
\begin{proof}
  \begin{mathpar}
    \ebrule[A1]{
      \hypo{0' = 0 \Rightarrow \bot}
      \infer1[\rRImp]{\Rightarrow \lnot(0' = 0)}
      \hypo{\ldots}
      \hypo{n' = 0 \Rightarrow \bot}
      \infer1[\rRImp]{\Rightarrow \lnot(n' = 0)}
      \hypo{\ldots}
      \infer4[\rRo]{\Rightarrow \forall x.\lnot(x' = 0)}
    }
  \end{mathpar}
\end{proof}

\section*{Problem 1}\label{0001}

We show that if $\PAd \Gamma \Rightarrow \Delta$, then there are $k, n < \omega$ such that for any closing substitution $\Theta$, there is an ordinal $\alpha < \omega \cdot n$ such that $\PAod{\alpha}{k} \Gamma[\Theta] \Rightarrow \Delta[\Theta]$.

\begin{proof}[Solution]
  We proceed by induction on the height of the derivation; we do a case analysis on the last applied rule.

  If $\Gamma \Rightarrow \Delta$ is an initial sequent, then there are three cases to consider: (1) the initial sequent is an instance of \rLBot, (2) the initial sequent is an instance of id, and (3) the initial sequent is a basic axiom.

  The only rule that requires us to fiddle with the cut rank $k$ is \rCut.
\end{proof}

% \bibliography{bib}
% \bibliographystyle{alpha}

\end{document}

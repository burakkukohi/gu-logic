\documentclass[a4paper]{article}
\usepackage[margin=1in]{geometry}

\usepackage{preamble}
\usepackage{log221-a5-macros}
\usepackage[final]{showkeys}

\title{LOG 221 - A5}
\author{Frank Tsai}

\begin{document}

\maketitle

% \begin{lemma}\label{0000}
%   Every basic axiom of arithmetic is derivable in $\PAo$ with height $< \omega$.
% \end{lemma}
% \begin{proof}
%   \begin{mathpar}
%     \ebrule[A1]{
%       \hypo{0' = 0 \Rightarrow \bot}
%       \infer1[\rRImp]{\Rightarrow \lnot(0' = 0)}
%       \hypo{\ldots}
%       \hypo{n' = 0 \Rightarrow \bot}
%       \infer1[\rRImp]{\Rightarrow \lnot(n' = 0)}
%       \hypo{\ldots}
%       \infer4[\rRo]{\Rightarrow \forall x.\lnot(x' = 0)}
%     }
%   \end{mathpar}
% \end{proof}

\section*{Problem 1}\label{0001}

\begin{notation}
  Given a substitution $\Theta$, we write $\Theta_{x}$ for the value of $\Theta$ at $x$, and write $\Theta_{a/x}$ for the following substitution:
  \[
    \Theta_y =
    \begin{cases}
      a & \text{if $x = y$},\\
      \Theta_y & \text{otherwise.}
    \end{cases}
  \]
\end{notation}

We show that if $\PAd \Gamma \Rightarrow \Delta$, then there are $k, n < \omega$ such that for any closing substitution $\Theta$, there is an ordinal $\alpha < \omega \cdot n$ such that $\PAod{\alpha}{k} \Gamma[\Theta] \Rightarrow \Delta[\Theta]$.

\begin{proof}[Solution]
  We proceed by induction on the height of the derivation; we do a case analysis on the last applied rule.

  If $\Gamma \Rightarrow \Delta$ is an initial sequent, then there are three cases to consider: (1) the initial sequent is an instance of \rLBot, (2) the initial sequent is an instance of id, and (3) the initial sequent is a basic axiom.
  In all of these cases, we choose $n = 2$ and $k = 0$; and for any closing substitution $\Theta$, choose $\alpha = \omega < \omega \cdot n = \omega \cdot 2$.

  In the first case, since $\bot$ is a closed formula, substitution has no effect on it; hence $\PAod{\alpha}{k} \Gamma[\Theta] \Rightarrow \Delta[\Theta]$ is an instance of \rLBot.
  In the second case, there is an atom $R(x) \in \Gamma$ and $\Delta$; hence we have $R(\Theta_x) \in \Gamma[\Theta]$ and $\Delta[\Theta]$.
  Clearly, $\NN \vDash \Theta_x = \Theta_x$, so $\PAod{\alpha}{k} \Gamma[\Theta] \Rightarrow \Delta[\Theta]$ by id.
  In the last case, note that every basic axiom $A$ is a closed formula, so substitution has no effect, \ie $A \in \Delta[\Theta]$.
  All basic axioms are derivable in $\PAo$ with finite height using no \rCut{}s.

  Every inference rule of $\PA$ and $\PAo$ except \rLo{} and \rRo{} has the following general form.
  \begin{mathpar}
    \ebrule[*]{
      \hypo{\Gamma_1 \Rightarrow \Delta_1}
      \hypo{\Gamma_2 \Rightarrow \Delta_2}
      \infer2{\Gamma \Rightarrow \Delta}
    }
  \end{mathpar}
  We analyze every inference rule except \rLEx, \rRAll, and \rCut{} at once.
  \begin{itemize}
  \item[*] We must show that there are $k, n < \omega$ such that for any closing substitution $\Theta$, there is an ordinal $\alpha < \omega \cdot n$ with $\PAod{\alpha}{k} \Gamma[\Theta] \Rightarrow \Delta[\Theta]$.
    By the induction hypothesis, there are $k_1,n_1 < \omega$ such that for any closing substitution $\Theta_1$, there is an ordinal $\alpha_1 < \omega \cdot n_1$ with $\PAod{\alpha_1}{k_1} \Gamma_1[\Theta_1] \Rightarrow \Delta_1[\Theta_1]$; and there are $k_2,n_2 < \omega$ such that for any closing substitution $\Theta_2$, there is an ordinal $\alpha_2 < \omega \cdot n_2$ with $\PAod{\alpha_2}{k_2} \Gamma_2[\Theta_2] \Rightarrow \Delta_1[\Theta_2]$.

    Choose $k = \max(k_1, k_2)$ and $n = n_1 + n_2 + 2$.
    Fix a closing substitution $\Theta$; then choose $\alpha = \omega \cdot (n_1 + n_2 + 1)$.
    Applying * to the two sequents given by the induction hypothesis yields $\Gamma[\Theta] \Rightarrow \Delta[\Theta]$; it remains to verify the bounds.
    The cut rank is easy because we did not use any additional \rCut.
    As for the height bound, it is easy to see that $\alpha < \omega \cdot n = \omega \cdot (n_1 + n_2 + 2)$.
    It remains to show that $\alpha$ is strictly larger than $\alpha_1$ and $\alpha_2$, but this is also easy since $\alpha_1 < \omega \cdot n_1 < \omega \cdot (n_1 + 1) \leq \alpha$ and $\alpha_2 < \omega \cdot n_2 < \omega \cdot (n_2 + 1) \leq \alpha$.
  \item[\rCut] This is the only rule where we need to fiddle with the cut rank.
    The argument is essentially the same as the previous case.
    The only required modification is that we choose $k = \max(k_1,k_2,|C|+1)$, where $C$ is the cut formula.
  \item[\rLEx] We must show that there are $k,n < \omega$ such that for any closing substitution $\Theta$, there is an ordinal $\alpha < \omega \cdot n$ with $\PAod{\alpha}{k} \Gamma[\Theta], \exists x.A(x)[\Theta] \Rightarrow \Delta[\Theta]$.
    By the induction hypothesis, there are $k',n' < \omega$ such that for any closing substitution $\Theta'$, there is an ordinal $\alpha' < \omega \cdot n'$ with $\PAod{\alpha'}{k'} \Gamma[\Theta'], A(a)[\Theta'] \Rightarrow \Delta[\Theta']$.

    Choose $k = k'$ and $n = n' + 1$.
    Fix any closing substitution $\Theta$; then choose $\alpha = \omega \cdot n'$.
    Note that $A(a)[\Theta_{\bar{n}/a}] = A(\bar{n})[\Theta]$ for all $n \in \NN$.
    Moreover, the eigenvariable condition on $a$ implies that $\Gamma[\Theta_{\bar{n}/a}] = \Gamma[\Theta]$ and $\Delta[\Theta_{\bar{n}/a}] = \Delta[\Theta]$ for all $n \in \NN$.
    Hence we obtain $\Gamma[\Theta],\exists x.A(x)[\Theta] \Rightarrow \Delta[\Theta]$ by \rLo.
    \begin{mathpar}
      \ebrule{
        \hypo{\ldots}
        \hypo{\Gamma[\Theta],A(\bar{n})[\Theta] \Rightarrow \Delta[\Theta]}
        \hypo{\ldots}
        \infer3[\rLo]{\Gamma[\Theta],\exists x.A(x)[\Theta] \Rightarrow \Delta[\Theta]}
      }
    \end{mathpar}
    It remains to verify the bounds.
    The cut rank is easy because we did not use any additional \rCut.
    Since the height of every premise is strictly bounded by $\omega \cdot n' = \alpha$, $\alpha$ is a valid height bound for the conclusion.
    It remains to verify $\alpha < \omega \cdot n$, but this is evident since $n = n' + 1$.
  \item[\rRAll] Analogous to the previous case.
  \item[\textsc{Ind}] By the induction hypothesis, there are $k', n' < \omega$ such that for any closing substitution $\Theta$, there is an ordinal $\alpha' < \omega \cdot n'$ with $\PAod{\alpha}{k} A(\bar{k})[\Theta],\Gamma[\Theta] \Rightarrow \Delta[\Theta], A(\text{succ}(\bar{k}))[\Theta]$ for any $k \in \NN$.
    This is justified by the eigenvariable condition.

    By inversion, it suffices to derive $A(\bar{0})[\Theta], \Gamma[\Theta] \Rightarrow \Delta[\Theta], \forall x.A(x)[\Theta]$.
    Choose $k = \max(k', |A(a)| + 1)$ and $n = n' + 3$.
    Fix any closing substitution $\Theta$; we choose $\alpha = \omega \cdot (n' + 2)$.
    We apply \rRo; the first premise is an initial sequent, while all remaining premises can be obtained by cutting the induction hypothesis against the previous premise.
    \begin{mathpar}
      \ebrule{
        \hypo{A(\bar{0}), \Gamma[\Theta] \Rightarrow \Delta[\Theta], A(\bar{0})[\Theta]}
        \hypo{A(\bar{0}), \Gamma[\Theta] \Rightarrow \Delta[\Theta], A(\bar{0})[\Theta]}
        \hypo{A(\bar{0})[\Theta],\Gamma[\Theta] \Rightarrow \Delta[\Theta], A(\bar{1})[\Theta]}
        \infer2[\rCut]{A(\bar{0}), \Gamma[\Theta] \Rightarrow \Delta[\Theta], A(\bar{1})[\Theta]}
        \hypo{\cdots}
        \infer3[\rRo]{A(\bar{0}[\Theta]), \Gamma[\Theta] \Rightarrow \Delta[\Theta], \forall x.A(x)[\Theta]}
      }
    \end{mathpar}
    Since the cut rank of every newly added \rCut{}s is $|A(a)|$, $k$ is sufficient for this derivation.
    Since we add an additional \rCut{} in each premise, the height of this derivation is bounded by $\omega \cdot n' + \omega + 1 = \omega \cdot (n' + 1) + 1 < \omega \cdot (n' + 1) + \omega = \omega \cdot (n' + 2) = \alpha$.
    Clearly, $\alpha < \omega \cdot n = \omega \cdot (n' + 3)$.
  \end{itemize}
  The embedding lemma follows by replacing $\alpha$ with $\omega \cdot n$; this is a valid height bound since $\alpha < \omega \cdot n$.
\end{proof}

% \bibliography{bib}
% \bibliographystyle{alpha}

\end{document}

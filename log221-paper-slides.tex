\documentclass{beamer}

\usepackage{beamer-preamble}
\usepackage{log221-paper-macros}

\title{Curry-Howard correspondence and normalization}
\author{Frank Tsai}

\begin{document}
\maketitle

\begin{frame}{Curry-Howard correspondence}
  Proof systems and computational calculi are closely related.
  The BHK interpretation can be regarded as an incarnation of the Curry-Howard correspondence.
  \begin{itemize}
  \item A \only<1>{proof}\only<2->{\alert{term}} of \only<1>{$A \vee B$}\only<2->{$A + B$} is either a \only<1>{proof}\only<2->{\alert{term}} of $A$ or a \only<1>{proof}\only<2->{\alert{term}} of $B$.
  \item A \only<1>{proof}\only<2->{\alert{term}} of $A \to B$ is a function that converts a \only<1>{proof}\only<2->{\alert{term}} of $A$ to a \only<1>{proof}\only<2->{\alert{term}} of $B$.
  \end{itemize}
\end{frame}

\AtBeginSection[]
{
  \begin{frame}<beamer>{Outline}
    \tableofcontents[currentsection]
  \end{frame}
}

\section{Static}

\begin{frame}{From proofs to terms}
  We attach terms to natural deduction inference rules.
  \only<2->{The additional rule, \alert{\rVar}, makes the use of assumptions explicit.}
  \begin{figure}
    \centering
    \begin{mathpar}
      \only<2->{
        \ebrule[\alert{\rVar}]{
          \hypo{x : A \in \Gamma}
          \infer1{\Gamma \vdash x : A}
        }\and
      }
      \ebrule[\rSumIl]{
        \hypo{\Gamma \vdash a : A}
        \infer1{\Gamma \vdash \tmInl{a} : A + B}
      }\and
      \ebrule[\rSumE]{
        \hypo{\Gamma \vdash s : A + B}
        \hypo{\Gamma, x : A \vdash c : C}
        \hypo{\Gamma, y : B \vdash c' : C}
        \infer3{\Gamma \vdash \tmCase{s}{c}{c'} : C}
      }\and
      \ebrule[\rFunI]{
        \hypo{\Gamma, x : A \vdash b : B}
        \infer1{\Gamma \vdash \tmLam{x : A}{b} : A \to B}
      }\and
      \ebrule[\rFunE]{
        \hypo{\Gamma \vdash f : A \to B}
        \hypo{\Gamma \vdash a : A}
        \infer2{\Gamma \vdash \tmApp{f}{a} : B}
      }
    \end{mathpar}
    \caption{Selected typing rules}
  \end{figure}
\end{frame}

% \begin{frame}{From terms to proofs}
%   Given a well-typed term $\Gamma \vdash t : A$, we can extract a natural deduction proof $\Gamma \vdash A$ by erasing every term in the typing derivation, \eg
%   \begin{onlyenv}<1>
%     \begin{mathpar}
%       \ebrule{
%         \hypo{x : A \in \{x : A\}}
%         \infer1[\rVar]{x : A \vdash x : A}
%         \infer1[\rSumIl]{x : A \vdash \tmInl{x} : A + B}
%         \infer1[\rFunI]{\vdash \tmLam{x : A}{\tmInl{x}} : A \to A + B}
%       }
%     \end{mathpar}
%   \end{onlyenv}
%   \begin{onlyenv}<2>
%     \begin{mathpar}
%       \ebrule{
%         \hypo{x : A \vdash x : A}
%         \infer1[\rSumIl]{x : A \vdash \tmInl{x} : A + B}
%         \infer1[\rFunI]{\vdash \tmLam{x : A}{\tmInl{x}} : A \to A + B}
%       }
%     \end{mathpar}
%   \end{onlyenv}
%   \begin{onlyenv}<3>
%     \begin{mathpar}
%       \ebrule{
%         \hypo{A \vdash A}
%         \infer1[$\vee$I$_1$]{A \vdash A \vee B}
%         \infer1[\rFunI]{\vdash A \to A \vee B}
%       }
%     \end{mathpar}
%   \end{onlyenv}
%   \begin{onlyenv}<4->
%     \begin{mathpar}
%       \ebrule{
%         \hypo{[A]}
%         \infer1[$\vee$I$_1$]{A \vee B}
%         \infer1[\rFunI]{A \to A \vee B}
%       }
%     \end{mathpar}
%   \end{onlyenv}
% \end{frame}

\section{Dynamic}

\begin{frame}{Detour conversions as a computational dynamic}
  \only<1-4>{Detour conversions: elimination rules are sufficiently strong, \eg}
  \only<5->{A natural computational dynamic as a term rewriting system generated by the following rules:}
  \begin{onlyenv}<1-2>
    \begin{mathpar}
      \ebrule{
        \hypo{[A]}
        \ellipsis{$\cD_2$}{B}
        \infer1[\rFunI]{A \to B}
        \hypo{}
        \ellipsis{$\cD_1$}{A}
        \infer2[\rFunE]{B}
      }\qquad\Rightarrow_{\beta}\qquad
      \ebrule{
        \hypo{}
        \ellipsis{$\cD_1$}{A}
        \ellipsis{$\cD_2$}{B}
      }
    \end{mathpar}
  \end{onlyenv}
  \begin{onlyenv}<2>
    \begin{mathpar}
      \tmApp{\tmLam{x : A}{b}}{a} \bRed b[a/x]
    \end{mathpar}
  \end{onlyenv}
  \begin{onlyenv}<3-4>
    \begin{mathpar}
      \ebrule{
        \hypo{}
        \ellipsis{$\cD_1$}{A}
        \infer1[$\vee$I$_1$]{A \vee B}
        \hypo{[A]}
        \ellipsis{$\cD_2$}{C}
        \hypo{[B]}
        \ellipsis{$\cD_3$}{C}
        \infer3[$\vee$E]{C}
      }\qquad\Rightarrow_{\beta}\qquad\ebrule{
        \hypo{}
        \ellipsis{$\cD_1$}{A}
        \ellipsis{$\cD_2$}{C}
      }
    \end{mathpar}
  \end{onlyenv}
  \begin{onlyenv}<4>
    \begin{mathpar}
      \tmCase{\tmInl{a}}{c}{c'} \bRed c[a/x]
    \end{mathpar}
  \end{onlyenv}
  \begin{onlyenv}<5->
    \begin{figure}
      \centering
      \begin{mathpar}
        \tmCase{\tmInl{a}}{c}{c'} \bRed c[a/x]\and
        \tmCase{\tmInr{b}}{c}{c'} \bRed c'[b/y]\and
        \tmApp{\tmLam{x : A}{b}}{a} \bRed b[a/x]
      \end{mathpar}
      \caption{Selected $\beta$-reduction relations}
      \label{fig:0001}
    \end{figure}
    Whenever a term $t$ has one of the terms on the left-hand side above as a subterm, we can replace it with the corresponding term on the right-hand side, resulting in a new term $t'$.
    We say that $t$ $\beta$-reduces to $t'$, written $t \bRed t'$.
  \end{onlyenv}
\end{frame}

\begin{frame}{Some definitions}
  Some definitions before we proceed:
  \begin{notation}
    We write $\bRed*$ for the reflexive, transitive closure of $\bRed$.
    Intuitively, $t \bRed* t'$ means $t$ $\beta$-reduces to $t'$ in 0 or more steps.
  \end{notation}
  \begin{definition}
    A well-typed term $t$ is \emph{in normal form} if it is $\beta$-irreducible.
  \end{definition}
  \begin{definition}
    A well-typed term $t$ is \emph{normalizing} if there exists a normal form $t'$ such that $t \bRed* t'$.
  \end{definition}
  \begin{definition}
    A term rewriting system admits \emph{normalization} if every well-typed term is normalizing.
  \end{definition}
\end{frame}

\begin{frame}{}
  \begin{theorem}\label{0003}
    The computational dynamic defined herein admits normalization, \ie if $\Gamma \vdash t : A$, then $t$ is normalizing.
  \end{theorem}
  \begin{proof}
    We proceed by induction on the typing derivation $\Gamma \vdash t : A$.
    \begin{itemize}
      \begin{onlyenv}<1>
      \item[\rVar:] Variables are $\beta$-irreducible.
      \item[\rSumIl:] We must show that $\tmInl{b}\tpAnno{B + C}$ is normalizing.
        By the induction hypothesis, $b$ is normalizing, so there is a normal form $b'$ such that $b \bRed* b'$.
        Hence $\tmInl{b} \bRed* \tmInl{b'}$.
        Note that $\tmInl{b'}$ is in normal form.
      \end{onlyenv}
      \begin{onlyenv}<2->
      \item[\rSumE:] By the induction hypothesis, $s\tpAnno{B + C}$, $a\tpAnno{A}$, and $a'\tpAnno{A}$ are normalizing, but is $\tmCase{s}{a}{a'}$ normalizing?
      \end{onlyenv}
    \end{itemize}
    \begin{onlyenv}<3->
      We don't know!
      $s$ could $\beta$-reduce to $\tmInl{b}$.
      Then $\tmCase{s}{a}{a'}$ $\beta$-reduces to $a[b/x]$, which is not \apriori normalizing.
    \end{onlyenv}
  \end{proof}
\end{frame}

\section{Tait computability}

\begin{frame}{Tait computability}
  We glue additional data onto each type.
  \begin{itemize}
  \item For each context $\Gamma$ and type $A$, a set $\comp{\Gamma}{A}$ of \emph{evidently} normalizing terms of type $A$, \eg $\tmInl{a}$ provided that $a$ is normalizing.
  \item For each context $\Gamma$ and type $A$, a subset $\neu{\Gamma}{A} \subseteq \comp{\Gamma}{A}$ of eliminators stuck on variables, \eg variables $x$, or $\tmCase{x}{t}{t'}$ provided that $t$ and $t'$ are normalizing.
  \end{itemize}
  \begin{remark}
    $\comp{\Gamma}{A}$ and $\neu{\Gamma}{A}$ can be equivalently viewed as unary relations.
    We rewrite $\comp{\Gamma}{A}(t)$ to mean $t \in \comp{\Gamma}{A}$ and $\neu{\Gamma}{A}(t)$ similarly.
  \end{remark}
\end{frame}

\begin{frame}{Tait computability: goals}
  \begin{theorem}[Tait's yoga]
    If $\comp{\Gamma}{A}(t)$, then $t$ is normalizing.
  \end{theorem}
  \begin{theorem}[Fundamental Theorem]
    Let $\Gamma = x_1 : A_1,\ldots,x_n : A_n$.
    If $\Gamma \vdash t : A$ and $\comp{\Delta}{A_i}(t_i)$ for all $1 \leq i \leq n$, then $\comp{\Delta}{A}(t[t_i/x_i])$.
  \end{theorem}
  \begin{uncoverenv}<2->
    \begin{corollary}[Normalization]
      If $\Gamma \vdash t : A$, then $t$ is normalizing.
    \end{corollary}
    \begin{proof}
      Since each variable $x : A_i \in \Gamma$ is a stuck term, we have $\comp{\Gamma}{A_i}(x)$.
      This implies that $t[x_i/x_i] \equiv t \in \comp{\Gamma}{A}$.
      Hence $t$ is normalizing.
    \end{proof}
  \end{uncoverenv}
\end{frame}

\begin{frame}{Substantiating Tait computability: stuck terms}
  \begin{definition}
    A \emph{neutral element} $u$ is either a variable or an eliminator whose major argument is a neutral element.
  \end{definition}
  \begin{figure}
    \centering
    \begin{mathpar}
      \ebrule{
        \hypo{\neu{\Gamma}{A + B}(u)}
        \hypo{c\text{ normalizing}}
        \hypo{c'\text{ normalizing}}
        \infer3{\neu{\Gamma}{C}(\tmCase{u}{c}{c'})}
      }
    \end{mathpar}
    \caption{Selected rule for stuck terms}
  \end{figure}
\end{frame}

\begin{frame}{Substantiating Tait computability: evidently normalizing terms}
  We define $\comp{\Gamma}{A}$ by induction on $A$.
  \begin{definition}
    $\comp{\Gamma}{A + B}(t)$ iff
    \begin{enumerate}
    \item $t \bRed* \tmInl{a}$ for some $a$ such that $\comp{\Gamma}{A}(a)$;
    \item $t \bRed* \tmInr{b}$ for some $b$ such that $\comp{\Gamma}{B}(b)$; or
    \item $t \bRed* t'$ for some $t'$ such that $\neu{\Gamma}{A + B}(t')$.
    \end{enumerate}
  \end{definition}
\end{frame}

\begin{frame}{Verifying the theorems: Tait's yoga}
  \begin{theorem}[Tait's yoga]
    If $\comp{\Gamma}{A}(t)$, then $t$ is normalizing.
  \end{theorem}
  \begin{proof}
    By induction on $A$.
    We treat the case where $A = B + C$.
    Suppose that $\comp{\Gamma}{B + C}(t)$; then $t$ $\beta$-reduces to...
    \begin{enumerate}
    \item $\tmInl{b}$ with $b$ normalizing;
    \item $\tmInr{c}$ with $c$ normalizing; or
    \item a stuck term.
    \end{enumerate}
    In every case, $t$ is normalizing.
  \end{proof}
\end{frame}

\begin{frame}{Verifying the theorems: Fundamental Theorem}
  \begin{theorem}[Fundamental Theorem]
    Let $\Gamma = x_1 : A_1,\ldots,x_n : A_n$.
    If $\Gamma \vdash t : A$ and $\comp{\Delta}{A_i}(t_i)$ for all $1 \leq i \leq n$, then $\comp{\Delta}{A}(t[t_i/x_i])$.
  \end{theorem}
  \begin{proof}
    By induction on the typing derivation $\Gamma \vdash t : A$.
    We treat the $\rSumE$ case.
    By the induction hypothesis, $\comp{\Delta}{B + C}(s[t_i/x_i])$.
    If $s[t_i/x_i]$ $\beta$-reduces (WLOG) to $\tmInl{b}$ with $\comp{\Delta}{B}(b)$, then $\tmCase{s}{a}{a'}[t_i/x_i]$ $\beta$-reduces to $a[t_i/x_i,b/x]$, which is in $\comp{\Delta}{A}$ by the induction hypothesis.
    Now suppose that $s[t_i/x_i]$ $\beta$-reduces to a stuck term.
    Since variables are evidently normalizing, $a[t_i/x_i,x/x]$ and $a'[t_i/x_i,y/y]$ are normalizing by the induction hypothesis, but this implies
    \[
      \tmCase{s}{a}{a'}[t_i/x_i] \in \neu{\Delta}{A} \subseteq \comp{\Delta}{A}.
    \]
  \end{proof}
\end{frame}

\begin{frame}{Q \& A}
  
\end{frame}

\end{document}

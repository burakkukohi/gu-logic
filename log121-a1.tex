\documentclass[a4paper]{article}

\usepackage{preamble}
\addbibresource{references/refs.bib}

\title{LOG121 Hand-in 1}
\author{Frank Tsai}

\newcommand{\N}{\mathbb{N}}
\let\P\relax
\newcommand{\P}{\mathcal{P}}
\newcommand{\U}{\mathcal{U}}
\newcommand{\impl}{\Rightarrow}

\begin{document}

\maketitle

\section*{Problem 1}
\begin{enumerate}
\item\label{prob:1-1}
  \begin{proof}
    Let us proceed by induction on $i$.
    \begin{itemize}
    \item When $i = 0$, we need to show that $\N \subseteq \P(\N)$, i.e., $n \in \N \impl n \in \P(\N)$ for all $n \in \N$.
      Again, let us proceed by induction on $n$.
      \begin{itemize}
      \item When $n = 0$, we clearly have $\varnothing \in \P(\N)$.
      \item When $n = s(k)$, note that $s(k) = k \cup \{k\}$ by definition.
        By the induction hypothesis, $k \subseteq \N$.
        And clearly, $\{k\} \subseteq \N$.
        Thus, it follows that $k \cup \{k\} \subseteq \N$, i.e., $k \cup \{k\} \in \P(\N)$.
      \end{itemize}
    \item When $i = s(k)$, we need to show that $\U_{s(k)} \subseteq \U_{s(s(k))}$.
      Suppose that $x \in \U_{s(k)} = \P(\U_{k})$, or equivalently, that $x \subseteq \U_{k}$.
      By the induction hypothesis, $\U_{k} \subseteq \U_{s(k)}$, so $x \subseteq \U_{s(k)}$.
      But this means that $x \in \P(\U_{s(k)}) = \U_{s(s(k))}$.
    \end{itemize}
  \end{proof}
\item\label{prob:1-2}
  \begin{proof}
    Proceed by case analysis on $i$.
    \begin{itemize}
    \item When $i = 0$, we need to show that if $x \in y$ and $y \in \N$ then $x \in \N$.
      Let us proceed by induction on $y$.
      \begin{itemize}
      \item When $y = 0$, the thesis holds vacuously as $x \notin \varnothing$ for any $x$.
      \item When $y = s(k)$, suppose that $x \in k \cup \{k\}$.
        If $x \in k$, then $x \in \N$ by the induction hypothesis.
        If $x \in \{k\}$ (i.e., $x = k$ by the axiom of pairing), then we are done as $k \in \N$.
      \end{itemize}
    \item When $i = s(k)$, suppose that $x \in y$ and $y \in \U_{s(k)} = \P(\U_{k})$.
      Then since $y \subseteq \U_{k}$, $x \in \U_{k}$.
      By \ref{prob:1-1}, $\U_{k} \subseteq \U_{s(k)}$, so $x \in \U_{s(k)}$.
    \end{itemize}
    From \ref{prob:1-1} and \ref{prob:1-2}, we can deduce that for any $y \in \U_{i}$, we have that $y \in \U_{s(i)} = \P(\U_{i})$, or equivalently, $y \subseteq \U_{i}$.
  \end{proof}
\end{enumerate}

%\printbibliography

\end{document}

\documentclass[a4paper]{article}

\usepackage{preamble}
\addbibresource{references/refs.bib}

\title{LOG121 Exercise 1}
\author{Frank Tsai}

\begin{document}

\maketitle

\section{4.12}
Let $x$, $y$, and $z$ be sets.
\begin{enumerate}
\item There is a unique set $(x,y,z)$.
  \begin{proof}
    By construction, $(x,y,z) := (x,(y,z))$.
    By Ex. 4.10, this set is unique. \qedhere
  \end{proof}
\item This construction satisfies the ordered triple property.
  \begin{proof}
    Suppose $(x,y,z) = (u,v,w)$.
    Then $(x,(y,z)) = (u,(v,w))$.
    By the ordered pair property, $x = u$ and $(y,z) = (v,w)$.
    By the ordered pair property again, $y = v$ and $z = w$.
  \end{proof}
\item $((x,y),z)$ does not use the same set as $(x,(y,z))$ to represent ordered triples.
  \begin{proof}
    Suppose that $((x,y),z) = (x,(y,z))$.
    Then $x = (x,y)$ and $z = (y,z)$.
    Now choose $x = y = z = \varnothing$.
    Then $\varnothing = \{\{\varnothing\}\}$, which is a contradiction.
  \end{proof}
\end{enumerate}

\section{4.14}
\begin{enumerate}
\item No. Choose $x = \varnothing$, $y = \{\varnothing\}$, $z = \{\varnothing\}$, $u = \varnothing$, $v = \varnothing$, and $w = \{\varnothing\}$.
\item No. Choose $x = \varnothing$, $y = \{\varnothing\}$, $z = \varnothing$, $u = \{\varnothing\}$, $v = \varnothing$, and $w = \{\varnothing\}$.
\item Ditto.
\item Yes.
\item No. Choose $x = \varnothing$, $y = \{\varnothing\}$, $z = \varnothing$, $u = \{\varnothing\}$, $v = \varnothing$, and $w = \{\varnothing\}$.
\item Yes.
\end{enumerate}

% \printbibliography

\end{document}

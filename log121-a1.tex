\documentclass[a4paper]{article}

\usepackage{preamble}
\addbibresource{references/refs.bib}

\title{LOG121 Hand-in 1}
\author{Frank Tsai}

\newcommand{\N}{\mathbb{N}}
\let\P\relax
\newcommand{\P}{\mathcal{P}}
\newcommand{\U}{\mathcal{U}}
\newcommand{\B}{\mathcal{B}}
\newcommand{\impl}{\Rightarrow}
\let\iff\relax
\newcommand{\iff}{\Leftrightarrow}

\begin{document}

\maketitle

\section*{Problem 1}
\begin{enumerate}
\item\label{prob:1-1}
  \begin{proof}
    Let us proceed by induction on $i$.
    \begin{itemize}
    \item When $i = 0$, we need to show that $\N \subseteq \P(\N)$, i.e., $n \in \N \impl n \in \P(\N)$ for all $n \in \N$.
      Again, let us proceed by induction on $n$.
      \begin{itemize}
      \item When $n = 0$, we clearly have $\varnothing \in \P(\N)$.
      \item When $n = s(k)$, note that $s(k) = k \cup \{k\}$ by definition.
        By the induction hypothesis, $k \subseteq \N$.
        And clearly, $\{k\} \subseteq \N$.
        Thus, it follows that $k \cup \{k\} \subseteq \N$, i.e., $k \cup \{k\} \in \P(\N)$.
      \end{itemize}
    \item When $i = s(k)$, we need to show that $\U_{s(k)} \subseteq \U_{s(s(k))}$.
      Suppose that $x \in \U_{s(k)} = \P(\U_{k})$, or equivalently, that $x \subseteq \U_{k}$.
      By the induction hypothesis, $\U_{k} \subseteq \U_{s(k)}$, so $x \subseteq \U_{s(k)}$.
      But this means that $x \in \P(\U_{s(k)}) = \U_{s(s(k))}$.
    \end{itemize}
  \end{proof}
\item\label{prob:1-2}
  \begin{proof}
    Proceed by case analysis on $i$.
    \begin{itemize}
    \item When $i = 0$, we need to show that if $x \in y$ and $y \in \N$ then $x \in \N$.
      Let us proceed by induction on $y$.
      \begin{itemize}
      \item When $y = 0$, the thesis holds vacuously as $x \notin \varnothing$ for any $x$.
      \item When $y = s(k)$, suppose that $x \in k \cup \{k\}$.
        If $x \in k$, then $x \in \N$ by the induction hypothesis.
        If $x \in \{k\}$ (i.e., $x = k$ by the axiom of pairing), then we are done as $k \in \N$.
      \end{itemize}
    \item When $i = s(k)$, suppose that $x \in y$ and $y \in \U_{s(k)} = \P(\U_{k})$.
      Then since $y \subseteq \U_{k}$, $x \in \U_{k}$.
      By \ref{prob:1-1}, $\U_{k} \subseteq \U_{s(k)}$, so $x \in \U_{s(k)}$.
    \end{itemize}
    From \ref{prob:1-1} and \ref{prob:1-2}, we can deduce that for any $y \in \U_{i}$, we have that $y \in \U_{s(i)} = \P(\U_{i})$, or equivalently, $y \subseteq \U_{i}$.
  \end{proof}
\end{enumerate}

\section*{Problem 2}
\begin{enumerate}
\item
  \begin{itemize}
  \item Axiom of extensionality.
    \begin{proof}
      Let $x,y \in \U$.
      The only if direction is trivial.
      For the if direction, suppose that $z \in x \iff z \in y$ for all $z \in \U$.
      To show $x = y$, we appeal to the ``external'' axiom of extensionality and show that the assumed hypothesis is sufficiently strong.

      Let $z'$ be an arbitrary set (not necessarily a priori in $\U$).
      We need to show that $z' \in x \iff z' \in y$.
      If $z' \in x$, then since $x \in \U_{i}$ for some $i$, $z' \in \U_{i}$ by transitivity, but this implies that $z' \in \U$.
      Thus, $z' \in y$ follows from the assumed hypothesis.
      The converse is completely analogous.
    \end{proof}
    In the proofs that follow, we follow the same strategy: we show that the universe $\U$ is closed under a given construction; then we can appeal to the external version of the relevant axiom to conclude the proof.
  \item Axiom of empty set.
    \begin{proof}
      Note that $\varnothing \in \U_{1} = \P(\N)$.
    \end{proof}
  \item Axiom of pairing.
    \begin{proof}
      Let $x,y \in \U$.
      Since $\U_{i} \subseteq \U_{s(i)}$ for all $i$, it is legitimate to assume that $x,y \in \U_{i}$ for some $i$.
      Then it follows that $\{x,y\} \subseteq \U_{i}$, i.e., $\{x,y\} \in \U_{s(i)}$.
      The rest of the statement follows from the external axiom of pairing.
    \end{proof}
  \item Axiom schema of separation.
    \begin{proof}
      Let $\varphi$ be any formula in the language of ZF set theory with free variables among $z,p_{1},\ldots,p_{n}$ and let $x \in \U$.
      It suffices to show that $y := \{z \in x \mid \varphi(z,p_{1},\ldots,p_{n})\} \in \U$.
      Since $x \in \U_{i}$ for some $i$, $x \subseteq \U_{i}$ by Problem 1.
      Then since $y \subseteq x$, $y \subseteq \U_{i}$, i.e., $y \in \U_{s(i)}$.
    \end{proof}
  \item Axiom of powerset.
    \begin{proof}
      It suffices to show that the universe $\U$ is closed under the powerset construction.
      The rest of the argument follows from the external axiom of powerset.

      Let $x \in \U$.
      Then $x \in \U_{i}$ for some $i$, so $x \subseteq \U_{i}$ by Problem 1.
      Then it follows that $y \subseteq \U_{i}$ for each subset of $x$.
      In other words, each subset of $x$ lives in $\U_{s(i)}$.
      Thus, the set containing exactly these sets, i.e., $\P(x)$, is a subset of $\U_{s(i)}$, so the powerset lives in $\U_{s(s(i))}$.
    \end{proof}
  \item Axiom of union.
    \begin{proof}
      It suffices to show that the universe $\U$ is closed under union.
      Let $x \in \U$, then $x \in \U_{i}$ for some $i$.
      For any arbitrary $y \in \bigcup x$, the external axiom of union implies that there is a set $v$ such that $y \in v$ and $v \in x$.
      Thus, by transitivity, $v \in \U_{i}$ and (by transitivity again) $y \in \U_{i}$.
      In other words, $\bigcup x \subseteq \U_{i}$, so $\bigcup x \in \U_{s(i)}$.
    \end{proof}
  \item Axiom of infinity.
    \begin{proof}
      Note that $\N \in \P(\N) = V_{1}$.
    \end{proof}
  \item Axiom of foundation.
    \begin{proof}
      Let $x \in \U$ be a nonempty set.
      By the external axiom of foundation, there is a set $z \in x$ such that $x \cap z = \varnothing$, so it suffices to show that $z$ lives in the universe $\U$.
      This follows immediately from transitivity: since $x \in \U_{i}$ for some $i$ and $z \in x$ by assumption, we have that $z \in \U_{i}$.
    \end{proof}
  \end{itemize}
\item
  \begin{proof}
    Consider $\varphi(x,y) := y = \P(x)$, or more concretely,
    \[
      \varphi(x,y) := \forall z.\,%
      z \in y \iff%
      (\forall a.\, a \in z \iff (\forall b.\, b \in a \impl b \in x)).
    \]
    The recursion scheme then yields the definable function $n \mapsto \U_{n}$.
    The image of this function under $\N$ is $\U$, which is too large to live in $\U$ by the axiom of foundation.
  \end{proof}
\end{enumerate}

\section*{Problem 3}
To model the axioms of extensionality, empty set, and foundation, it suffices to choose $\B = \{\varnothing\}$.

To model the axiom of pairing, we need to extend this universe.
Let us construct a universe as follows:
\begin{align*}
  \B := \bigcup\{\B_{i} \mid i \in \N\} && \text{with} && \B_{0} := \{\varnothing\} && \B_{s(n)} := \P(\B_{n}).
\end{align*}

This is (probably) more than enough to model the axiom of pairing.

Using the same argument in Problems 1 and 2, we can show that this universe is (1) transitive and (2) closed under pairing.
Thus, $\B$ provides a model for ZF3 without losing $\B$-bounded extensionality and foundation.

In fact, $\B$ is closed under all constructions guaranteed by ZF3-6, so $\B$ actually provides a model for ZF1-6,9.

Now, observe that the powerset of a finite set is finite, so every set in $\B$ is finite.
Thus, ZF1-6,9 cannot guarantee the existence of infinite sets.

%\printbibliography

\end{document}

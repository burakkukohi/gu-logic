\documentclass[a4paper]{article}

\usepackage{preamble}
\addbibresource{references/refs.bib}

\title{LOG111 Hand-in 2}
\author{Frank Tsai}

\begin{document}

\maketitle

\section{Problem 1.15(2)}
\begin{proof}
  Since $\uparrow$ is expressively adequate, it suffices to show that $\circ_{7}$ and $\to$ can express $\uparrow$.
  \begin{center}
    \begin{tabular}[h]{cc|c}
      $\varphi$ & $\psi$ & $\psi \to (\varphi \circ_{7} \psi)$\\
      T & T & F\\
      T & F & T\\
      F & T & T\\
      F & F & T\\
    \end{tabular}
  \end{center}
\end{proof}

\section{Problem 2.2(13)}\label{sec:problem2}
\[
  \begin{prooftree}
    \hypo{[\varphi \wedge \psi]^{1}}
    \infer1[$\wedge$E$_{2}$]{\psi}
    \hypo{\chi \to \varphi \to \psi}
    \hypo{[\chi]^{2}}
    \infer2[$\to$E]{\varphi \to \lnot \psi}
    \hypo{[\varphi \wedge \psi]^{1}}
    \infer1[$\wedge$E$_{1}$]{\varphi}
    \infer2[$\to$E]{\lnot \psi}
    \infer2[$\lnot$E]{\bot}
    \infer1[$\lnot$I$^{2}$]{\lnot \chi}
    \infer1[$\to$I$^{1}$]{\varphi \wedge \psi \to \lnot \chi}
  \end{prooftree}
\]

\section{Problem 3.4}
\begin{proof}
  The if direction is an immediate consequence of monotonicity (Prop 1.19(4)).
  It remains to prove the only if direction.
  
  Suppose that $\Gamma \models \varphi$.
  Then by completeness, $\Gamma \vdash \varphi$.
  Then it follows from compactness (of $\vdash$) and soundness that there is a finite subset $\Gamma_{0} \subseteq \Gamma$ such that $\Gamma_{0} \models \varphi$.
\end{proof}

\section{Problem 3.5}
\begin{enumerate}
\item
  \begin{proof}
    Suppose that $\varphi \vee \psi \in \Gamma$.
    Then since $\Gamma$ is complete, either $\varphi \in \Gamma$ or $\lnot \varphi \in \Gamma$.
    The former case proves the thesis.
    In the latter case, we have either $\psi \in \Gamma$ or $\lnot \psi \in \Gamma$.
    Again, the former case proves the thesis.
    In the latter case, consider the unsatisfiable subset $\varphi \vee \psi, \lnot \varphi, \lnot \psi$.
    This contradicts finite satisfiability.

    Conversely, suppose that either $\varphi \in \Gamma$ or $\psi \in \Gamma$.
    Completeness demands that either $\varphi \vee \psi \in \Gamma$ or $\lnot(\varphi \vee \psi) \in \Gamma$.
    The latter case is impossible because neither $\lnot(\varphi \vee \psi), \varphi$ nor $\lnot(\varphi \vee \psi), \psi$ is satisfiable.
  \end{proof}
\item
  \begin{proof}
    Suppose that $\varphi \to \psi \in \Gamma$.
    If $\varphi \in \Gamma$ then it must be the case that $\psi \in \Gamma$ because $\varphi \to \psi, \varphi, \lnot\psi$ is not satisfiable.
    If $\varphi \notin \Gamma$ then the thesis follows immediately.

    Conversely, suppose that either $\varphi \notin \Gamma$ or $\psi \in \Gamma$.
    It suffices to show that $\lnot(\varphi \to \psi) \notin \Gamma$.
    To this end, we show that we can always construct an unsatisfiable subset.
    Note that neither $\lnot(\varphi \to \psi), \lnot \varphi$ nor $\lnot(\varphi \to \psi), \varphi, \psi$ is satisfiable.

    If $\varphi \notin \Gamma$ then $\lnot\varphi \in \Gamma$, so we can choose the first set.
    If $\psi \in \Gamma$ then we choose the first set if $\varphi \notin \Gamma$, otherwise we choose the second set.
  \end{proof}
\end{enumerate}

\section{Problem 4.9}
\begin{enumerate}
\item The formula is satisfied because $\mathscr{M}, s \models A(f(z),c) \to \forall y.\,A(y,3) \vee A(f(y),3)$.
\item Let $\mathscr{M}'$ be the same structure as $\mathscr{M}$ except the relation $A^{\mathscr{M}'}$ is $\{\langle2,3\rangle\}$ and $f^{\mathscr{M}'}$ maps $1$ to $1$, and let $s'$ be a variable assignment mapping every variable to $3$.
  Then $\mathscr{M}',s' \nvDash \exists x.\,A(f(z),c) \to \forall y.\,A(y,x) \vee A(f(y),x)$.
\end{enumerate}

\section{}
\begin{itemize}
\item Rules:
  \[
    \begin{prooftree}
      \hypo{[\varphi]^{1}}
      \ellipsis{}{\psi}
      \hypo{[\lnot\varphi]^{1}}
      \ellipsis{}{\chi}
      \infer2[$\diamond$I$^1$]{\diamond(\varphi,\psi,\chi)}
    \end{prooftree}
    \qquad
    \begin{prooftree}
      \hypo{\diamond(\varphi,\psi,\chi)}
      \hypo{\varphi}
      \infer2[$\diamond$E$_{1}$]{\psi}
    \end{prooftree}
    \qquad
    \begin{prooftree}
      \hypo{\diamond(\varphi,\psi,\chi)}
      \hypo{\lnot\varphi}
      \infer2[$\diamond$E$_{2}$]{\chi}
    \end{prooftree}
  \]
\item Soundness: It suffices to add three additional cases to Theorem 2.22.

  When the last rule is $\diamond\text{I}$, the induction hypothesis yields $\Gamma,\varphi \models \psi$ and $\Gamma,\lnot\varphi \models \chi$.
  Let $v$ be any valuation such that $v \models \Gamma$.
  If $v \models \varphi$, then $v \models \psi$.
  On the other hand, if $v \models \lnot\varphi$, then $v \models \chi$.
  Thus, $v \models \diamond(\varphi,\psi,\chi)$ by definition.

  When the last rule is $\diamond\text{E}_{1}$, the induction hypothesis yields $\Gamma \models \diamond(\varphi,\psi,\chi)$ and $\Gamma \models \varphi$.
  Then it follows that $\Gamma,\varphi \models \psi$.
  Thus, transitivity yields $\Gamma \models \psi$.

  The last case ($\diamond\text{E}_{2}$) is completely analogous to the previous case.
\item Completeness: Note that every lemma that the Completeness Theorem (Corollary 3.7) relies on can be repaired.
  The only one that requires additional work is Proposition 3.2.
  We need to show that if $\Gamma$ is complete and consistent, then $\diamond(\varphi,\psi,\chi) \in \Gamma$ iff either $\psi \in \Gamma$ whenever $\varphi \in \Gamma$ or $\chi \in \Gamma$ whenever $\lnot \varphi \in \Gamma$.
  \begin{proof}
    Suppose that $\diamond(\varphi,\psi,\chi) \in \Gamma$.
    If $\varphi \in \Gamma$ and $\lnot\psi \in \Gamma$ then clearly $\Gamma \vdash \psi$, but this implies that $\Gamma$ is inconsistent, contradicting the hypothesis.
    Similarly, if $\lnot\varphi\in \Gamma$ and $\lnot \chi \in \Gamma$ then $\Gamma$ is inconsistent.

    Assume the converse.
    If $\varphi \in \Gamma$ and $\psi \in \Gamma$, then $\Gamma \vdash \diamond(\varphi,\psi,\chi)$.
    Similarly, if $\lnot\varphi \in \Gamma$ and $\chi \in \Gamma$, then $\Gamma \vdash \diamond(\varphi,\psi,\chi)$.
    Thus, it follows that $\diamond(\varphi,\psi,\chi) \in \Gamma$ because $\Gamma$ is complete and consistent.
    \[
      \begin{prooftree}
        \hypo{\psi}
        \hypo{\varphi}
        \hypo{[\lnot\varphi]}
        \infer2{\chi}
        \infer2{\diamond(\varphi,\psi,\chi)}
      \end{prooftree}
      \qquad
      \begin{prooftree}
        \hypo{[\varphi]}
        \hypo{\lnot \varphi}
        \infer2{\psi}
        \hypo{\chi}
        \infer2{\diamond(\varphi,\psi,\chi)}
      \end{prooftree}
    \]
  \end{proof}
\end{itemize}

\section{}
This would reduce the number of provable formulas.
For example, \cref{sec:problem2} requires two discharges because we need to eliminate the same conjunction twice to obtain enough information for the negation elimination.
A simpler example is $\vdash \varphi \wedge \psi \to \psi \wedge \varphi$.
This requires two eliminations on $\varphi \wedge \psi$.

% \printbibliography

\end{document}

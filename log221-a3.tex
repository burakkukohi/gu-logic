\documentclass[a4paper]{article}

\usepackage{preamble}
\usepackage[final]{showkeys}
\newcommand{\gtrans}[1]{\prn{#1}^{\mathrm{g}}}

\title{LOG221 - A3}
\author{Frank Tsai}

\begin{document}

\maketitle

\section*{Admissible rules}

We freely use the admissible rules listed in \cref{fig:0000,fig:0001}.
Note that every admissible rule in the intuitionistic sequent calculus is also admissible in the classical sequent calculus.

\begin{figure}[h]
  \centering
  \begin{mathpar}
    \ebrule[R$\lnot\lnot$]{
      \hypo{\Gamma \Rightarrow A}
      \infer1{\Gamma \Rightarrow \lnot\lnot A}
    }\and
    \ebrule[L$\lnot$]{
      \hypo{\Gamma \Rightarrow A}
      \infer1{\Gamma, \lnot A \Rightarrow \bot}
    }\and
    \ebrule[Con]{
      \hypo{\Gamma,A \Rightarrow B}
      \infer1{\Gamma,\lnot B \Rightarrow \lnot A}
    }\and
    \ebrule[LR$\lnot\lnot$]{
      \hypo{\Gamma,A \Rightarrow B}
      \infer1{\Gamma,\lnot\lnot A \Rightarrow \lnot\lnot B}
    }
  \end{mathpar}

  \caption{Admissible rules in ISC}
  \label{fig:0000}
\end{figure}
\begin{figure}[h]
  \centering
  \begin{mathpar}
    \ebrule[L$\lnot\lnot$]{
      \hypo{\Gamma,A \Rightarrow \Delta}
      \infer1{\Gamma,\lnot\lnot A \Rightarrow \Delta}
    }
  \end{mathpar}
  \caption{Admissible rules in CSC}
  \label{fig:0001}
\end{figure}


\section{}
In light of the cut elimination theorem, we may freely use the cut rule since every instance of it can be removed after the fact.
We proceed by induction on $A$; the cases where $A = p$, $A = B \vee C$, and $A = \exists x. B(x)$ are not applicable.
\begin{itemize}
\item $A = \bot$:\qquad
  $
  \ebrule{
    \hypo{\bot \Rightarrow \bot}
    \infer1[R$\to$]{\Rightarrow \lnot \bot}
    \infer1[L$\lnot$]{\lnot\lnot\bot \Rightarrow \bot}
  }
  $
\item $A = B \wedge C$: By the induction hypothesis, we have $\vdash_0 \lnot\lnot B \Rightarrow B$ and $\vdash_0 \lnot\lnot C \Rightarrow C$; hence we have $\vdash_0 \lnot\lnot B \wedge \lnot\lnot C \Rightarrow B \wedge C$ by applying the right conjunction rule followed by the left conjunction rule.
  It suffices to show that $\vdash \lnot\lnot(B \wedge C) \Rightarrow \lnot\lnot B \wedge \lnot\lnot C$; then the required sequent is obtained by an application of the cut rule.
  \[
    \ebrule{
      \hypo{B,C \Rightarrow B}
      \infer1[L$\wedge$]{B \wedge C \Rightarrow B}
      \infer1[LR$\lnot\lnot$]{\lnot\lnot(B \wedge C) \Rightarrow \lnot\lnot B}
      \hypo{}
      \ellipsis{analogous}{\lnot\lnot(B \wedge C) \Rightarrow \lnot\lnot C}
      \infer2[R$\wedge$]{\lnot\lnot(B \wedge C) \Rightarrow \lnot\lnot B \wedge \lnot\lnot C}
    }
  \]
\item $A = B \to C$: By the induction hypothesis, we have $\vdash_0 \lnot\lnot C \Rightarrow C$; weakening gives $\vdash \lnot\lnot C, B \Rightarrow C$.
  We show that $\lnot\lnot(B \to C),B \Rightarrow \lnot\lnot C$; then the desired sequent is obtained by an application of the cut rule followed by R$\to$.
  \[
    \ebrule{
      \hypo{B \to C,B \Rightarrow B}
      \hypo{B \to C,B,C \Rightarrow C}
      \infer2[L$\to$]{B \to C,B \Rightarrow C}
      \infer1[LR$\lnot\lnot$]{\lnot\lnot(B \to C),B \Rightarrow \lnot\lnot C}
    }
  \]
\item $A = \forall x.A(x)$: Let $a$ be a constant not occurring in $A$.
  By the induction hypothesis, we have $\vdash \lnot\lnot A(a) \Rightarrow A(a)$.
  Hence we have the following derivation:
  \[
    \ebrule{
      \hypo{\lnot\lnot A(a) \Rightarrow A(a)}
      \infer1[L$\forall$]{\forall x.\lnot\lnot A(x) \Rightarrow A(a)}
      \infer1[R$\forall$]{\forall x.\lnot\lnot A(x) \Rightarrow \forall x.A(x)}
    }
  \]
  Note that the eigenvariable condition is satisfied since $a$ does not occur in $A$.
  It suffices to show that $\vdash \lnot\lnot\forall x.A(x) \Rightarrow \forall x.\lnot\lnot A(x)$; the desired sequent is then obtained by an application of the cut rule.
  \[
    \ebrule{
      \hypo{A(a) \Rightarrow A(a)}
      \infer1[L$\forall$]{\forall x.A(x) \Rightarrow A(a)}
      \infer1[LR$\lnot\lnot$]{\lnot\lnot\forall x.A(x) \Rightarrow \lnot\lnot A(a)}
      \infer1[R$\forall$]{\lnot\lnot\forall x.A(x) \Rightarrow \forall x.\lnot\lnot A(x)}
    }
  \]
  Again, the eigenvariable condition is satisfied since $a$ does not occur in $A$.
\end{itemize}

As a consequence, the following rules are admissible in ISC:
\begin{mathpar}
  \ebrule[L$\lnot\lnot$]{
    \hypo{\Gamma,A \Rightarrow B}
    \infer1{I \vdash \Gamma,\lnot\lnot A \Rightarrow B}
  }\and
  \ebrule[R$\lnot\lnot$-Inv]{
    \hypo{\Gamma \Rightarrow \lnot\lnot A}
    \infer1{I \vdash \Gamma \Rightarrow A}
  }\and
  \ebrule[L$\lnot$-Inv]{
    \hypo{\Gamma, \lnot A \Rightarrow \bot}
    \infer1{I \vdash \Gamma \Rightarrow A}
  }\and
  \text{for any negative formula $A$}
\end{mathpar}

\section{}

\begin{lemma}\label{thm:0003}
  $C \vdash \gtrans{A} \Rightarrow A$ and $C \vdash A \Rightarrow \gtrans{A}$.
\end{lemma}
\begin{proof}
  We proceed by induction on $A$.
  \begin{itemize}
  \item $A = p$:
    \begin{mathpar}
      \ebrule{
        \hypo{p \Rightarrow p}
        \infer1[L$\lnot\lnot$]{\lnot\lnot p \Rightarrow p}
      }\and
      \ebrule{
        \hypo{p \Rightarrow p}
        \infer1[R$\lnot\lnot$]{p \Rightarrow \lnot\lnot p}
      }
    \end{mathpar}
  \item $A = \bot$: This case is trivial since $\gtrans{\bot} = \bot$.
  \item $A = B \wedge C$: By the induction hypothesis, we have $\vdash \gtrans{B} \Rightarrow B$, $\vdash \gtrans{C} \Rightarrow C$, $\vdash B \Rightarrow \gtrans{B}$, and $\vdash C \Rightarrow \gtrans{C}$.
    \begin{mathpar}
      \ebrule{
        \hypo{\gtrans{B}, \gtrans{C} \Rightarrow B}
        \hypo{\gtrans{B}, \gtrans{C} \Rightarrow C}
        \infer2[R$\wedge$]{\gtrans{B}, \gtrans{C} \Rightarrow B \wedge C}
        \infer1[L$\wedge$]{\gtrans{B} \wedge \gtrans{C} \Rightarrow B \wedge C}
      }\and
      \ebrule{
        \hypo{}
        \ellipsis{analogous}{B \wedge C \Rightarrow \gtrans{B} \wedge \gtrans{C}}
      }
    \end{mathpar}
  \item $A = B \vee C$: By the induction hypothesis, we have $\vdash \gtrans{B} \Rightarrow B$, $\vdash \gtrans{C} \Rightarrow C$, $\vdash B \Rightarrow \gtrans{B}$, and $\vdash C \Rightarrow \gtrans{C}$.
    \begin{mathpar}
      \ebrule{
        \hypo{\gtrans{B} \Rightarrow B, C}
        \infer1[R$\to$]{\Rightarrow B, C, \lnot\gtrans{B}}
        \hypo{\gtrans{C} \Rightarrow B, C}
        \infer1[R$\to$]{\Rightarrow B, C, \lnot\gtrans{C}}
        \infer2[R$\wedge$]{\Rightarrow B, C, \lnot\gtrans{B} \wedge \lnot\gtrans{C}}
        \infer1[R*$\vee$]{\Rightarrow B \vee C, \lnot\gtrans{B} \wedge \lnot\gtrans{C}}
        \infer1[L$\lnot$]{\lnot(\lnot\gtrans{B} \wedge \lnot\gtrans{C}) \Rightarrow B \vee C}
      }\and
      \ebrule{
        \hypo{B, \lnot\gtrans{C} \Rightarrow \gtrans{B}}
        \infer1[L$\lnot$]{B, \lnot\gtrans{B}, \lnot\gtrans{C} \Rightarrow \bot}
        \hypo{C, \lnot\gtrans{B} \Rightarrow \gtrans{C}}
        \infer1[L$\lnot$]{C, \lnot\gtrans{B}, \lnot\gtrans{C} \Rightarrow \bot}
        \infer2[L$\vee$]{B \vee C, \lnot\gtrans{B}, \lnot\gtrans{C} \Rightarrow \bot}
        \infer1[L$\wedge$]{B \vee C, \lnot\gtrans{B} \wedge \lnot\gtrans{C} \Rightarrow \bot}
        \infer1[R$\to$]{B \vee C \Rightarrow \lnot(\lnot\gtrans{B} \wedge \lnot\gtrans{C})}
      }
    \end{mathpar}
    Note that R*$\vee$ is an admissible rule.
  \item $A = B \to C$: By the induction hypothesis, we have $\vdash \gtrans{B} \Rightarrow B$, $\vdash \gtrans{C} \Rightarrow C$, $\vdash B \Rightarrow \gtrans{B}$, and $\vdash C \Rightarrow \gtrans{C}$.
    \begin{mathpar}
      \ebrule{
        \hypo{B \Rightarrow \gtrans{B}}
        \hypo{\gtrans{C} \Rightarrow C}
        \infer2[L$\to$]{\gtrans{B} \to \gtrans{C}, B \Rightarrow C}
        \infer1[R$\to$]{\gtrans{B} \to \gtrans{C} \Rightarrow B \to C}
      }\and
      \ebrule{
        \hypo{}
        \ellipsis{analogous}{B \to C \Rightarrow \gtrans{B} \to \gtrans{C}}
      }
    \end{mathpar}
  \item $A = \forall x.B(x)$: By the induction hypothesis, $\vdash \gtrans{B(s)} \Rightarrow B(s)$ and $\vdash B(s) \Rightarrow \gtrans{B(s)}$ for any $s$.
    Let $b$ be a constant not occurring in $B$.
    \begin{mathpar}
      \ebrule{
        \hypo{\gtrans{B(b)} \Rightarrow B(b)}
        \infer1[L$\forall$]{\forall x.\gtrans{B(x)} \Rightarrow B(b)}
        \infer1[R$\forall$]{\forall x.\gtrans{B(x)} \Rightarrow \forall x.B(x)}
      }\and
      \ebrule{
        \hypo{}
        \ellipsis{analogous}{\forall x.B(x) \Rightarrow \forall x.\gtrans{B(x)}}
      }
    \end{mathpar}
    The eigenvariable condition is satisfied since $b$ does not occur in $B$.
  \item $A = \exists x.B(x)$: By the induction hypothesis, $\vdash \gtrans{B(s)} \Rightarrow B(s)$ and $\vdash B(s) \Rightarrow \gtrans{B(s)}$ for any $s$.
    Let $b$ be a constant not occurring in $B$.
    \begin{mathpar}
      \ebrule{
        \hypo{\gtrans{B(b)} \Rightarrow B(b)}
        \infer1[R$\to$]{\Rightarrow B(b), \lnot \gtrans{B(b)}}
        \infer1[R$\exists$]{\Rightarrow \exists x.B(x), \lnot \gtrans{B(b)}}
        \infer1[R$\forall$]{\Rightarrow \exists x.B(x), \forall x.\lnot\gtrans{B(x)}}
        \infer1[L$\lnot$]{\lnot(\forall x.\lnot\gtrans{B(x)}) \Rightarrow \exists x.B(x)}
      }\and
      \ebrule{
        \hypo{B(b) \Rightarrow \gtrans{B(b)}}
        \infer1[L$\lnot$]{B(b), \lnot\gtrans{B(b)} \Rightarrow \bot}
        \infer1[L$\forall$]{B(b), \forall x.\lnot\gtrans{B(x)} \Rightarrow \bot}
        \infer1[L$\exists$]{\exists x.B(x), \forall x.\lnot\gtrans{B(x)} \Rightarrow \bot}
        \infer1[R$\to$]{\exists x.B(x) \Rightarrow \lnot(\forall x.\lnot\gtrans{B(x)})}
      }
    \end{mathpar}
    The eigenvariable condition is satisfied since $b$ does not occur in $B$.
  \end{itemize}
\end{proof}

\begin{remark}\label{rmk:0000}
  Observe that the only places where classical rules are required in \cref{thm:0003} are when $A = p$, $A = B \vee C$, and $A = \exists x.B(x)$; hence if we restrict $A$ to negative formulas, then we also have $I \vdash \gtrans{A} \Rightarrow A$ and $I \vdash A \Rightarrow \gtrans{A}$.
\end{remark}

\cref{thm:0003} together with cut elimination imply that the CSC admits the following invertible rules:
\begin{mathpar}
  \ebrule{
    \hypo{\Gamma \Rightarrow \gtrans{A}}
    \infer[double]1[R$g$]{\Gamma \Rightarrow A}
  }\and
  \ebrule{
    \hypo{\Gamma, \gtrans{A} \Rightarrow \Delta}
    \infer[double]1[L$g$]{\Gamma, A \Rightarrow \Delta}
  }
\end{mathpar}

L$g$ facilitates the proof of the following lemma.
\begin{lemma}\label{thm:0006}
  If $C \vdash \gtrans{\Gamma} \Rightarrow \Delta$, then $C \vdash \Gamma \Rightarrow \Delta$.
\end{lemma}
\begin{proof}
  We proceed by induction on the length of $\Gamma$; the base case is trivial.
  Suppose that $\Gamma = \Pi, A$ and $C \vdash \gtrans{\Pi},\gtrans{A} \Rightarrow \Delta$.
  We push $\gtrans{A}$ to the right-hand side via R$\lnot$; then by the induction hypothesis, we have $C \vdash \Pi \Rightarrow \lnot\gtrans{A},\Delta$.
  Now, we can push $\gtrans{A}$ back to the left-hand side and apply L$g$.
  \[
    \ebrule{
      \hypo{\Pi,\gtrans{A} \Rightarrow \Delta}
      \infer1[L$g$]{\Pi,A \Rightarrow \Delta}
    }
  \]
\end{proof}

\begin{lemma}\label{thm:000A}
  If $C \vdash \Gamma \Rightarrow \Delta$, then $I \vdash \gtrans{\Gamma}, \lnot\gtrans{\Delta} \Rightarrow \bot$.
\end{lemma}
\begin{proof}
  We proceed by induction on the derivation of $C \vdash \Gamma \Rightarrow \Delta$.
  \begin{itemize}
  \item id: In this case, there is an atom $p \in \Gamma \cap \Delta$; hence $\lnot\lnot p \in \gtrans{\Gamma}$ and $\lnot\lnot\lnot p \in \lnot\gtrans{\Delta}$.
    Then the result follows by applying L$\to$ on $\lnot\lnot\lnot p$.
  \item L$\bot$: Since $\bot \in \Gamma$ and $\gtrans{\bot} = \bot$, the required sequent is an instance of L$\bot$.
  \item R$\wedge$: By the induction hypothesis, we have $I \vdash \gtrans{\Gamma}, \lnot\gtrans{\Delta}, \lnot\gtrans{A} \Rightarrow \bot$ and $I \vdash \gtrans{\Gamma}, \lnot\gtrans{\Delta}, \lnot\gtrans{B} \Rightarrow \bot$.
    \begin{mathpar}
      \ebrule{
        \hypo{\gtrans{\Gamma},\lnot\gtrans{\Delta},\lnot\gtrans{A} \Rightarrow \bot}
        \infer1[L$\lnot$-Inv]{\gtrans{\Gamma},\lnot\gtrans{\Delta} \Rightarrow \gtrans{A}}
        \hypo{\gtrans{\Gamma},\lnot\gtrans{\Delta},\lnot\gtrans{B} \Rightarrow \bot}
        \infer1[L$\lnot$-Inv]{\gtrans{\Gamma},\lnot\gtrans{\Delta} \Rightarrow \gtrans{B}}
        \infer2[R$\wedge$]{\gtrans{\Gamma},\lnot\gtrans{\Delta} \Rightarrow \gtrans{A} \wedge \gtrans{B}}
        \infer1[L$\lnot$]{I \vdash \gtrans{\Gamma},\lnot\gtrans{\Delta},\lnot(\gtrans{A} \wedge \gtrans{B}) \Rightarrow \bot}
      }
    \end{mathpar}
  \item L$\wedge$: Immediate by applying L$\wedge$ to the induction hypothesis.
  \item R$\vee$: Without loss of generality, we assume that the rule applied is R$\vee_1$.
    By the induction hypothesis, we have $I \vdash \gtrans{\Gamma},\lnot\gtrans{\Delta},\lnot\gtrans{A} \Rightarrow \bot$.
    \begin{mathpar}
      \ebrule{
        \hypo{\gtrans{\Gamma},\lnot\gtrans{\Delta},\lnot\gtrans{A}, \lnot\gtrans{B} \Rightarrow \bot}
        \infer1[L$\wedge$]{\gtrans{\Gamma},\lnot\gtrans{\Delta},\lnot\gtrans{A} \wedge \lnot\gtrans{B} \Rightarrow \bot}
        \infer1[L$\lnot\lnot$]{%
          \gtrans{\Gamma},\lnot\gtrans{\Delta},\lnot\lnot(\lnot\gtrans{A} \wedge \lnot\gtrans{B}) \Rightarrow \bot
        }
      }
    \end{mathpar}
  \item L$\vee$: By the induction hypothesis, we have $I \vdash \gtrans{\Gamma},\lnot\gtrans{\Delta},\gtrans{A} \Rightarrow \bot$ and $I \vdash \gtrans{\Gamma},\lnot\gtrans{\Delta},\gtrans{B} \Rightarrow \bot$.
    \begin{mathpar}
      \ebrule{
        \hypo{\gtrans{\Gamma},\lnot\gtrans{\Delta},\gtrans{A} \Rightarrow \bot}
        \infer1[R$\to$]{\gtrans{\Gamma},\lnot\gtrans{\Delta} \Rightarrow \lnot\gtrans{A}}
        \hypo{\gtrans{\Gamma},\lnot\gtrans{\Delta},\gtrans{B} \Rightarrow \bot}
        \infer1[R$\to$]{\gtrans{\Gamma},\lnot\gtrans{\Delta} \Rightarrow \lnot\gtrans{B}}
        \infer2[R$\wedge$]{\gtrans{\Gamma},\lnot\gtrans{\Delta} \Rightarrow \lnot\gtrans{A} \wedge \lnot\gtrans{B}}
        \infer1[L$\lnot$]{\gtrans{\Gamma},\lnot\gtrans{\Delta},\lnot(\lnot\gtrans{A} \wedge \lnot\gtrans{B}) \Rightarrow \bot}
      }
    \end{mathpar}
  \item R$\to$: By the induction hypothesis, we have $I \vdash \gtrans{\Gamma},\lnot\gtrans{\Delta},\gtrans{A},\lnot\gtrans{B} \Rightarrow \bot$.
    \begin{mathpar}
      \ebrule{
        \hypo{\gtrans{\Gamma},\lnot\gtrans{\Delta},\gtrans{A},\lnot\gtrans{B} \Rightarrow \bot}
        \infer1[L$\lnot$-Inv]{\gtrans{\Gamma},\lnot\gtrans{\Delta},\gtrans{A} \Rightarrow \gtrans{B}}
        \infer1[R$\to$]{\gtrans{\Gamma},\lnot\gtrans{\Delta} \Rightarrow \gtrans{A} \to \gtrans{B}}
        \infer1[L$\lnot$]{\gtrans{\Gamma},\lnot\gtrans{\Delta},\lnot(\gtrans{A} \to \gtrans{B}) \Rightarrow \bot}
      }
    \end{mathpar}
  \item L$\to$: By the induction hypothesis, we have $I \vdash \gtrans{\Gamma},\lnot\gtrans{\Delta},\lnot\gtrans{A} \Rightarrow \bot$ and $I \vdash \gtrans{\Gamma},\lnot\gtrans{\Delta},\gtrans{B} \Rightarrow \bot$.
    \begin{mathpar}
      \ebrule{
        \hypo{\gtrans{\Gamma},\lnot\gtrans{\Delta},\lnot\gtrans{A} \Rightarrow \bot}
        \infer1[L$\lnot$-Inv]{\gtrans{\Gamma},\lnot\gtrans{\Delta} \Rightarrow \gtrans{A}}
        \hypo{\gtrans{\Gamma},\lnot\gtrans{\Delta},\gtrans{B} \Rightarrow \bot}
        \infer2[L$\to$]{\gtrans{\Gamma},\lnot\gtrans{\Delta},\gtrans{A} \to \gtrans{B} \Rightarrow \bot}
      }
    \end{mathpar}
  \item R$\forall$: By the induction hypothesis we have $I \vdash \gtrans{\Gamma},\lnot\gtrans{\Delta},\lnot \gtrans{A(a)} \Rightarrow \bot$, where $a$ is fresh for $\Gamma$, $\Delta$, and $\forall x.A(x)$.
    \begin{mathpar}
      \ebrule{
        \hypo{\gtrans{\Gamma},\lnot\gtrans{\Delta},\lnot \gtrans{A(a)} \Rightarrow \bot}
        \infer1[L$\lnot\lnot$-Inv]{\gtrans{\Gamma},\lnot\gtrans{\Delta} \Rightarrow \gtrans{A(a)}}
        \infer1[R$\forall$]{\gtrans{\Gamma},\lnot\gtrans{\Delta} \Rightarrow \forall x.\gtrans{A(x)}}
        \infer1[L$\lnot$]{\gtrans{\Gamma},\lnot\gtrans{\Delta},\lnot\forall x.\gtrans{A(x)} \Rightarrow \bot}
      }
    \end{mathpar}
    The eigenvariable condition is satisfied since it is satisfied in the original derivation.
  \item L$\forall$: Immediate by applying L$\forall$ to the induction hypothesis.
  \item R$\exists$: By the induction hypothesis, we have $I \vdash \gtrans{\Gamma},\lnot\gtrans{\Delta},\lnot\gtrans{A(t)} \Rightarrow \bot$ for some term $t$.
    \begin{mathpar}
      \ebrule{
        \hypo{\gtrans{\Gamma},\lnot\gtrans{\Delta},\lnot \gtrans{A(t)} \Rightarrow \bot}
        \infer1[L$\forall$]{\gtrans{\Gamma},\lnot\gtrans{\Delta},\forall x.\lnot \gtrans{A(x)} \Rightarrow \bot}
        \infer1[L$\lnot\lnot$]{\gtrans{\Gamma},\lnot\gtrans{\Delta},\lnot\lnot\forall x.\lnot \gtrans{A(x)} \Rightarrow \bot}
      }
    \end{mathpar}
  \item L$\exists$: By the induction hypothesis, we have $I \vdash \gtrans{\Gamma},\lnot\gtrans{\Delta},\gtrans{A(a)} \Rightarrow \bot$.
    \begin{mathpar}
      \ebrule{
        \hypo{\gtrans{\Gamma},\lnot\gtrans{\Delta},\gtrans{A(a)} \Rightarrow \bot}
        \infer1[R$\to$]{\gtrans{\Gamma},\lnot\gtrans{\Delta} \Rightarrow \lnot\gtrans{A(a)}}
        \infer1[R$\forall$]{\gtrans{\Gamma},\lnot\gtrans{\Delta} \Rightarrow \forall x.\lnot\gtrans{A(x)}}
        \infer1[L$\lnot$]{\gtrans{\Gamma},\lnot\gtrans{\Delta},\lnot\forall x.\lnot\gtrans{A(x)} \Rightarrow \bot}
      }
    \end{mathpar}
    The eigenvariable condition is satisfied since it is satisfied in the original derivation.
  \end{itemize}
\end{proof}

We show the main theorem separately.

\begin{proposition}\label{thm:0004}
  If $C \vdash \Gamma \Rightarrow A$ then $I \vdash \gtrans{\Gamma} \Rightarrow \gtrans{A}$.
\end{proposition}
\begin{proof}
  Suppose that $C \vdash \Gamma \Rightarrow A$.
  By \cref{thm:000A}, we have $I \vdash \gtrans{\Gamma},\lnot\gtrans{A} \Rightarrow \bot$.
  Since $\lnot\gtrans{A}$ is a negative formula, we may apply \textsc{L$\lnot$-Inv} to obtain the required sequent.
\end{proof}

\begin{proposition}\label{thm:0005}
  If $I \vdash \gtrans{\Gamma} \Rightarrow \gtrans{A}$, then $C \vdash \Gamma \Rightarrow A$.
\end{proposition}
\begin{proof}
  Suppose that $I \vdash \gtrans{\Gamma} \Rightarrow \gtrans{A}$.
  Since every intuitionistic proof is automatically a classical proof, we have $C \vdash \gtrans{\Gamma} \Rightarrow \gtrans{A}$.
  By R$g$, we have $C \vdash \gtrans{\Gamma} \Rightarrow A$; then the result follows by \cref{thm:0006}.
\end{proof}

\cref{thm:0004,thm:0005} imply the main result.

\section{}

Everything provable intuitionistically is provable classically; hence it suffices to show that $C \vdash \Rightarrow A$ implies $I \vdash \Rightarrow A$ for all negative formulas $A$.

Suppose that $C \vdash \Rightarrow A$.
By \cref{thm:0004}, we have $I \vdash \Rightarrow \gtrans{A}$.
Since $A$ is negative, we have $I \vdash \gtrans{A} \Rightarrow A$ by \cref{rmk:0000}; then we obtain the required sequent by a cut:
\[
  \ebrule{
    \hypo{\Rightarrow \gtrans{A}}
    \hypo{\gtrans{A} \Rightarrow A}
    \infer2[Cut]{I \vdash \Rightarrow A}
  }
\]

\section{}

By Problems 2 and 3, the translation $\gtrans{-}$ embeds classical logic into intuitionistic logic \emph{conservatively}---everything provable intuitionistically is (trivially) provable classically and, moreover, anything that one want to prove classically can be translated and proved intuitionistically.

In fact, the construction of the embedding implies that every classical connective can be simulated by intuitionistic-$\bot$, $\to$, $\wedge$, and $\forall$.
In this sense, intuitionistic logic extends classical logic by two \emph{very} strict logical operators, $\vee$ and $\exists$, satisfying the disjunction and existential properties respectively.

% \bibliography{bib}
% \bibliographystyle{alpha}

\end{document}

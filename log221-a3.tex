\documentclass[a4paper]{article}

\usepackage{preamble}

\title{LOG221 - A3}
\author{Frank Tsai}

\begin{document}

\maketitle

\section*{Derived rules}

We freely use the following derived rules.
\begin{mathpar}
  \ebrule[R$\lnot\lnot$]{
    \hypo{\Gamma \Rightarrow A}
    \infer1{\Gamma \Rightarrow \lnot\lnot A}
  }\and
  \ebrule[L$\lnot$]{
    \hypo{\Gamma \Rightarrow A}
    \infer1{\Gamma, \lnot A \Rightarrow \bot}
  }\and
  \ebrule[Con]{
    \hypo{\Gamma,A \Rightarrow B}
    \infer1{\Gamma,\lnot B \Rightarrow \lnot A}
  }\and
  \ebrule[LR$\lnot\lnot$]{
    \hypo{\Gamma,A \Rightarrow B}
    \infer1{\Gamma,\lnot\lnot A \Rightarrow \lnot\lnot B}
  }
\end{mathpar}

\section{}
In light of the cut elimination theorem, we may freely use the cut rule since every instance of it can be removed after the fact; additionally, we freely use derived rules to keep derivations short.
We proceed by induction on $A$.
\begin{itemize}
\item $A = \bot$:\qquad
  $
  \ebrule{
    \hypo{\bot \Rightarrow \bot}
    \infer1[R$\to$]{\Rightarrow \lnot \bot}
    \infer1[L$\lnot$]{\lnot\lnot\bot \Rightarrow \bot}
  }
  $
\item $A = B \wedge C$: By the induction hypothesis, we have $\vdash_0 \lnot\lnot B \Rightarrow B$ and $\vdash_0 \lnot\lnot C \Rightarrow C$; hence we have $\vdash_0 \lnot\lnot B \wedge \lnot\lnot C \Rightarrow B \wedge C$ by applying the right conjunction rule followed by the left conjunction rule.
  It suffices to show that $\vdash \lnot\lnot(B \wedge C) \Rightarrow \lnot\lnot B \wedge \lnot\lnot C$; then the required sequent is obtained by an application of the cut rule.
  \[
    \ebrule{
      \hypo{B,C \Rightarrow B}
      \infer1[L$\wedge$]{B \wedge C \Rightarrow B}
      \infer1[LR$\lnot\lnot$]{\lnot\lnot(B \wedge C) \Rightarrow \lnot\lnot B}
      \hypo{}
      \ellipsis{analogous}{\lnot\lnot(B \wedge C) \Rightarrow \lnot\lnot C}
      \infer2[R$\wedge$]{\lnot\lnot(B \wedge C) \Rightarrow \lnot\lnot B \wedge \lnot\lnot C}
    }
  \]
\item $A = B \to C$: By the induction hypothesis, we have $\vdash_0 \lnot\lnot B \Rightarrow B$ and $\vdash_0 \lnot\lnot C \Rightarrow C$.
  Weakening the latter sequent gives $\vdash \lnot\lnot C, B \Rightarrow C$.
  We show that $\lnot\lnot(B \to C),B \Rightarrow \lnot\lnot C$; then the desired sequent is obtained by an application of the cut rule followed by R$\to$.
  \[
    \ebrule{
      \hypo{B \to C,B \Rightarrow B}
      \hypo{B \to C,B,C \Rightarrow C}
      \infer2[L$\to$]{B \to C,B \Rightarrow C}
      \infer1[LR$\lnot\lnot$]{\lnot\lnot(B \to C),B \Rightarrow \lnot\lnot C}
    }
  \]
\item $A = \forall x.A(x)$: Let $a$ be a constant not occurring in $A$.
  By the induction hypothesis, we have $\vdash \lnot\lnot A(a) \Rightarrow A(a)$.
  Hence
  \[
    \ebrule{
      \hypo{\lnot\lnot A(a) \Rightarrow A(a)}
      \infer1[L$\forall$]{\forall x.\lnot\lnot A(x) \Rightarrow A(a)}
      \infer1[R$\forall$]{\forall x.\lnot\lnot A(x) \Rightarrow \forall x.A(x)}
    }
  \]
  Note that the eigenvariable condition is satisfied since $a$ does not occur in $A$.
  It suffices to show that $\vdash \lnot\lnot\forall x.A(x) \Rightarrow \forall x.\lnot\lnot A(x)$; the desired sequent is then obtained by an application of the cut rule.
  \[
    \ebrule{
      \hypo{A(a) \Rightarrow A(a)}
      \infer1[L$\forall$]{\forall x.A(x) \Rightarrow A(a)}
      \infer1[LR$\lnot\lnot$]{\lnot\lnot\forall x.A(x) \Rightarrow \lnot\lnot A(a)}
      \infer1[R$\forall$]{\lnot\lnot\forall x.A(x) \Rightarrow \forall x.\lnot\lnot A(x)}
    }
  \]
  Again, the eigenvariable condition is satisfied since $a$ does not occur in $A$.
\end{itemize}

As a consequence, the following rule is admissible.
\begin{mathpar}
  \ebrule[L$\lnot\lnot$]{
    \hypo{\Gamma, A \Rightarrow B}
    \infer1{\Gamma, \lnot\lnot A \Rightarrow B}
  }\qquad\text{for any negative formula $A$}
\end{mathpar}

\section{}

We show that if $\Gamma$ consists of negative formulas only and $A$ is negative, then $I \vdash \Gamma \Rightarrow A$ implies $C \vdash \Gamma \Rightarrow A$.

% \bibliography{bib}
% \bibliographystyle{alpha}

\end{document}

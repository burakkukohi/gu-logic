\documentclass[a4paper]{article}

\usepackage{preamble}

\newcommand{\gtrans}[1]{\prn{#1}^{\mathrm{g}}}

\title{LOG221 - A3}
\author{Frank Tsai}

\begin{document}

\maketitle

\section*{Admissible rules}

We freely use the admissible rules listed in \cref{fig:0000,fig:0001}.
Note that every admissible rule for the intuitionistic sequent calculus is also admissible for the classical sequent calculus.

\begin{figure}[h]
  \centering
  \begin{mathpar}
    \ebrule[R$\lnot\lnot$]{
      \hypo{\Gamma \Rightarrow A}
      \infer1{\Gamma \Rightarrow \lnot\lnot A}
    }\and
    \ebrule[L$\lnot$]{
      \hypo{\Gamma \Rightarrow A}
      \infer1{\Gamma, \lnot A \Rightarrow \bot}
    }\and
    \ebrule[Con]{
      \hypo{\Gamma,A \Rightarrow B}
      \infer1{\Gamma,\lnot B \Rightarrow \lnot A}
    }\and
    \ebrule[LR$\lnot\lnot$]{
      \hypo{\Gamma,A \Rightarrow B}
      \infer1{\Gamma,\lnot\lnot A \Rightarrow \lnot\lnot B}
    }\and
    \ebrule[Contra]{
      \hypo{\Gamma,A,\lnot A \Rightarrow B}
    }
  \end{mathpar}
  \caption{Admissible rules in ISC}
  \label{fig:0000}
\end{figure}
\begin{figure}[h]
  \centering
  \begin{mathpar}
    \ebrule[L$\lnot\lnot$]{
      \hypo{\Gamma,A \Rightarrow \Delta}
      \infer1{\Gamma,\lnot\lnot A \Rightarrow \Delta}
    }\and
    \ebrule[Lem]{
      \hypo{\Gamma \Rightarrow A, \lnot A}
    }
  \end{mathpar}
  \caption{Admissible rules in CSC}
  \label{fig:0001}
\end{figure}


\section{}
In light of the cut elimination theorem, we may freely use the cut rule since every instance of it can be removed after the fact.
We proceed by induction on $A$.
\begin{itemize}
\item $A = \bot$:\qquad
  $
  \ebrule{
    \hypo{\bot \Rightarrow \bot}
    \infer1[R$\to$]{\Rightarrow \lnot \bot}
    \infer1[L$\lnot$]{\lnot\lnot\bot \Rightarrow \bot}
  }
  $
\item $A = B \wedge C$: By the induction hypothesis, we have $\vdash_0 \lnot\lnot B \Rightarrow B$ and $\vdash_0 \lnot\lnot C \Rightarrow C$; hence we have $\vdash_0 \lnot\lnot B \wedge \lnot\lnot C \Rightarrow B \wedge C$ by applying the right conjunction rule followed by the left conjunction rule.
  It suffices to show that $\vdash \lnot\lnot(B \wedge C) \Rightarrow \lnot\lnot B \wedge \lnot\lnot C$; then the required sequent is obtained by an application of the cut rule.
  \[
    \ebrule{
      \hypo{B,C \Rightarrow B}
      \infer1[L$\wedge$]{B \wedge C \Rightarrow B}
      \infer1[LR$\lnot\lnot$]{\lnot\lnot(B \wedge C) \Rightarrow \lnot\lnot B}
      \hypo{}
      \ellipsis{analogous}{\lnot\lnot(B \wedge C) \Rightarrow \lnot\lnot C}
      \infer2[R$\wedge$]{\lnot\lnot(B \wedge C) \Rightarrow \lnot\lnot B \wedge \lnot\lnot C}
    }
  \]
\item $A = B \to C$: By the induction hypothesis, we have $\vdash_0 \lnot\lnot C \Rightarrow C$; weakening gives $\vdash \lnot\lnot C, B \Rightarrow C$.
  We show that $\lnot\lnot(B \to C),B \Rightarrow \lnot\lnot C$; then the desired sequent is obtained by an application of the cut rule followed by R$\to$.
  \[
    \ebrule{
      \hypo{B \to C,B \Rightarrow B}
      \hypo{B \to C,B,C \Rightarrow C}
      \infer2[L$\to$]{B \to C,B \Rightarrow C}
      \infer1[LR$\lnot\lnot$]{\lnot\lnot(B \to C),B \Rightarrow \lnot\lnot C}
    }
  \]
\item $A = \forall x.A(x)$: Let $a$ be a constant not occurring in $A$.
  By the induction hypothesis, we have $\vdash \lnot\lnot A(a) \Rightarrow A(a)$.
  Hence we have the following derivation:
  \[
    \ebrule{
      \hypo{\lnot\lnot A(a) \Rightarrow A(a)}
      \infer1[L$\forall$]{\forall x.\lnot\lnot A(x) \Rightarrow A(a)}
      \infer1[R$\forall$]{\forall x.\lnot\lnot A(x) \Rightarrow \forall x.A(x)}
    }
  \]
  Note that the eigenvariable condition is satisfied since $a$ does not occur in $A$.
  It suffices to show that $\vdash \lnot\lnot\forall x.A(x) \Rightarrow \forall x.\lnot\lnot A(x)$; the desired sequent is then obtained by an application of the cut rule.
  \[
    \ebrule{
      \hypo{A(a) \Rightarrow A(a)}
      \infer1[L$\forall$]{\forall x.A(x) \Rightarrow A(a)}
      \infer1[LR$\lnot\lnot$]{\lnot\lnot\forall x.A(x) \Rightarrow \lnot\lnot A(a)}
      \infer1[R$\forall$]{\lnot\lnot\forall x.A(x) \Rightarrow \forall x.\lnot\lnot A(x)}
    }
  \]
  Again, the eigenvariable condition is satisfied since $a$ does not occur in $A$.
\end{itemize}

As a consequence, the following rules are admissible in ISC:
\begin{mathpar}
  \ebrule[L$\lnot\lnot$]{
    \hypo{\Gamma,A \Rightarrow B}
    \infer1{I \vdash \Gamma,\lnot\lnot A \Rightarrow B}
  }\and
  \ebrule[R$\lnot\lnot$-Inv]{
    \hypo{\Gamma \Rightarrow \lnot\lnot A}
    \infer1{I \vdash \Gamma \Rightarrow A}
  }\and
  \text{for any negative formula $A$}
\end{mathpar}

\section{}

\begin{lemma}\label{thm:0003}
  $C \vdash \gtrans{A} \Rightarrow A$ and $C \vdash A \Rightarrow \gtrans{A}$.
\end{lemma}
\begin{proof}
  We proceed by induction on $A$.
  \begin{itemize}
  \item $A = p$:
    \begin{mathpar}
      \ebrule{
        \hypo{p \Rightarrow p}
        \infer1[L$\lnot\lnot$]{\lnot\lnot p \Rightarrow p}
      }\and
      \ebrule{
        \hypo{p \Rightarrow p}
        \infer1[R$\lnot\lnot$]{p \Rightarrow \lnot\lnot p}
      }
    \end{mathpar}
  \item $A = \bot$: This case is trivial since $\gtrans{\bot} = \bot$.
  \item $A = B \wedge C$: By the induction hypothesis, we have $\vdash \gtrans{B} \Rightarrow B$, $\vdash \gtrans{C} \Rightarrow C$, $\vdash B \Rightarrow \gtrans{B}$, and $\vdash C \Rightarrow \gtrans{C}$.
    \begin{mathpar}
      \ebrule{
        \hypo{\gtrans{B}, \gtrans{C} \Rightarrow B}
        \hypo{\gtrans{B}, \gtrans{C} \Rightarrow C}
        \infer2[R$\wedge$]{\gtrans{B}, \gtrans{C} \Rightarrow B \wedge C}
        \infer1[L$\wedge$]{\gtrans{B} \wedge \gtrans{C} \Rightarrow B \wedge C}
      }\and
      \ebrule{
        \hypo{}
        \ellipsis{analogous}{B \wedge C \Rightarrow \gtrans{B} \wedge \gtrans{C}}
      }
    \end{mathpar}
  \item $A = B \vee C$: By the induction hypothesis, we have $\vdash \gtrans{B} \Rightarrow B$, $\vdash \gtrans{C} \Rightarrow C$, $\vdash B \Rightarrow \gtrans{B}$, and $\vdash C \Rightarrow \gtrans{C}$.
    \begin{mathpar}
      \ebrule{
        \hypo{\gtrans{B} \Rightarrow B, C}
        \infer1[R$\to$]{\Rightarrow B, C, \lnot\gtrans{B}}
        \hypo{\gtrans{C} \Rightarrow B, C}
        \infer1[R$\to$]{\Rightarrow B, C, \lnot\gtrans{C}}
        \infer2[R$\wedge$]{\Rightarrow B, C, \lnot\gtrans{B} \wedge \lnot\gtrans{C}}
        \infer1[R*$\vee$]{\Rightarrow B \vee C, \lnot\gtrans{B} \wedge \lnot\gtrans{C}}
        \infer1[L$\lnot$]{\lnot(\lnot\gtrans{B} \wedge \lnot\gtrans{C}) \Rightarrow B \vee C}
      }\and
      \ebrule{
        \hypo{B, \lnot\gtrans{C} \Rightarrow \gtrans{B}}
        \infer1[L$\lnot$]{B, \lnot\gtrans{B}, \lnot\gtrans{C} \Rightarrow \bot}
        \hypo{C, \lnot\gtrans{B} \Rightarrow \gtrans{C}}
        \infer1[L$\lnot$]{C, \lnot\gtrans{B}, \lnot\gtrans{C} \Rightarrow \bot}
        \infer2[L$\vee$]{B \vee C, \lnot\gtrans{B}, \lnot\gtrans{C} \Rightarrow \bot}
        \infer1[L$\wedge$]{B \vee C, \lnot\gtrans{B} \wedge \lnot\gtrans{C} \Rightarrow \bot}
        \infer1[R$\to$]{B \vee C \Rightarrow \lnot(\lnot\gtrans{B} \wedge \lnot\gtrans{C})}
      }
    \end{mathpar}
  \item $A = B \to C$: By the induction hypothesis, we have $\vdash \gtrans{B} \Rightarrow B$, $\vdash \gtrans{C} \Rightarrow C$, $\vdash B \Rightarrow \gtrans{B}$, and $\vdash C \Rightarrow \gtrans{C}$.
    \begin{mathpar}
      \ebrule{
        \hypo{B \Rightarrow \gtrans{B}}
        \hypo{\gtrans{C} \Rightarrow C}
        \infer2[L$\to$]{\gtrans{B} \to \gtrans{C}, B \Rightarrow C}
        \infer1[R$\to$]{\gtrans{B} \to \gtrans{C} \Rightarrow B \to C}
      }\and
      \ebrule{
        \hypo{}
        \ellipsis{analogous}{B \to C \Rightarrow \gtrans{B} \to \gtrans{C}}
      }
    \end{mathpar}
  \item $A = \forall x.B(x)$: By the induction hypothesis, $\vdash \gtrans{B(s)} \Rightarrow B(s)$ and $\vdash B(s) \Rightarrow \gtrans{B(s)}$ for any $s$.
    Let $b$ be a constant not occurring in $B$.
    \begin{mathpar}
      \ebrule{
        \hypo{\gtrans{B(b)} \Rightarrow B(b)}
        \infer1[L$\forall$]{\forall x.\gtrans{B(x)} \Rightarrow B(b)}
        \infer1[R$\forall$]{\forall x.\gtrans{B(x)} \Rightarrow \forall x.B(x)}
      }\and
      \ebrule{
        \hypo{}
        \ellipsis{analogous}{\forall x.B(x) \Rightarrow \forall x.\gtrans{B(x)}}
      }
    \end{mathpar}
    The eigenvariable condition is satisfied since $b$ does not occur in $B$.
  \item $A = \exists x.B(x)$: By the induction hypothesis, $\vdash \gtrans{B(s)} \Rightarrow B(s)$ and $\vdash B(s) \Rightarrow \gtrans{B(s)}$ for any $s$.
    Let $b$ be a constant not occurring in $B$.
    \begin{mathpar}
      \ebrule{
        \hypo{\gtrans{B(b)} \Rightarrow B(b)}
        \infer1[R$\to$]{\Rightarrow B(b), \lnot \gtrans{B(b)}}
        \infer1[R$\exists$]{\Rightarrow \exists x.B(x), \lnot \gtrans{B(b)}}
        \infer1[R$\forall$]{\Rightarrow \exists x.B(x), \forall x.\lnot\gtrans{B(x)}}
        \infer1[L$\lnot$]{\lnot(\forall x.\lnot\gtrans{B(x)}) \Rightarrow \exists x.B(x)}
      }\and
      \ebrule{
        \hypo{B(b) \Rightarrow \gtrans{B(b)}}
        \infer1[L$\lnot$]{B(b), \lnot\gtrans{B(b)} \Rightarrow \bot}
        \infer1[L$\forall$]{B(b), \forall x.\lnot\gtrans{B(x)} \Rightarrow \bot}
        \infer1[L$\exists$]{\exists x.B(x), \forall x.\lnot\gtrans{B(x)} \Rightarrow \bot}
        \infer1[R$\to$]{\exists x.B(x) \Rightarrow \lnot(\forall x.\lnot\gtrans{B(x)})}
      }
    \end{mathpar}
    The eigenvariable condition is satisfied since $b$ does not occur in $B$.
  \end{itemize}
\end{proof}

\begin{remark}\label{rmk:0000}
  Observe that the only places where classical rules are required in \cref{thm:0003} are when $A = p$, $A = B \vee C$, and $A = \exists x.B(x)$; moreover, it is easy to verify that $I \vdash \lnot p \Rightarrow \lnot\lnot\lnot p$ and $I \vdash \lnot\lnot\lnot p \Rightarrow \lnot p$ for all atoms $p$.
  Hence if we restrict $A$ to negative formulas, then we also have $I \vdash \gtrans{A} \Rightarrow A$ and $I \vdash A \Rightarrow \gtrans{A}$.
\end{remark}

\cref{thm:0003} together with cut elimination imply that the CSC admits the following invertible rules:
\begin{mathpar}
  \ebrule{
    \hypo{\Gamma \Rightarrow \gtrans{A}}
    \infer[double]1[R$g$]{\Gamma \Rightarrow A}
  }\and
  \ebrule{
    \hypo{\Gamma, \gtrans{A} \Rightarrow \Delta}
    \infer[double]1[L$g$]{\Gamma, A \Rightarrow \Delta}
  }
\end{mathpar}

L$g$ facilitates the proof of the following lemma.
\begin{lemma}\label{thm:0006}
  If $C \vdash \gtrans{\Gamma} \Rightarrow \Delta$, then $C \vdash \Gamma \Rightarrow \Delta$.
\end{lemma}
\begin{proof}
  We proceed by induction on the length of $\Gamma$; the base case is trivial.
  Suppose that $\Gamma = \Pi, A$ and $C \vdash \gtrans{\Pi},\gtrans{A} \Rightarrow \Delta$.
  We push $\gtrans{A}$ to the right-hand side via R$\lnot$; then by the induction hypothesis, we have $C \vdash \Pi \Rightarrow \lnot\gtrans{A},\Delta$.
  Now, we can push $\gtrans{A}$ back to the left-hand side and apply L$g$.
  \[
    \ebrule{
      \hypo{\Pi,\gtrans{A} \Rightarrow \Delta}
      \infer1[L$g$]{\Pi,A \Rightarrow \Delta}
    }
  \]
\end{proof}

We show the main theorem separately.

\begin{proposition}\label{thm:0004}
  If $C \vdash \Gamma \Rightarrow A$ then $I \vdash \gtrans{\Gamma} \Rightarrow \gtrans{A}$.
\end{proposition}
\begin{proof}
  We proceed by induction on the derivation of $C \vdash \Gamma \Rightarrow A$.
  Since $\to$, $\wedge$, and $\forall$ commute with $\gtrans{-}$, the required sequent, in these cases, is obtained by applying the corresponding rule to the induction hypothesis; we omit these cases.
  \begin{itemize}
  \item id: In this case, $A \in \Gamma$.
    Hence $\gtrans{A} \in \gtrans{\Gamma}$; then it follows that $I \vdash \gtrans{\Gamma} \Rightarrow \gtrans{A}$ (not necessarily through id).
  \item $\bot$: Since $\gtrans{\bot} = \bot$, $I \vdash \gtrans{\Gamma} \Rightarrow \gtrans{A}$ is an instance of L$\bot$.
  \item R$\vee$:
    $
    \ebrule{
      \hypo{\Gamma \Rightarrow B}
      \infer1{C \vdash \Gamma \Rightarrow B \vee C}
    }
    $.

    There are two right rules, but it suffices to show one of them since the other one is completely analogous.
    By the induction hypothesis, we have $I \vdash \gtrans{\Gamma} \Rightarrow \gtrans{B}$.
    \begin{mathpar}
      \ebrule{
        \hypo{\gtrans{\Gamma}, \lnot\gtrans{C} \Rightarrow \gtrans{B}}
        \infer1[L$\lnot$]{\gtrans{\Gamma}, \lnot\gtrans{B}, \lnot\gtrans{C} \Rightarrow \bot}
        \infer1[L$\wedge$]{\gtrans{\Gamma}, \lnot\gtrans{B} \wedge \lnot\gtrans{C} \Rightarrow \bot}
        \infer1[R$\to$]{I \vdash \gtrans{\Gamma} \Rightarrow \lnot(\lnot\gtrans{B} \wedge \lnot\gtrans{C})}
      }
    \end{mathpar}
  \item L$\vee$:
    $
    \ebrule{
      \hypo{\Gamma, B \Rightarrow A}
      \hypo{\Gamma, C \Rightarrow A}
      \infer2{C \vdash \Gamma, B \vee C \Rightarrow A}
    }
    $.

    By the induction hypothesis, we have $I \vdash \gtrans{\Gamma},\gtrans{B} \Rightarrow \gtrans{A}$ and $I \vdash \gtrans{\Gamma},\gtrans{C} \Rightarrow \gtrans{A}$.
    \[
      \ebrule{
        \hypo{\gtrans{\Gamma}, \gtrans{B} \Rightarrow \gtrans{A}}
        \infer1[Con]{\gtrans{\Gamma}, \lnot\gtrans{A} \Rightarrow \lnot\gtrans{B}}
        \hypo{\gtrans{\Gamma}, \gtrans{C} \Rightarrow \gtrans{A}}
        \infer1[Con]{\gtrans{\Gamma}, \lnot\gtrans{A} \Rightarrow \lnot\gtrans{C}}
        \infer2[R$\wedge$]{\gtrans{\Gamma}, \lnot\gtrans{A} \Rightarrow \lnot\gtrans{B} \wedge \lnot\gtrans{C}}
        \infer1[Con]{\gtrans{\Gamma}, \lnot(\lnot\gtrans{B} \wedge \lnot\gtrans{C}) \Rightarrow \lnot\lnot\gtrans{A}}
        \infer1[R$\lnot\lnot$-Inv]{I \vdash \gtrans{\Gamma}, \lnot(\lnot\gtrans{B} \wedge \lnot\gtrans{C}) \Rightarrow \gtrans{A}}
      }
    \]
  \item R$\exists$:
    $
    \ebrule{
      \hypo{\Gamma \Rightarrow B(t)}
      \infer1{C \vdash \Gamma \Rightarrow \exists x.B(x)}
    }.
    $

    By the induction hypothesis, we have $I \vdash \gtrans{\Gamma} \Rightarrow \gtrans{B(t)}$.
    \[
      \ebrule{
        \hypo{\gtrans{\Gamma} \Rightarrow \gtrans{B(t)}}
        \infer1[L$\lnot$]{\gtrans{\Gamma}, \lnot\gtrans{B(t)} \Rightarrow \bot}
        \infer1[L$\forall$]{\gtrans{\Gamma}, \forall x.\lnot\gtrans{B(x)} \Rightarrow \bot}
        \infer1[R$\to$]{I \vdash \gtrans{\Gamma} \Rightarrow \lnot\forall x.\lnot\gtrans{B(x)}}
      }
    \]
  \item L$\exists$:
    $
    \ebrule{
      \hypo{\Gamma, B(b) \Rightarrow A}
      \infer1{C \vdash \Gamma, \exists x.B(x) \Rightarrow A}
    }
    $.

    By the induction hypothesis, we have $I \vdash \gtrans{\Gamma}, \gtrans{B(b)} \Rightarrow \gtrans{A}$.
    \begin{mathpar}
      \ebrule{
        \hypo{\gtrans{\Gamma}, \gtrans{B(b)} \Rightarrow \gtrans{A}}
        \infer1[Con]{\gtrans{\Gamma}, \lnot\gtrans{A} \Rightarrow \lnot \gtrans{B(b)}}
        \infer1[R$\forall$]{\gtrans{\Gamma}, \lnot\gtrans{A} \Rightarrow \forall x.\lnot\gtrans{B(x)}}
        \infer1[Con]{\gtrans{\Gamma}, \lnot\forall x.\lnot \gtrans{B(x)} \Rightarrow \lnot\lnot\gtrans{A}}
        \infer1[R$\lnot\lnot$-Inv]{I \vdash \gtrans{\Gamma}, \lnot\forall x.\lnot \gtrans{B(x)} \Rightarrow \gtrans{A}}
      }
    \end{mathpar}
    The eigenvariable condition is satisfied since it is satisfied in the original derivation.
  \end{itemize}
\end{proof}

\begin{proposition}\label{thm:0005}
  If $I \vdash \gtrans{\Gamma} \Rightarrow \gtrans{A}$, then $C \vdash \Gamma \Rightarrow A$.
\end{proposition}
\begin{proof}
  Suppose that $I \vdash \gtrans{\Gamma} \Rightarrow \gtrans{A}$.
  Since every intuitionistic proof is automatically a classical proof, we have $C \vdash \gtrans{\Gamma} \Rightarrow \gtrans{A}$.
  By R$g$, we have $C \vdash \gtrans{\Gamma} \Rightarrow A$; then the result follows by \cref{thm:0006}.
\end{proof}

\cref{thm:0004,thm:0005} imply the main result.

\section{}

Everything provable intuitionistically is provable classically; hence it suffices to show that $C \vdash \Rightarrow A$ implies $I \vdash \Rightarrow A$ for all negative formulas $A$.

By \cref{thm:0004}, we have $I \vdash \Rightarrow \gtrans{A}$.
Since $A$ is negative, we have $I \vdash \gtrans{A} \Rightarrow A$ by \cref{rmk:0000}; then we obtain the required sequent by a cut:
\[
  \ebrule{
    \hypo{\Rightarrow \gtrans{A}}
    \hypo{\gtrans{A} \Rightarrow A}
    \infer2[Cut]{I \vdash \Rightarrow A}
  }
\]

\section{}

By Problems 2 and 3, the translation $\gtrans{-}$ embeds classical logic into intuitionistic logic \emph{conservatively}---indeed, the construction of the embedding implies that every classical connective can be simulated by intuitionistic-$\bot$, $\to$, $\wedge$, and $\forall$.
We can also support this view from a semantics perspective: every classical model is a trivial Kripke model.

% \bibliography{bib}
% \bibliographystyle{alpha}

\end{document}

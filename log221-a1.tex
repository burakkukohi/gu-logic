\documentclass[a4paper]{article}

\usepackage{preamble}

\newcommand{\seqarr}{\Rightarrow}
\newcommand{\cD}{\mathcal{D}}

\title{LOG 221 - A1}
\author{Frank Tsai}

\begin{document}

\maketitle

\section{}
For this question, we will consider any reasonable sequent calculus in which structural rules can be dispensed with.
We claim that the sequent $\Gamma,A \seqarr A$ is derivable with height at most $2w(A)$.
The structure of this proof is identical to that of \cite[Lemma 2.3.3]{Negri2001}.

We proceed by (strong) induction on $w(A)$; when $w(A) \leq 1$ there are three possible cases: $A = \bot$; $A = p$ for some atom $p$; or $A = \bot \to \bot$.
In the first case, $\Gamma,\bot \seqarr \bot$ is an instance of $L\bot$; in the second case, $\Gamma,p \seqarr p$ is an axiom.
In both cases, the given derivations have height $0$, which is clearly bounded by $2w(A) = 0$.
If $A = \bot \to \bot$, then consider the following derivation of height 1 (instances of L$\bot$ have height 0).
\[
  \begin{prooftree}
    \infer0[L$\bot$]{\Gamma,\bot \to \bot, \bot \seqarr \bot}
    \infer1[R$\to$]{\Gamma,\bot \to \bot \seqarr \bot \to \bot}
  \end{prooftree}
\]
The height of this derivation is clearly bounded by $2w(A) = 2$.

Now we show that $\Gamma,A \seqarr A$ has a derivation of height at most $2w(A)$ for $w(A) \leq n + 1$.
There are only three remaining cases to consider: $A = B \wedge C$; $A = B \vee C$; and $A = B \to C$.
In any of these cases, $w(B) \leq n$ and $w(C) \leq n$; hence, by the induction hypothesis, for any sequence $\Delta$, $\Delta,B \seqarr B$ and $\Delta,C \seqarr C$ or any permutation thereof are derivable with height at most $2w(B)$ and $2w(C)$ respectively.
Consider the following derivations:
\[
  \begin{prooftree}
    \hypo{\cD_1}
    \ellipsis{}{\Gamma,B,C \seqarr B}
    \infer1[L$\wedge$]{\Gamma,B \wedge C \seqarr B}
    \hypo{\cD_2}
    \ellipsis{}{\Gamma,B,C \seqarr C}
    \infer1[L$\wedge$]{\Gamma,B \wedge C \seqarr C}
    \infer2[R$\wedge$]{\Gamma,B \wedge C \seqarr B \wedge C}
  \end{prooftree}
  \qquad
  \begin{prooftree}
    \hypo{\cD_1}
    \ellipsis{}{\Gamma,B \seqarr B}
    \infer1[R$\vee$]{\Gamma,B \seqarr B \vee C}
    \hypo{\cD_2}
    \ellipsis{}{\Gamma,C \seqarr C}
    \infer1[R$\vee$]{\Gamma,C \seqarr B \vee C}
    \infer2[L$\vee$]{\Gamma,B \vee C \seqarr B \vee C}
  \end{prooftree}
\]
\[
  \begin{prooftree}
    \hypo{\cD_1}
    \ellipsis{}{\Gamma,B \seqarr B}
    \hypo{\cD_2}
    \ellipsis{}{\Gamma,C \seqarr C}
    \infer2[L$\to$]{\Gamma,B \to C, B \seqarr C}
    \infer1[R$\to$]{\Gamma,B \to C \seqarr B \to C}
  \end{prooftree}
\]
In any of these derivations, the height is $\max(\cD_1,\cD_2) + 2$, which is bounded by $\max(2w(B), 2w(C)) + 2$; this is in turn bounded by $2w(B) + 2w(C) + 2 = 2w(A)$.

\section{}
We assume admissibility of structural rules in the sequent calculus and choose to use the elimination rules in their \emph{general} form, as they admit normalization \cite[Section 1.2]{Negri2001}.

We proceed by induction on the number of inference rules in $\Gamma \vdash A$; as a result of this choice, the induction hypothesis applies to every premise of each rule being analyzed in the following case analysis.

If the number of inference rules is 0, then $A \in \Gamma$; thus $\Gamma \seqarr A$ is already an initial sequent, i.e., $I \vdash \Gamma \seqarr A$ trivially.

If the number of inference rules is $n+1$, we do a case analysis on the last applied rule:
\begin{itemize}
\item
  $
  \begin{prooftree}
    \hypo{\Gamma \vdash \bot}
    \infer1[$\bot$E]{\Gamma \vdash A}
  \end{prooftree}
  $:\qquad
  $
  \begin{prooftree}
    \hypo{\Gamma \seqarr \bot}
    \infer0[L$\bot$]{\Gamma,\bot \seqarr A}
    \infer2[cut]{\Gamma \seqarr A}
  \end{prooftree}
  $
\item
  $
  \begin{prooftree}
    \hypo{\Gamma \vdash B}
    \infer1[$\vee$I$_1$]{\Gamma \vdash B \vee C}
  \end{prooftree}
  $:\qquad
  $
  \begin{prooftree}
    \hypo{\Gamma \seqarr B}
    \infer1[R$\vee_1$]{\Gamma \seqarr B \vee C}
  \end{prooftree}
  $
\item
  $
  \begin{prooftree}
    \hypo{\Gamma \vdash C}
    \infer1[$\vee$I$_2$]{\Gamma \vdash B \vee C}
  \end{prooftree}
  $:\qquad
  $
  \begin{prooftree}
    \hypo{\Gamma \seqarr C}
    \infer1[R$\vee_2$]{\Gamma \seqarr B \vee C}
  \end{prooftree}
  $
\item
  $
  \begin{prooftree}
    \hypo{\Gamma \vdash B \vee C}
    \hypo{\Gamma,B \vdash A}
    \hypo{\Gamma,C \vdash A}
    \infer3[$\vee$E]{\Gamma \vdash A}
  \end{prooftree}
  $:\qquad
  $
  \begin{prooftree}
    \hypo{\Gamma \seqarr B \vee C}
    \hypo{\Gamma,B \seqarr A}
    \hypo{\Gamma,C \seqarr A}
    \infer2[L$\vee$]{\Gamma,B \vee C \seqarr A}
    \infer2[cut]{\Gamma \seqarr A}
  \end{prooftree}
  $
\item
  $
  \begin{prooftree}
    \hypo{\Gamma \vdash B}
    \hypo{\Gamma \vdash C}
    \infer2[$\wedge$I]{\Gamma \vdash B \wedge C}
  \end{prooftree}
  $:\qquad
  $
  \begin{prooftree}
    \hypo{\Gamma \seqarr B}
    \hypo{\Gamma \seqarr C}
    \infer2[R$\wedge$]{\Gamma \seqarr B \wedge C}
  \end{prooftree}
  $
\item
  $
  \begin{prooftree}
    \hypo{\Gamma \vdash B \wedge C}
    \hypo{\Gamma,B,C \vdash A}
    \infer2[$\wedge$E]{\Gamma \vdash A}
  \end{prooftree}
  $:\qquad
  $
  \begin{prooftree}
    \hypo{\Gamma \seqarr B \wedge C}
    \hypo{\Gamma,B,C \seqarr A}
    \infer1[L$\wedge$]{\Gamma,B \wedge C \seqarr A}
    \infer2[cut]{\Gamma \seqarr A}
  \end{prooftree}
  $
\item
  $
  \begin{prooftree}
    \hypo{\Gamma,B \vdash C}
    \infer1[$\to$I]{\Gamma \vdash B \to C}
  \end{prooftree}
  $:\qquad
  $
  \begin{prooftree}
    \hypo{\Gamma,B \seqarr C}
    \infer1[R$\to$]{\Gamma \seqarr B \to C}
  \end{prooftree}
  $
\item
  $
  \begin{prooftree}
    \hypo{\Gamma \vdash B \to C}
    \hypo{\Gamma \vdash B}
    \hypo{\Gamma,C \vdash A}
    \infer3[$\to$E]{\Gamma \vdash A}
  \end{prooftree}
  $:\qquad
  $
  \begin{prooftree}
    \hypo{\Gamma \seqarr B \to C}
    \hypo{\Gamma \seqarr B}
    \hypo{\Gamma,C \seqarr A}
    \infer2[L$\to$]{\Gamma,B \to C \seqarr A}
    \infer2[cut]{\Gamma \seqarr A}
  \end{prooftree}
  $
\end{itemize}

\section{}
Let $\cD$ be a normal derivation of $\Gamma \vdash A$; we show how to eliminate cut in the previous problem.
By the normal form property, the major premise of each elimination rule in $\cD$ is already an assumption.
Hence the corresponding cut formula is already in $\Gamma$, allowing us to replace every cut with a left contraction.

\bibliography{bib}
\bibliographystyle{alpha}
\end{document}

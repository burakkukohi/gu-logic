\documentclass[a4paper]{article}

\usepackage{preamble}
\usepackage{semdtt-macros}
\RequirePackage{showlabels}

\title{Category with families}
\author{Frank Tsai}

\begin{document}

\maketitle

These are my personal notes on \cite{Hofmann1997}.
Any mistake is my own.

\section{Introduction}\label{sec:intro}

Unlike simple types, dependent types, as the name suggests, depend on or vary with values.
The quintessential example of a dependent type is the vector type.
Similar to the list type, the vector type is a container of items; however, the length of the container is explicitly stored in the type.

Semantics is a compositional assignment of mathematical objects to syntactic objects; for instance, types and terms may be interpreted as sets and functions respectively.

The aim of \cite{Hofmann1997} is to develop an abstract notion of semantics that provides a framework upon which one can develop various interpretations of type theory: to obtain an interpretation is to check that one has an instance of the semantic notion.

\section{Dependent type theory}\label{sec:dtt}

A dependent type theory is presented by its \emph{judgments}; for instance, the typehood of $\sigma$ is expressed by the judgment $\typeJudg{\sigma}$ and the elementhood of $t$ in $\sigma$ is expressed by the judgment $t : \sigma$.
The elementhood of an open term depends on the types of its open variables; for instance, we cannot make the judgment $x + y : \bN$ unless we already know that $x : \bN$ and $y : \bN$.
Since types can depend on values, the typehood of an expression such as $\sigma(x)$ must also be made relative to a list of \emph{variable declarations}.
Such lists of declarations are called \emph{contexts}.

More complex notions of context exist in the literature, but for our purpose, a context is a list of well-formed variable declarations of the form $x_1 : \sigma_1, x_2 : \sigma_2, \ldots, x_n : \sigma_n$ where each type $\sigma_i$ is a type in the context $x_1 : \sigma_1,\ldots, x_{i-1} : \sigma_{i-1}$.
The judgment $\ctxJudg{\Gamma}$ expresses that $\Gamma$ is a well-formed context.

Finally, we have a notion of \emph{definitional equality} built into the theory; for example, we'd like to regard $0 : \bN$ and $0 + 0 : \bN$ as definitionally equal terms and $\typeJudg{\sigma(0)}$ and $\typeJudg{\sigma(0+0)}$ as definitionally equal types.
We express these equalities via the judgments $0 = 0 + 0 : \bN$ and $\typeJudg{\sigma(0) = \sigma(0+0)}$ respectively.

To summarize, there are six kinds of judgments:
\begin{align*}
  &\vdash \ctxJudg{\Gamma} && \text{$\Gamma$ is a well-formed context}&&\\
  \Gamma &\vdash \typeJudg{\sigma} && \text{$\sigma$ is a type in context $\Gamma$}&&\\
  \Gamma &\vdash t : \sigma && \text{$t$ is a term of type $\sigma$ in context $\Gamma$}&&\\
  &\vdash \ctxJudg{\Gamma = \Delta} && \text{$\Gamma$ and $\Delta$ are definitionally equal contexts}&&\\
  \Gamma &\vdash \typeJudg{\sigma = \tau} && \text{$\sigma$ and $\tau$ are definitionally equal types in context $\Gamma$}&&\\
  \Gamma &\vdash t = s : \sigma && \text{$t$ and $s$ are definitionally equal terms of type $\sigma$ in context $\Gamma$}&&\\
\end{align*}

\subsection{Dependent product types}\label{sec:dpt}
\subsection{Dependent sum types}\label{sec:dst}
\subsection{Identity types}\label{sec:it}
\subsection{Universes}\label{sec:u}
\subsection{Examples of type theories}\label{sec:eott}

Martin-L\"of's Logical Framework (LF) is a type theory with $\Pi$-types and a universe.

The Calculus of Constructions (CoC) is a type theory with $\Pi$-types and a universe closed under impredicative quantification.

\section{Pre-syntax and context morphisms}\label{sec:pscm}
\section{Category with families}\label{sec:cwf}
\subsubsection{Terms and sections}\label{sec:tas}
\subsubsection{Weakening}\label{sec:w}
\subsection{Semantic type formers}\label{sec:stf}
\subsubsection{The term model}\label{sec:ttm}
\subsubsection{The set model}\label{sec:tsm}
\subsubsection{The $\omega$-set model}\label{sec:twsm}
\subsubsection{Modest sets}\label{sec:ms}
\subsubsection{The presheaf model}\label{sec:tpm}
\subsection{Interpretation of the syntax}\label{sec:iots}
\subsection{Conservativity of the logical framework}\label{sec:cotlf}
\section{Other notions of semantics}\label{sec:onos}

\bibliography{semdtt}
\bibliographystyle{alpha}

\end{document}

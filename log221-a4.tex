\documentclass[a4paper]{article}
\usepackage[margin=1in]{geometry}

\usepackage{preamble}
\usepackage{log221-a4-macros}
\usepackage{showkeys}

\title{LOG 221 - A4}
\author{Frank Tsai}

\begin{document}

\maketitle

\section*{Problem 1}\label{0001}

\begin{lemma}\label{0000}
  If $\dVdash{n}{\ell}{k}{q} \Gamma \Rightarrow C,\Delta$ and $\dVdash{m}{\ell}{k}{q} \Lambda,C \Rightarrow \Pi$ with $\dRank{C} = \ell + 1$, $\cRank{C} < k$, and $\qRank{C} < q$, then $\dVdash{h}{\ell}{k}{q} \Gamma,\Lambda \Rightarrow \Delta,\Pi$, where $h \leq \max(n,m) + 2$.
\end{lemma}
\begin{proof}
  We proceed by case analysis on $C$.
  Since $\dRank{C} \geq 1$, we do not have to consider atoms, $\bot$, and $Qx.A(x)$.
  \begin{itemize}
  \item[$\lnot A$:] By inversion, we have $\dVdash{n}{\ell}{k}{q} \Gamma,A \Rightarrow \Delta$ and $\dVdash{m}{\ell}{k}{q} \Gamma \Rightarrow A, \Delta$.
    Since $\dRank{A} = \ell$, we can afford using \rCut{} to obtain the required sequent.
    The height of this derivation is bounded by $\max(n,m) + 1$.
  \item[$A \wedge B$:] By inversion, we have $\dVdash{n}{\ell}{k}{q} \Gamma \Rightarrow A,\Delta$; $\dVdash{n}{\ell}{k}{q} \Gamma \Rightarrow B,\Delta$; and $\dVdash{m}{\ell}{k}{q} \Lambda,A,B \Rightarrow \Pi$.
    Since $\dRank{A}$ and $\dRank{B}$ are bounded by $\ell$, we can afford two \rCut{}s to obtained the required sequent.
    \[
      \ebrule{
        \hypo{\Gamma \Rightarrow A,\Delta}
        \hypo{\Gamma \Rightarrow B,\Delta}
        \hypo{\Lambda,A,B \Rightarrow \Pi}
        \infer2[\rCut]{\Gamma,\Lambda,A \Rightarrow \Delta,\Pi}
        \infer2[\rCut]{\Gamma,\Lambda \Rightarrow \Delta,\Pi}
      }
    \]
    The height of this derivation is bounded by $\max(n,m) + 2$.
  \item[$A \vee B$:] This is analogous to the previous case: apply inversion on both assumptions; and then apply two \rCut{}s.
  \end{itemize}
\end{proof}

Observe that the proof of \cref{0000} goes through even if we weaken the assumption to allow $\cRank{C} = k$ since inversion yields strictly smaller formulas.

\section*{Problem 2}\label{0009}

\begin{lemma}\label{0004}
  If $\dVdash{n}{0}{k}{q} \Gamma \Rightarrow C,\Delta$ and $\dVdash{m}{0}{k}{q} \Lambda,C \Rightarrow \Pi$ with $\cRank{C} < k$ and $\qRank{C} = q$, then $\dVdash{h}{k}{k}{q} \Gamma,\Lambda \Rightarrow \Delta,\Pi$, where $h \leq n + m$.
\end{lemma}
\begin{proof}
  We proceed by induction on $n + m$; there are two cases to consider: (1) $C$ is principal in both assumptions, or (2) $C$ is not principal in at least one assumption.

  We treat the first case by case analysis on $C$.
  Note that $\bot$ does not apply in the case since it does not have a right rule; there are only three cases to consider:
  \begin{itemize}
  \item[Atom:] In this case we have $C \in \Gamma$ and $C \in \Pi$; hence $\dVdash{0}{0}{k}{q} \Gamma,\Lambda \Rightarrow \Delta,\Pi$ is also an instance of \rId.
  \item[$\forall x.A(x)$:] By inversion, we have $\dVdash{n}{0}{k}{q} \Gamma \Rightarrow A(s),\Delta$ for any $s$.
    The lack of left inversion means we must handle implicit contraction correctly.

    We apply the induction hypothesis to $\dVdash{n}{0}{k}{q} \Gamma \Rightarrow \forall x.A(x), \Delta$ and $\dVdash{m-1}{0}{k}{q} \Lambda,A(t),\forall x.A(x) \Rightarrow \Pi$ to eliminate $\forall x.A(x)$.
    This yields $\dVdash{n+m-1}{k}{k}{q} \Gamma,\Lambda,A(t) \Rightarrow \Delta,\Pi$.

    Since $\qRank{A(t)} < q$, we can afford to use \rCut{} to remove $A(t)$.
    Note that $\dRank{A(t)} \leq \cRank{\forall x.A(x)} < k$; hence the resulting derivation satisfies all the complexity bounds.
  \item[$\exists x.A(x)$:] Completely analogous to the previous case.
    This time, we lack right inversion, so we handle implicit contraction by applying the induction hypothesis.
  \end{itemize}
  In the latter case, we may assume, by symmetry of the argument, that $C$ is not principal in the first assumption.
  Every sequent calculus rule has the following general form:
  \begin{mathpar}
    \ebrule[*]{
      \hypo{\Gamma_1 \Rightarrow \Delta_1}
      \hypo{\Gamma_2 \Rightarrow \Delta_2}
      \infer2{\Gamma \Rightarrow \Delta}
    }
  \end{mathpar}
  Hence we have $\dVdash{n-1}{0}{k}{q} \Gamma_1 \Rightarrow C,\Delta_1$ and $\dVdash{n-1}{0}{k}{q} \Gamma_2 \Rightarrow C,\Delta_2$.
  Then by the induction hypothesis, we have $\dVdash{n+m-1}{k}{k}{q} \Gamma,\Lambda_1 \Rightarrow \Delta,\Pi_1$ and $\dVdash{n+m-1}{k}{k}{q} \Gamma,\Lambda_2 \Rightarrow \Delta,\Pi_2$.
  Now apply *.
\end{proof}

\begin{lemma}\label{0002}
  If $\dVdash{n}{\ell + 1}{k}{q} \Gamma \Rightarrow \Delta$, then $\dVdash{2n}{\ell}{k}{q} \Gamma \Rightarrow \Delta$.
\end{lemma}
\begin{proof}
  We proceed by induction on $n$; and we do a case analysis on the last applied rule.
  Most cases are uninteresting: apply the induction hypothesis to the premise(es) and then the corresponding rule.
  We present a few representative cases for completeness.
  The only interesting case is \rCut.
  \begin{itemize}
  \item[\rId:] In this case, there is no cut formula; hence we already have $\dVdash{2n}{\ell}{k}{q} \Gamma \Rightarrow \Delta$.
  \item[\rLBot:] Analogous to \rId.
  \item[\rRNot:] The induction hypothesis yields $\dVdash{2n - 2}{\ell}{k}{q} \Gamma, A \Rightarrow \Delta$.
    Then we obtain $\dVdash{2n - 1}{\ell}{k}{q} \Gamma \Rightarrow \lnot A, \Delta$ by applying \rRNot.
    This is clearly bounded by $2n$.
  \item[\rLDis:] The induction hypothesis yields $\dVdash{2n - 2}{\ell}{k}{q} \Gamma, A \Rightarrow \Delta$ and $\dVdash{2n - 2}{\ell}{k}{q} \Gamma, B \Rightarrow \Delta$, then we obtain $\dVdash{2n - 1}{\ell}{k}{q} \Gamma, A \vee B \Rightarrow \Delta$ by applying \rLDis.
  \item[\rCut:] Let $C$ be the cut formula.
    By the induction hypothesis, we have $\dVdash{2(n-1)}{\ell}{k}{q} \Gamma \Rightarrow C,\Delta$ and $\dVdash{2(n-1)}{\ell}{k}{q} \Gamma,C \Rightarrow \Delta$.
    There are two cases to consider: (1) $\dRank{C} < \ell + 1$, or (2) $\dRank{C} = \ell + 1$.
    In the former case, we do not have to alter anything since we can afford this \rCut, while in the latter case, we apply \cref{0000}, yielding $\dVdash{2n}{\ell}{k}{q} \Gamma \Rightarrow \Delta$.
  \end{itemize}
\end{proof}

By applying \cref{0002} recursively on $\ell$, we get \cref{0003}.

\begin{corollary}\label{0003}
  If $\dVdash{n}{\ell}{k}{q} \Gamma \Rightarrow \Delta$, then $\dVdash{2^{\ell}n}{0}{k}{q} \Gamma \Rightarrow \Delta$.
\end{corollary}

\begin{lemma}\label{0005}
  If $\dVdash{n}{0}{k}{q+1} \Gamma \Rightarrow \Delta$, then $\dVdash{(2^{k+1})^n}{k}{k}{q} \Gamma \Rightarrow \Delta$.
\end{lemma}
\begin{proof}
  Similar to \cref{0002}, most cases are uninteresting.
  We look at \rCut.
\end{proof}

% \bibliography{bib}
% \bibliographystyle{alpha}

\end{document}

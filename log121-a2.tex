\documentclass[a4paper]{article}

\usepackage{preamble}
\addbibresource{references/refs.bib}

\newcommand{\curry}{\mathsf{curry}}
\newcommand{\uncurry}{\mathsf{uncurry}}
\newcommand{\ran}{\mathsf{ran}}
\newcommand{\N}{\mathbb{N}}

\title{Assignment 2: Cardinality (Draft)}
\author{Frank Tsai}

\begin{document}

\maketitle

\section*{6.13}
\begin{enumerate}
\item
  \begin{proof}
    Consider the following functions
    \begin{align*}
      f(a,(b,c)) &= ((a,b),c)\\
      g((a,b),c) &= (a,(b,c))
    \end{align*}
    This pair of functions are inverses of each other since $f(g((a,b),c)) = f(a,(b,c)) = ((a,b),c)$ and $g(f(a,(b,c))) = g((a,b),c) = (a,(b,c))$, so we have the required bijection.
  \end{proof}
\item
  \begin{proof}
    Consider the following functions
    \begin{align*}
      \curry(f)(a)(b) &= f(a,b)\\
      \uncurry(g)(a,b) &= g(a)(b)
    \end{align*}
    This pair of functions are inverses of each other because $\curry(\uncurry(g))(a)(b) = \uncurry(g)(a,b) = g(a)(b)$.
    Functional extensionality implies that $\curry(\uncurry(g)) = g$.
    We also have $\uncurry(\curry(g))(a,b) = \curry(g)(a)(b) = g(a,b)$.
    Functional extensionality implies that $\uncurry(\curry(g)) = g$.
    We have the required bijection.
  \end{proof}
\end{enumerate}

\section*{6.33}
\begin{proof}
  There is an obvious injection, namely the inclusion $A \hookrightarrow A \cup B$, so $\N \preceq A \cup B$.
  It suffices to construct an injection $A \cup B \to \N$.
  Consider the following function
  \[h(c) =
    \begin{cases}
      2f(c) & \text{if $c \in A$,}\\
      2g(c) + 1 & \text{o.w.}
    \end{cases}
  \]
  blah blah blah so this is an injection.
\end{proof}

\section*{6.37}
\begin{proof}
  It suffices to rule out $X \prec \N$.
  Suppose that $X \prec \N$.
  Let $f\colon X \to \N$ be an injection witnessing this fact.
  Consider $\ran(f) \subseteq \N$.
  If $\ran(f)$ is infinite, then we have $X \approx \ran(f) \approx \N$.
  This is a contradiction.

  If $\ran(f)$ is finite, then $X$ is finite, but this contradicts the assumption that $X$ is infinite.
\end{proof}

%\printbibliography

\end{document}

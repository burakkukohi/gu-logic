\documentclass[a4paper]{article}

\usepackage{preamble}
\addbibresource{references/refs.bib}

\newcommand{\curry}{\mathsf{curry}}
\newcommand{\uncurry}{\mathsf{uncurry}}
\newcommand{\ran}{\mathsf{ran}}
\newcommand{\N}{\mathbb{N}}
\newcommand{\To}[3]{#1\colon#2\to#3}
\let\P\relax
\newcommand{\P}{\mathcal{P}}

\title{Assignment 2: Cardinality (Draft)}
\author{Frank Tsai}

\begin{document}

\maketitle

\section*{6.13}
\begin{enumerate}
\item
  \begin{proof}
    Consider the following functions
    \begin{align*}
      f(a,(b,c)) &= ((a,b),c)\\
      g((a,b),c) &= (a,(b,c))
    \end{align*}
    By direct computation, we have \[f(g((a,b),c)) = f(a,(b,c)) = ((a,b),c)\] and \[g(f(a,(b,c))) = g((a,b),c) = (a,(b,c)),\] so we have the required bijection.
  \end{proof}
\item
  \begin{proof}
    Consider the following functions
    \begin{align*}
      \curry(f)(a)(b) &= f(a,b)\\
      \uncurry(g)(a,b) &= g(a)(b)
    \end{align*}
    By direct computation, we have \[\curry(\uncurry(g))(a)(b) = \uncurry(g)(a,b) = g(a)(b)\] and \[\uncurry(\curry(f))(a,b) = \curry(f)(a)(b) = f(a,b).\]
    Then functional extensionality implies that we have the required bijection.
  \end{proof}
\end{enumerate}

\section*{6.33}
\begin{proof}
  Let $\To{i}{A}{\N}$ and $\To{j}{B}{\N}$ be two injections witnessing the fact that $A \preceq \N$ and $B \preceq \N$ respectively.
  Consider the function $\To{f}{A \cup B}{\N}$ defined by \[
    f(x) =
    \begin{cases}
      2 \cdot i(x) & \text{if $x \in A$,}\\
      2 \cdot j(x) + 1 & \text{otherwise.}
    \end{cases}
  \]
  Suppose that $f(x) = f(x')$.
  If both $x, x' \in A$, then $x = x'$ because $2 \cdot -$ and $i$ are injective.
  Similarly, if $x, x' \in B \setminus A$, then $x = x'$ because $2 \cdot -$, $j$, and $- + 1$ are all injective.
  Only the interesting cases remain.
  Without loss of generality, assume that $x \in A$ and $x' \in B \setminus A$.
  Then by assumption, $2 \cdot i(x) = 2 \cdot j(x') + 1$.
  This is not possible because an even number never equals an odd number.
  Thus, $f$ is an injection witnessing $A \cup B \preceq \N$, i.e., $A \cup B$ is countable.
\end{proof}

\section*{6.37}
\begin{proof}
  It suffices to rule out $X \prec \N$.
  Let $\To{f}{X}{\N}$ be an injection witnessing $X \prec \N$.
  Note that $f$ restricts to a bijection $\To{f'}{X}{\ran(f)}$ because it is injective, i.e., $X \approx \ran(f)$.

  Since $\ran(f) \subseteq \N$, it is either finite or $\ran(f) \approx \N$.
  The former case implies that $X$ is finite, which is a contradiction.
  The latter case implies that $X \approx \N$, which is also a contradiction.
\end{proof}

\section*{6.55}
\begin{lemma}\label{thm:no-fixed-points}
  For any natural number $n \geq 2$, there is a function $\To{\overline{(-)}}{n}{n}$ with no fixed points.
\end{lemma}
\begin{proof}
  By induction on $n$.
  In the base case, consider the function defined by
  \begin{align*}
    \overline{0} &= 1\\
    \overline{1} &= 0\\
  \end{align*}
  This is clearly a function with no fixed points.

  In the induction step, we have a function $\To{\overline{(-)}}{k}{k}$ with no fixed points.
  Consider $\To{\overline{(-)}_{k+1}}{k+1}{k+1}$ defined as follows.
  \[
    \overline{m}_{k+1} =
    \begin{cases}
      \overline{m} & \text{if $m < k$,}\\
      0 & \text{otherwise.}
    \end{cases}
  \]
  Let $m \in k + 1$.
  If $m < k$ then it cannot be a fixed point because $\overline{(-)}$ does not have any fixed point.
  If $m = k$ then $\overline{k}_{k+1} = 0$.
  Since $k$ is at least $2$, it cannot be a fixed point either.
\end{proof}

\begin{enumerate}
\item This is not true: let $X$ be any singleton set; $X \approx X \times X$ but $X$ is clearly not infinite.
  Let us additionally assume that $X$ is not the singleton set.
  We show that $X$ is infinite.
  \begin{proof}
    If $X$ is finite, then $X \approx n$ for some natural number $n$.
    Thus, $n \times n \approx X \times X \approx X \approx n$.
    We show that this is impossible for any $n \geq 2$.

    Suppose that there is a bijection $\To{f}{n}{n \times n}$.
    Let $\To{\hat{f}}{n}{\underbrace{n \times \cdots \times n}_{n~\text{times}}}$ be the composite of the following $n - 1$ bijections.
    % https://q.uiver.app/#q=WzAsNSxbMCwwLCJuIl0sWzEsMCwibiBcXHRpbWVzIG4iXSxbMiwwLCJuIFxcdGltZXMgbiBcXHRpbWVzIG4iXSxbMywwLCJcXGNkb3RzIl0sWzQsMCwiXFx1bmRlcmJyYWNle24gXFx0aW1lcyBcXGNkb3RzIFxcdGltZXMgbn1fe25+XFx0ZXh0e3RpbWVzfX0iXSxbMCwxLCJmIl0sWzEsMiwiMSBcXHRpbWVzIGYiXSxbMiwzLCIxIFxcdGltZXMgKDEgXFx0aW1lcyBmKSJdLFszLDRdXQ==
    \[\begin{tikzcd}
	n & {n \times n} & {n \times n \times n} & \cdots & {\underbrace{n \times \cdots \times n}_{n~\text{times}}}
	\arrow["f", from=1-1, to=1-2]
	\arrow["{1 \times f}", from=1-2, to=1-3]
	\arrow["{1 \times (1 \times f)}", from=1-3, to=1-4]
	\arrow[from=1-4, to=1-5]
      \end{tikzcd}\]
    By \cref{thm:no-fixed-points}, there is a function $\To{\overline{(-)}}{n}{n}$ with no fixed points.
    Consider the pair $\delta := (\overline{f(0).0},\overline{f(1).1},\ldots,\overline{f(n-1).n-1})$, where $(-).i$ is the $i$-th projection.

    Suppose that $f(k) = \delta$.
    Then $f(k).k = \delta.k = \overline{f(k).k}$, but this is a fixed point of $\overline{(-)}$ so we have a contradiction.
  \end{proof}
\item
  \begin{proof}
    If $X$ is $\varnothing$ the statement is trivial: there is a unique function from $0$ to any set.
    This statement is not true for singleton sets since $2^{1} \approx 2$, but there is a unique function from any set to the singleton set, i.e., $1^{1} \approx 1$.
    
    If $X$ is infinite, there is an evident injection $2^{X} \to X^{X}$.
    Consider the function $\To{j}{X^{X}}{\P(X \times X)}$ defined by \[j(f) = \{(x,f(x)) \mid x \in X\}.\]
    If $j(f) = j(g)$ then for any $x \in X$, we have $(x,f(x)) = (x,g(x))$, so $f(x) = g(x)$.
    Then functional extensionality implies that $f = g$.
    Thus, the following composition gives the required injection.
    % https://q.uiver.app/#q=WzAsNCxbMCwyLCJcXFAoWCBcXHRpbWVzIFgpIl0sWzEsMiwiXFxQKFgpIl0sWzIsMiwiMl57WH0iXSxbMCwwLCJYXntYfSJdLFszLDAsImoiLDIseyJzdHlsZSI6eyJ0YWlsIjp7Im5hbWUiOiJtb25vIn19fV0sWzAsMSwiXFxhcHByb3giLDEseyJzdHlsZSI6eyJib2R5Ijp7Im5hbWUiOiJub25lIn0sImhlYWQiOnsibmFtZSI6Im5vbmUifX19XSxbMSwyLCJcXGFwcHJveCIsMSx7InN0eWxlIjp7ImJvZHkiOnsibmFtZSI6Im5vbmUifSwiaGVhZCI6eyJuYW1lIjoibm9uZSJ9fX1dLFszLDIsIiIsMSx7InN0eWxlIjp7InRhaWwiOnsibmFtZSI6Im1vbm8ifSwiYm9keSI6eyJuYW1lIjoiZGFzaGVkIn19fV1d
    \[\begin{tikzcd}
	{X^{X}} \\
	\\
	{\P(X \times X)} & {\P(X)} & {2^{X}}
	\arrow["j"', tail, from=1-1, to=3-1]
	\arrow[dashed, tail, from=1-1, to=3-3]
	\arrow["\approx"{description}, draw=none, from=3-1, to=3-2]
	\arrow["\approx"{description}, draw=none, from=3-2, to=3-3]
      \end{tikzcd}\]
  \end{proof}
\item
  \begin{proof}
    This is not true if $X$ is the singleton set.
    Indeed, $1 + 1 \approx 2$.

    The statement is trivial when $X = \varnothing$.

    If $X$ is infinite.
    Let $\To{i}{2}{X}$ be any injection.
    Define $\To{j}{X + X}{X \times X}$ by $j(x,0) = (x,i(0))$ and $j(x,1) = (x,i(1))$.
  \end{proof}
\end{enumerate}

% \printbibliography

\end{document}

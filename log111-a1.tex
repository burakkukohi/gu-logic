\documentclass[a4paper]{article}

\usepackage{preamble}
\addbibresource{references/refs.bib}

\title{LOG 111 Hand-in 1}
\author{Frank Tsai (gustsafe)}

\begin{document}

\maketitle

\section{Prop 1.6}
\begin{proof}
  Let us proceed by induction.
  \begin{itemize}
  \item Case $p$: In this case, $p$ is a propositional variable.
    Its only proper initial segment is the empty string $\varepsilon$, which is not a formula.

  \item Case $\lnot\varphi$: Let $\sigma$ be a proper initial segment of $\lnot\varphi$.
    The $\varepsilon$ case is immediate.
    In the non-$\varepsilon$ case, suppose that $\sigma \equiv \lnot\xi$ and that it is a formula.
    Then it follows that $\xi$ is a formula by inverting the inductive definition.
    Since $\xi$ is a proper initial segment of $\varphi$, this contradicts the induction hypothesis.

  \item Case $(\varphi \star \psi)$: In this case, $\star$ denotes any binary connective.
    The $\varepsilon$ case is immediate.
    Suppose that $\sigma$ is a proper initial segment of $(\varphi \star \psi)$ and that $\sigma$ is a formula.
    Then by Prop 1.5, $\sigma$ is balanced.
    This rules out $\sigma \equiv (\varphi$ and $\sigma \equiv (\varphi \star \psi$ because $\varphi$ and $\psi$ are also balanced.
    Thus, there are only two cases to consider:
    \begin{itemize}
    \item Case $\sigma \equiv (\sigma'$: In this case, $\sigma'$ is a proper initial segment of $\varphi$.
      By inversion, $\sigma' \equiv \xi \star' \zeta$, where $\star'$ is any binary connective.
      Note that $\xi$ is a proper initial segment of $\varphi$.
      This contradicts the induction hypothesis.
    \item Case $\sigma \equiv (\varphi \star \sigma'$: In this case, $\sigma'$ is a proper initial segment of $\psi$.
      By inversion, $\sigma' \equiv \xi)$.
      The rest of the argument mirrors the previous case. \qedhere
    \end{itemize}
  \end{itemize}
\end{proof}

\section{Prob 1.3}
Let $\varphi$ denote the given formula in each sub-problem.
\begin{enumerate}
\item \[\varphi \equiv p_{0}[\varphi/p_{0}].\]
\item
  \begin{align*}
    \varphi &\equiv p_{0}[\varphi/p_{0}]\\
    \varphi &\equiv (\lnot p_{0} \wedge p_{1})[p_{0}/p_{1}].
  \end{align*}
\item \[\varphi \equiv p_{0}[\varphi/p_{0}].\]
\item \[\varphi \equiv p_{0}[\varphi/p_{0}].\]
\item
  \begin{align*}
    \varphi &\equiv p_{0}[\varphi/p_{0}]\\
    \varphi &\equiv ((\lnot p_{0} \to p_{1}) \wedge p_{2})[(p_{0} \to p_{1})/p_{0}, (p_{0} \vee p_{1})/p_{1}, \lnot(p_{0} \wedge p_{1})/p_{2}].
  \end{align*}
\end{enumerate}

\section{Prob 1.6}

\begin{tabular}[t]{ccc|c}
  $\varphi$ & $\psi$ & $\chi$ & $\diamond(\varphi,\psi,\chi)$\\
  T & T & T & T\\
  T & T & F & T\\
  T & F & T & F\\
  T & F & F & F\\
  F & T & T & T\\
  F & T & F & F\\
  F & F & T & T\\
  F & F & F & F\\
\end{tabular}

\section{Prob 1.13}
\begin{itemize}
\item[] DNF: $p_{0} \wedge \lnot p_{1} \wedge p_{2}$
\item[] CNF:
  \begin{align*}
    (p_{0} \vee p_{1} \vee p_{2}) &\wedge\\
    (p_{0} \vee p_{1} \vee \lnot p_{2}) &\wedge\\
    (p_{0} \vee \lnot p_{1} \vee \lnot p_{2}) &\wedge\\
    (p_{0} \vee \lnot p_{1} \vee p_{2}) &\wedge\\
    (\lnot p_{0} \vee p_{1} \vee p_{2}) &\wedge\\
    (\lnot p_{0} \vee \lnot p_{1} \vee p_{2}) &\wedge\\
    (\lnot p_{0} \vee \lnot p_{1} \vee \lnot p_{2})
  \end{align*}
\end{itemize}

\section{}
A three-valued semantics can describe the local view of a process in a distributed system, where the valuations of some propositional variables cannot be locally determined because they relay on an external process.

\begin{tabular}[t]{c|c}
  $\varphi$ & $\lnot \varphi$\\
  T & F\\
  F & T\\
  U & U\\
\end{tabular}
\qquad
\begin{tabular}[t]{cc|c}
  $\varphi$ & $\psi$ & $\varphi \vee \psi$\\
  T & T & T\\
  T & F & T\\
  T & U & T\\
  F & T & T\\
  F & F & F\\
  F & U & U\\
  U & T & T\\
  U & F & U\\
  U & U & U\\
\end{tabular}
\qquad
\begin{tabular}[t]{cc|c}
  $\varphi$ & $\psi$ & $\varphi \wedge \psi$\\
  T & T & T\\
  T & F & F\\
  T & U & U\\
  F & T & F\\
  F & F & F\\
  F & U & F\\
  U & T & U\\
  U & F & F\\
  U & U & U\\
\end{tabular}
\qquad
\begin{tabular}[t]{cc|c}
  $\varphi$ & $\psi$ & $\varphi \to \psi$\\
  T & T & T\\
  T & F & F\\
  T & U & U\\
  F & T & T\\
  F & F & T\\
  F & U & T\\
  U & T & T\\
  U & F & U\\
  U & U & U\\
\end{tabular}

The Law of Excluded Middle is not a tautology in this semantics because $U \vee \lnot U = U$, but every tautology in this semantics is automatically a tautology in classical logic (by deleting rows with $U$).

% \printbibliography

\end{document}
